\chapter{Autómatas Finitos y Expresiones Regulares}

\section{Autómatas Finitos Determinísticos}

\subsection{Definición}

Un \textbf{autómata finito determinístico} (AFD) $M$ es una 5-upla $\langle Q, \Sigma, \delta, q_0, F \rangle$ donde:
\begin{itemize}
    \item $Q$ es el conjunto finito de \textbf{estados del autómata}.
    \item $\Sigma$ es el \textbf{alfabeto de entrada}.
    \item $\delta: Q \times \Sigma \to Q$ es la \textbf{función de transición}: si el autómata se encuentra en el estado $q \in Q$ y lee el símbolo $a \in \Sigma$ de la entrada, pasa al estado $\delta(q, a)$.
    \item $q_0 \in Q$ es el \textbf{estado inicial}.
    \item $F \subseteq Q$ es el conjunto de \textbf{estados finales}/\textbf{estados de aceptación}.
\end{itemize}


Estos autómatas toman una cadena de entrada $w \in \Sigma^*$ e iteran por los símbolos de la misma. Empezando en el estado inicial, cada símbolo indica el siguiente estado a tomar. Si, al terminar de leer la entrada, el autómata se encuentra en un estado final, la cadena es \textbf{aceptada}. De lo contrario, es \textbf{rechazada}.

\subsubsection{Visualización}

Estos autómatas se suelen representar como \textbf{grafos} $G = \langle V, E \rangle$, donde:
\begin{itemize}
    \item Los \textbf{nodos} representan los estados del autómata ($V = Q$).
    \item Las \textbf{aristas} están dadas por la función de transición: $(v, w) \in E \iff \exists a \in \Sigma \mid \delta(v, a) = w$. Además, las aristas se etiquetan con los símbolos por los cuales se puede tomar la transición.
    \item El estado inicial se representa con una arista entrante que no viene de otro estado.
    \item Los estados finales se respresentan con doble borde.
\end{itemize}

\begin{figure}[H]
    \centering
    \begin{tikzpicture}
        \node[state, initial] (0) {$q_0$};
        \node[state, accepting, right of=0] (1) {$q_1$};
        \node[state, below right of=0] (2) {$q_2$};

        \draw   (0) edge[above] node{$a$} (1)
                (0) edge[above] node{$b$} (2)
                (1) edge[below, bend left, right=0.3] node{$a,b$} (2)
                (2) edge[above, bend left, left=0.3] node{$b$} (1)
                (2) edge[loop right] node{$a$} (2);
    \end{tikzpicture}
    \caption*{Visualización del autómata $M = \langle \{q_0, q_1, q_2\}, \{a, b\}, \delta, q_0, \{q_1\} \rangle$, donde la función de transición $\delta$ está dada por las flechas entre los estados.}
\end{figure}

\subsection{Configuraciones Instantáneas}

La ejecución de los AFDs se puede formalizar usando el concepto de las \textbf{configuraciones instantáneas}: $\langle q, s \rangle \in Q \times \Sigma^*$ representa el punto de la ejecución del autómata en el que se encuentra en el estado $q$ y tiene a $s$ como entrada restante.

La función de transición $\delta$ de un autómata $M$ se puede adaptar a una relación entre configuraciones instantáneas, la \textbf{relación de transición} $\vdash_M$:
$$
(q, a \cdot w) \vdash_M (r, w) \iff \delta(q, a) = r
$$
Notemos que, en el caso de los autómatas finitos, esta relación es una \textbf{función} (para cada $(q, s)$, hay un único $(q', s')$ tal que $(q, s) \vdash_m (q', s')$) y es \textbf{total}, excepto por los pares $(q, \lambda)$.

Luego, el \textbf{lenguaje aceptado} por el autómata $M$, denotado $\L(M)$ se puede definir como:
$$
\L(M) := \{ w \in \Sigma^* \mid \exists q_f \in F : (q_0, w) \vdash^* (q_f, \lambda) \}
$$

\subsection{Función de Transición Extendida}

Una forma alternativa (y análoga) de dar el lenguaje aceptado es por medio de la \textbf{función de transición extendida}. Esta es una función $\hat \delta : Q \times \Sigma^* \to \mathcal P (Q)$ (toma cadenas en vez de símbolos) definida de la siguiente manera:
$$
\begin{aligned}
    \hat \delta(q, \lambda) & := \{q\} \\
    \hat \delta(q, w \cdot a) & := \delta (\hat \delta (q, w), a)
\end{aligned}
$$
Donde $q \in Q, a \in \Sigma, w \in \Sigma^*$.

Bajo esta definición el conjunto de estados aceptados es simplemente:
$$
\L (M) := \{w \in \Sigma^* \mid \hat \delta (q_0, w) \in F \}
$$
