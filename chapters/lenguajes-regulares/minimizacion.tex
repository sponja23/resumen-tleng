\section{Minimización de Autómatas Finitos}

Para cualquier autómata finito determinístico, hay uno que reconoce el mismo lenguaje y tiene la mínima cantidad de estados. Dicho de otra forma, para cualquier lenguaje regular existe un autómata finito determinístico \textbf{mínimo} que lo reconoce. En el proceso de demostrar esto, daremos un algoritmo para encontrar el

\subsection{Indistinguibilidad}

Sea $M = \langle Q, \Sigma, \delta, q_0, F \rangle$ un autómata finito determinístico. Luego, se puede definir la \textbf{relación de indistinguibilidad} entre estados del autómata $\mathop\equiv \subseteq Q \times Q$, de la siguiente manera:
$$
    q \equiv r \iff (\forall \alpha \in \Sigma^*, \hat \delta(q, \alpha) \in F \iff \hat \delta(r, \alpha) \in F)
$$

Es decir, dos estados son indistinguibles cuando las mismas cadenas terminan en estados finales al comenzar su ejecución en esos estados.

\label{estados-indistinguibles-lema-cadenas}
\begin{lemma*}
    Todo par de estados indistinguibles en un autómata $M = \langle Q, \Sigma, \delta, q_0, F \rangle$, al consumir una cadena $\alpha$, llega a otro par de estados indistinguibles. Es decir,
    $$
        q \equiv r \implies \forall \alpha \in \Sigma^*, \hat \delta(q, \alpha) \equiv \hat \delta(r, \alpha)
    $$
\end{lemma*}
\begin{proof}
    La demostración será por el absurdo: supongamos que $q \equiv r$, pero existe una cadena $\alpha \in \Sigma^*$ tal que $\hat \delta(q, \alpha) \not\equiv \hat \delta(r, \alpha)$. Esto implica que hay una cadena $\beta \in \Sigma^*$ que \textbf{distingue} a $\hat \delta(q, \alpha)$ de $\hat \delta(r, \alpha)$.

    Supongamos, sin pérdida de generalidad, que $\beta$ cumple:

    $$
        \hat \delta(\hat \delta(q, \alpha), \beta) \in F \land \hat \delta(\hat \delta(r, \alpha), \beta) \notin F
    $$

    Luego, por la definición de $\hat \delta$, esto quiere decir que:
    $$
        \hat \delta(q, \alpha \cdot \beta) \in F \land \hat \delta(r, \alpha \cdot \beta) \notin F
    $$

    Por ende, $q \not\equiv r$. Esto es absurdo.
\end{proof}

\subsection{Estados Inaccesibles}

Los \textbf{estados accesibles} de un AFD $M = \langle Q, \Sigma, \delta, q_0, F \rangle$ son aquellos alcanzables desde el estado inicial. Es decir, $q \in Q$ es accesible cuando:
$$
    \exists \alpha \in \Sigma^* \mid \hat \delta(q_0, \alpha) = q
$$

Es claro que estos estados pueden ser eliminados sin cambiar el lenguaje aceptado por $M$. Esta es una forma fácil de reducir la cantidad de estados del mismo. El siguiente algoritmo computa el conjunto de estados accesibles de un autómata por medio de DFS:



\begin{algorithm}[H]
    \caption{Estados accesibles de un AFD}
    \label{accesibles}
    \begin{algorithmic}[1]
        \Procedure{EstadosAccessibles}{$\langle Q, \Sigma, \delta, q_0, F \rangle$}
        \State $accesibles \gets \{q_0\}$
        \State $nuevos \gets \{q_0\}$
        \Repeat
        \State $temp \gets \emptyset$
        \ForAll{$q \in nuevos$}
        \ForAll{$c \in \Sigma$}
        \State $temp \gets temp \cup \{\delta(q, c)\}$
        \EndFor
        \EndFor
        \State $nuevos \gets temp \setminus accesibles$
        \State $accesibles \gets accesibles \cup nuevos$
        \Until{{$nuevos = \emptyset$}}
        \State \Return $accesibles$
        \EndProcedure
    \end{algorithmic}
\end{algorithm}

\subsection{Definición de AFD Mínimo}

Dado un autómata $M = \langle Q, \Sigma, \delta, q_0, F \rangle$, un AFD mínimo equivalente es uno definido como $M_{min} = \langle Q_{min}, \Sigma, \delta_{min}, q_{min_0}, F_{min} \rangle$, donde:

\begin{itemize}
    \item $Q_{min} := (Q / \mathop\equiv)$ (las clases de equivalencia de $\addcontentsline{file}{secunit}{entry}equiv$)
    \item $\delta_{min} := [\delta(q, a)]$ ($[q]$ es la clase de equivalencia del estado $q$)
    \item $q_{min_0} := [q_0]$
    \item $F_{min} = \{[q] \mid q \in F\}$
\end{itemize}

\subsubsection{Demostración de Equivalencia}

Para ver que el autómata es efectivamente equivalente, veamos la siguiente propiedad:

\begin{lemma*}
    Si $\hat \delta(q, \alpha) = r$, entonces $\hat\delta_{min}([q], \alpha) = [r]$
\end{lemma*}
\begin{proof}
    Será por inducción en la longitud de $\alpha$
    \begin{itemize}
        \item \textbf{Caso Base}: Si $|\alpha| = 0$, $\alpha = \lambda$. Entonces, se puede ver que $\hat \delta(q, \lambda) = q$ (por definición de $\hat \delta$), y que $\hat\delta_{min}([q], \lambda) = [q]$ (por definición de $\hat\delta_{min}$).
        \item \textbf{Paso Inductivo}: Si $|\alpha| > 0$, la cadena es de la forma $\alpha = \beta a$. Supongamos que $\hat \delta(q, \beta a) = r$. Por definición de $\hat \delta$, esto significa que debe haber algún estado $s \in Q$ tal que:
              $$\hat \delta(q, \beta) = s \land \delta(s, a) = r$$

              Pero entonces, por hipótesis inductiva,
              $$\hat\delta_{min}([q], \beta) = [s]$$

              Por otro lado, si un estado $t$ es indistiguible de $s$, se tiene que
              $$t \equiv s \implies \hat\delta(s, a) = r \equiv \hat\delta(t, a)$$

              Esto es equivalente a decir que la clase de equivalencia $[s]$ cumple:
              $$
                  \hat\delta_{min}([s], a) = [r]
              $$

              Luego, por la definición de $\hat\delta_{min}$:
              $$
                  \hat\delta_{min}([q], \beta) = [s] \land \hat\delta_{min}([s], a) = [r] \implies \hat\delta_{min}([q], \beta a) = [r]
              $$
    \end{itemize}
\end{proof}

Ahora la demostración de equivalencia es bastante directa:

\begin{theorem*}
    Dado un AFD $M$, su autómata mínimo correspondiente $M_{min}$ es equivalente, es decir, $\L(M) = \L(M_{min})$.
\end{theorem*}
\begin{proof}
    Sea $M = \langle Q, \Sigma, \delta, q_0, F \rangle$. Demostremos la doble inclusión:
    \begin{itemize}
        \item $\L(M) \subseteq \L(M_{min})$: Una cadena está en $\L(M)$ cuando
              $$
                  w \in \L(M) \iff \hat\delta(q_0, w) \in F \iff \exists q_f \in F \mid \hat\delta(q_0, w) = q_f
              $$

              Por el lema anterior, esto implica que
              $$\exists q_f \in F \mid \hat\delta_{min}([q_0], w) = [q_f] \iff \hat\delta_{min}([q_0], w) \in F_{min} \iff w \in \L(M_{min})$$
        \item Por otro lado, también se tiene:
              $$
                  w \in \L(M_{min}) \iff \hat\delta_{min}([q_0], w) \in F_{min} \implies
                  \exists q_f \in F \mid \hat\delta(q_0, w) \equiv q_f
              $$

              Pero la única forma de que un estado sea equivalente a uno final es que éste mismo también lo sea, porque $q \equiv r \implies (\hat\delta(q, \lambda) \in F \iff \hat\delta(r, \lambda) \in F)$, es decir, $q \in F \iff r \in F$. Esto quiere decir que:
              $$
                  \hat\delta(q_0, w) \equiv q_f \iff \hat\delta(q_0, w) \in F \iff w \in F
              $$
    \end{itemize}
\end{proof}

\subsubsection{Demostración de Minimalidad}

Ahora, veamos que $M_{min}$ es el autómata equivalente con la menor cantidad de estados posibles. Para ello, primero usamos el siguiente lema:

\begin{lemma*}
    Sean $M = \langle Q, \Sigma, \delta, q_0, F \rangle$ y $M' = \langle Q', \Sigma, \delta', q_0', F' \rangle$, donde $M$ no posee estados inaccesibles. Si $|Q| > |Q'|$, hay algún par de cadenas $w_1, w_2 \in \Sigma$ tal que $\hat\delta(q_0, w_1) \neq \hat\delta(q_0, w_2)$ y $\hat\delta'(q_0', w_1) = \hat\delta'(q_0', w_2)$.
\end{lemma*}
\begin{proof}
    \textbf{TODO}
\end{proof}

