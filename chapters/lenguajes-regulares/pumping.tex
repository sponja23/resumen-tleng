\section{Lema de Pumping}

El \textbf{lema de pumping} es una propiedad que cumplen todos los lenguajes regulares. Se puede enunciar de la siguiente manera:

\begin{theorem*}
    Sea $\L$ un lenguaje regular. Luego, existe una longitud $p \in \N$ tal que cualquier cadena $z$ con longitud $|z| \geq p$ se puede \textbf{factorizar} de la siguiente manera:
    $$
        z = u v w
    $$
    Donde las subcadenas cumplen las siguientes propiedades:
    $$
        \begin{aligned}
            |uv|    & \leq p                    \\
            |v|     & \geq 1                    \\
            u v^i w & \in \L \ \forall i \geq 0
        \end{aligned}
    $$
\end{theorem*}

En otras palabras, para cualquier lenguaje regular existe una longitud (denominada \textbf{constante de pumping}\footnote{Es importante notar que, bajo esta definición, la constante de pumping no es única: es fácil ver que, si $p$ es constante de pumping para un lenguaje $\L$, cualquier $m > p$ también lo es. Lo que sí es único para cada lenguaje es su \textbf{mínima} constante de pumping.}) a partir de la cual las cadenas se pueden ``bombear'', es decir, descomponer en prefijo $\cdot$ ``parte bombeable'' $\cdot$ sufijo, y cualquier cadena que resulte de repetir la parte bombeable también pertenecerá al lenguaje.

\subsection{Demostración}

\begin{proof}
    Para demostrar el lema de pumping, es convienente tomar un autómata finito determinístico $M = \langle Q, \Sigma, \delta, q_0, F \rangle$ que reconoce $\L$, es decir, que cumple $\L(M) = \L$. Demostraremos que una posible constante de pumping es precisamente la cantidad de estados del autómata $p = |Q|$.

    Sea $z \in \L(M)$ una cadena con longitud $m \geq p$. Recordemos que una forma de definir el lenguaje aceptado por $M$ es:
    $$
        z \in \L(M) \iff \exists q_f \in F, (q_0, z) \vdash^* (q_f, \lambda)
    $$

    Por ende, sea $q_f$ el estado final en el que termina $M$ al consumir $z$. Por otro lado, según la \hyperref[subsubsec-propiedades-rel-transicion]{propiedad de linealidad} de la relación de transición:
    $$
        |z| = m \implies (q_0, z) \vdash^m (q_f, \lambda)
    $$

    Es decir, el autómata debe pasar por $m$ transiciones hasta llegar al estado final, lo cual implica que pasa por $m + 1$ estados. Como hay sólo $p < m + 1$ estados, el autómata debe pasar 2 veces por algún estado (debido al Principio del Palomar). Más aún, consideremos el prefijo $\alpha$ de $z$ de longitud $p$ ($z = \alpha \cdot \beta$): su ejecución por el autómata también debe repetir algún estado, porque pasa por $p + 1$ estados.

    Tomemos la cadena de configuraciones instantáneas empezando en $(q_0, \alpha)$, con la notación $\alpha = a_0 \cdots a_p$:
    $$
        (q_0, a_0 \cdots a_p) \vdash (r_1, a_1 \cdots a_p) \vdash \cdots \vdash (r_i, a_i \cdots a_p) \vdash \cdots \vdash (r_i, a_j \cdots a_p) \vdash \cdots \vdash (r_k, \lambda)
    $$

    Como mencionamos, el autómata debe pasar 2 veces por algún estado: en este caso ese estado se denota $r_i$. Luego, llamemos:
    $$
        \begin{aligned}
            u & := a_0 \cdots a_{i - 1}       \\
            v & := a_i \cdots a_{j - 1}       \\
            w & := a_j \cdots a_p \cdot \beta
        \end{aligned}
    $$

    Se puede ver que $uvw = a_0 \cdots a_p \cdot \beta = \alpha \cdot \beta = z$. Ahora, demostremos las propiedades del lema de pumping:
    \begin{itemize}
        \item $|uv| = j \leq p$: Trivial, por construcción.
        \item $|v| \geq 1$: No se podría tener $|v| = 0$, porque la cadena de transiciones $(r_i, a_i \cdots a_p) \vdash \cdots \vdash (r_i, a_j \cdots a_p)$ no puede estar vacía.
        \item $u v^i w \in \L \ \forall i \geq 0$: Gracias a la propiedad de invarianza, se tiene que:
              $$
                  (r_i, v) \vdash (r_i, \lambda) \implies (r_i, v^m) \vdash (r_i, \lambda) \ \forall m \geq 0
              $$
              Por ende, para cualquier natural $m \geq 0$, se tiene:
              $$
                  (q_0, u \cdot v^m \cdot w) \vdash^* (r_i, v^m \cdot w) \vdash^* (r_i, w) \vdash^* (q_f, \lambda)
              $$
              Es decir, $u v^m w \in \L(M)$.
    \end{itemize}
\end{proof}

\subsection{Refutando Regularidad}

Una de las aplicaciones del Lema de Pumping es en demostrar que un lenguaje particular no puede ser regular: como todos los lenguajes regulares cumplen este lema, si un lenguaje no lo cumple entonces no es regular.

Tomemos como ejemplo el lenguaje $\L = \{a^n b^n \mid n \in \N\}$.

\begin{theorem*}
    El lenguaje $\L = \{a^n b^n \mid n \in \N\}$ no es regular.
\end{theorem*}
\begin{proof}
    Lo demostraremos por el absurdo: si $\L$ fuera regular, entonces existiría una constante de pumping $p \in \N$. Luego, tomemos la cadena $z = a^p b^p$. Es claro que:
    \begin{itemize}
        \item $z \in \L$
        \item $|z| = 2p \geq p$
    \end{itemize}

    Por ende, esta cadena se puede factorizar como $z = uvw$, con $|uv| = m \leq p$ y $|v| \geq 1$. Como $uv$ es prefijo de $z$ con longitud a lo sumo $p$, debe estar compuesta únicamente por el símbolo $a$ (porque $z$ empieza con $p$ $a$s). Esto a su vez implica que $v = a^k$ para algún $k \geq 1$. Además, sabemos que:
    $$uv^2w = a^{m-k}a^{2k}a^{p - m}b^p = a^{p + k}b^p \in \L$$

    Pero esto es absurdo, porque $p + k \geq p + 1 > p \implies p + k \neq p$.
\end{proof}
