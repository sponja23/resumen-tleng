\section{Propiedades de Clausura de Lenguajes Regulares}

Ahora que sabemos que los lenguajes regulares pueden ser generados por tanto expresiones regulares como autómatas finitos, podemos demostrar que esta clase de lenguajes está \textbf{cerrada} bajo ciertas operaciones, es decir, que aplicar éstas a operandos regulares resultará en otro lenguaje regular.

\subsection{Concatenación}

\begin{theorem*}
    Sean $\L_1, \L_2 \subseteq \Sigma^*$ lenguajes regulares. Su concatenación $\L_1 \cdot \L_2$ es regular.
\end{theorem*}
\begin{proof}
    Es trivial: sean $r_1, r_2$ expresiones regulares para $\L_1$ y $\L_2$. Entonces, la expresión $r_1 r_2$ reconoce al lenguaje $\L_1 \cdot \L_2$ y, por ende, éste es regular.
\end{proof}

\subsection{Unión}

\begin{theorem*}
    Sean $\L_1, \L_2 \subseteq \Sigma^*$ lenguajes regulares. Su unión $\L_1 \cup \L_2$ es regular.
\end{theorem*}
\begin{proof}
    La demostración es análoga al caso anterior: sean $r_1, r_2$ expresiones regulares para $\L_1$ y $\L_2$. Entonces, la expresión $r_1|r_2$ reconoce al lenguaje $\L_1 \cup \L_2$ y, por ende, éste es regular.
\end{proof}

\subsection{Clausura de Kleene/Positiva}

\begin{theorem*}
    Sea $\L \subseteq \Sigma^*$ un lenguaje regular. Tanto $\L^*$ como $\L^+$ son regulares.
\end{theorem*}
\begin{proof}
    Sea $r$ una expresión regular que reconoce $\L$. Entonces, $r^*$ y $r^+$ reconocen $\L^*$ y $\L^+$ respectivamente y, por ende, ambos lenguajes son regulares.
\end{proof}

\subsection{Complemento}

\begin{theorem*}
    Sea $\L \subseteq \Sigma^*$ un lenguaje regular. Su complemento $\L^c$ es regular.
\end{theorem*}
\begin{proof}
    Sea $M = \langle Q, \Sigma, \delta, q_0, F \rangle$ un autómata finito tal que $\L(M) = \L$. Luego, tomemos $M' = \langle Q, \Sigma, \delta, q_0, Q \setminus F \rangle$, es decir, un autómata idéntico con los estados finales invertidos. Entonces, una cadena es aceptada por $M'$ cuando:
    $$
        w \in \L(M') \iff \hat \delta(q_0, w) \in Q \setminus F \iff \delta(q_0, w) \notin F \iff w \notin \L(M)
    $$
    Por ende, el lenguaje aceptado por $M'$ es $\L^c$, y esto implica que ese lenguaje es regular (porque se puede construir un autómata finito que lo acepta).
\end{proof}

\subsection{Intersección}

\begin{theorem*}
    Sean $\L_1, \L_2 \subseteq \Sigma^*$ lenguajes regulares. Su intersección $\L_1 \cap \L_2$ es regular.
\end{theorem*}
\begin{proof}
    Según las Leyes de De Morgan, $\L_1 \cap \L_2 = \left(\L_1^c \cap \L_2^c\right)^c$. Como la clase de lenguajes regulares está cerrada por unión y complemento, este lenguaje también es regular.
\end{proof}

\subsection{Reverso}

\begin{theorem*}
    Sea $\L \subseteq \Sigma^*$ un lenguaje regular. Su reverso $\L^r$ es regular.
\end{theorem*}
\begin{proof}
    Tomemos un autómata finito determinístico $M = \langle Q, \Sigma, \delta, q_0, F \rangle$ tal que $\L(M) = \L$. Luego, construyamos el autómata no determinístico $M' = \langle Q', \Sigma, \delta', q_0', F' \rangle$, donde:
    \begin{itemize}
        \item El conjunto de estados es $Q' := Q \cup \{q_0'\}$, es decir, se agrega un nuevo estado $q_0'$.
        \item La función de transición está dada por:
              $$
                  \delta'(q, a) :=
                  \begin{cases}
                      \{r \mid \delta(r, a) = q\} & \si q \neq q_0'                \\
                      F                           & \si q = q_0' \land a = \lambda
                  \end{cases}
              $$
              Es decir, se \textbf{revierten} todas las transiciones de los estados originales, mientras que el nuevo estado está conectado mediante transiciones lambda con todos los estados finales anteriores.
        \item El estado inicial es el nuevo estado $q_0'$.
        \item El conjunto de estados finales es $F' = \{q_0\}$, es decir, tiene sólo al estado inicial original.
    \end{itemize}

    Luego, veamos que la relación de transición del nuevo autómata cumple $(q, w) \vdash_{M'}^* (r, \lambda) \iff (r, w^r) \vdash_M^* (q, \lambda)$ para los estados originales $q, r \in Q$. La demostración será por inducción en la longitud de $w$:
    \begin{itemize}
        \item \textbf{Caso Base}: Si $|w| = 0$, entonces $w = \lambda$. Como $M'$ no tiene ninguna transición lambda entre los estados originales, la única forma de que se cumpla $(q, \lambda) \vdash_{M'}^* (r, \lambda)$ es que $q = r$ y, por ende, también se cumple $(r, \lambda^r) = (r, \lambda) \vdash_M^* (q, \lambda)$. Lo mismo vale para la implicación inversa: el autómata original no tiene transiciones lambda porque es determinístico.
        \item \textbf{Paso Inductivo}: Si $|w| > 0$, entonces $w = a \cdot w'$. Por hipótesis inductiva, sabemos que $(q, w') \vdash_{M'}^* (r, \lambda) \iff (r, w'^r) \vdash_M^* (q, \lambda)$ para cualquier par de estados originales $q, r \in Q$. Demostremos que $(q, a \cdot w') \vdash_{M'}^* (r, \lambda) \iff (r, w'^r \cdot a) \vdash_M^* (q, \lambda)$.
              \begin{itemize}
                  \item $\implies$) Si $(q, a \cdot w') \vdash_{M'}^* (r, \lambda)$ entonces, por la \hyperref[subsubsec-propiedades-rel-transicion]{propiedad de la relación de transición} de siempre tomar algún estado, sabemos que existe algún estado $s \in Q$ tal que:
                        $$
                            (q, a) \vdash_{M'}^* (s, \lambda) \land (s, w') \vdash_{M'}^* (r, \lambda)
                        $$
                        Como el autómata no tiene transiciones lambda entre los estados originales, $(q, a) \vdash_{M'}^* (s, \lambda)$ sólo puede valer cuando $(q, a) \vdash_{M'} (s, \lambda)$, y esto a su vez implica que $s \in \delta'(q, a)$. Por la definición de $\delta'$, tenemos que $s \in \delta'(q, a) \implies \delta(s, a) = q \implies (s, a) \vdash_M (q, \lambda)$. Por otro lado, la hipótesis inductiva garantiza que $(s, w') \vdash_{M'}^* (r, \lambda)$ implica $(r, w'^r) \vdash_M^* (s, \lambda)$.

                        Por la propiedad de concatenación de la relación de transición, se tiene que:
                        $$
                            (r, w'^r) \vdash_M^* (s, \lambda) \land (s, a) \vdash_M^* (q, \lambda) \implies (r, w'^r \cdot a) \vdash_M^* (q, \lambda)
                        $$
                  \item $\impliedby$) La demostración es análoga: asumiendo $(r, w'^r \cdot a) \vdash_M^* (q, \lambda)$, podemos usar la propiedad de existencia de estados intermedios para obtener un estado $s \in Q$ tal que:
                        $$
                            (r, w'^r) \vdash_M^* (s, \lambda) \land (s, a) \vdash_M^* (q, \lambda)
                        $$
                        Nuevamente, sabemos que $(s, a) \vdash_M^* (q, \lambda) \implies (s, a) \vdash_M (q, \lambda) \implies \delta(s, a) = q$ y, por la definición de $\delta'$, esto implica que $s \in \delta'(q, a) \iff (q, a) \vdash_{M'} (s, \lambda)$. Por otro lado, la hipótesis inductiva implica que $(r, w'^r) \vdash_M^* (s, \lambda) \implies (s, w') \vdash_{M'} (r, \lambda)$.

                        Por la propiedad de concatenación de la relación de transición, se tiene que:
                        $$
                            (q, a) \vdash_{M'}^* (s, \lambda) \land (s, w') \vdash_{M'}^* (r, \lambda) \implies (q, a \cdot w') \vdash_{M'}^* (r, \lambda)
                        $$
              \end{itemize}
    \end{itemize}

    Ahora que sabemos que $(q, w) \vdash_{M'}^* (r, \lambda) \iff (r, w^r) \vdash_M^* (q, \lambda)$ para estados originales $q, r \in Q$, analicemos las cadenas aceptadas por $M'$:
    $$
        w \in \L(M') \iff (q_0', w) \vdash_{M'}^* (q_0, \lambda)
    $$
    El único estado final es $q_0$, así que la ejecución del autómata debe terminar en ese estado. Por otro lado, las únicas transiciones salientes de $q_0'$ son las transiciones lambda hacia los estados finales, así que se tiene:
    $$
        (q_0', w) \vdash_{M'}^* (q_0, \lambda) \iff \exists q_f \in F \mid (q_f, w) \vdash_{M'}^* (q_0, \lambda)
    $$
    Entonces, por la propiedad demostrada anteriormente, se tiene que:
    $$
    \begin{aligned}        
        w \in \L(M') & \iff \exists q_f \in F \mid (q_f, w) \vdash_{M'}^* (q_0, \lambda) \\
        & \iff \exists q_f \in F \mid (q_0, w^r) \vdash_{M}^* (q_f, \lambda) \iff w^r \in \L(M)
    \end{aligned}
    $$
    Es decir, $\L(M') = {w^r \mid w \in \L(M)} = (\L(M))^r = \L^r$, y por ende, $\L^r$ es regular.
\end{proof}
