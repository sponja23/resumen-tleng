\section{Autómatas Finitos Determinísticos}
\label{definicion-afds}

Un \textbf{autómata finito determinístico} (AFD) $M$ es una 5-upla $\langle Q, \Sigma, \delta, q_0, F \rangle$ donde:
\begin{itemize}
    \item $Q$ es el conjunto finito de \textbf{estados del autómata}.
    \item $\Sigma$ es el \textbf{alfabeto de entrada}.
    \item $\delta: Q \times \Sigma \to Q$ es la \textbf{función de transición}: si el autómata se encuentra en el estado $q \in Q$ y lee el símbolo $a \in \Sigma$ de la entrada, pasa al estado $\delta(q, a)$.
    \item $q_0 \in Q$ es el \textbf{estado inicial}.
    \item $F \subseteq Q$ es el conjunto de \textbf{estados finales}/\textbf{estados de aceptación}.
\end{itemize}


Estos autómatas toman una cadena de entrada $w \in \Sigma^*$ e iteran por los símbolos de la misma. Empezando en el estado inicial, cada símbolo indica el siguiente estado a tomar. Si, al terminar de leer la entrada, el autómata se encuentra en un estado final, la cadena es \textbf{aceptada}. De lo contrario, es \textbf{rechazada}.

\subsubsection{Visualización}

Estos autómatas se suelen representar como \textbf{grafos} $G = \langle V, E \rangle$, donde:
\begin{itemize}
    \item Los \textbf{nodos} representan los estados del autómata ($V = Q$).
    \item Las \textbf{aristas} están dadas por la función de transición: $(v, w) \in E \iff \exists a \in \Sigma \mid \delta(v, a) = w$. Además, las aristas se etiquetan con los símbolos por los cuales se puede tomar la transición.
    \item El estado inicial se representa con una arista entrante que no viene de otro estado.
    \item Los estados finales se respresentan con doble borde.
\end{itemize}

\begin{figure}[H]
    \centering
    \begin{tikzpicture}
        \node[state, initial] (0) {$q_0$};
        \node[state, accepting, right of=0] (1) {$q_1$};
        \node[state, below right of=0] (2) {$q_2$};

        \draw   (0) edge[above] node{$a$} (1)
        (0) edge[above] node{$b$} (2)
        (1) edge[below, bend left, right=0.3] node{$a,b$} (2)
        (2) edge[above, bend left, left=0.3] node{$b$} (1)
        (2) edge[loop right] node{$a$} (2);
    \end{tikzpicture}
    \caption*{Visualización del autómata $M = \langle \{q_0, q_1, q_2\}, \{a, b\}, \delta, q_0, \{q_1\} \rangle$, donde la función de transición $\delta$ está dada por las flechas entre los estados.}
\end{figure}

\subsection{Configuraciones Instantáneas}

La ejecución de los AFDs se puede formalizar usando el concepto de las \textbf{configuraciones instantáneas}: $\langle q, s \rangle \in Q \times \Sigma^*$ representa el punto de la ejecución del autómata en el que se encuentra en el estado $q$ y tiene a $s$ como entrada restante.

La función de transición $\delta$ de un autómata $M$ se puede adaptar a una relación entre configuraciones instantáneas, la \textbf{relación de transición} $\vdash_M$:
$$
    (q, a \cdot w) \vdash_M (r, w) \iff \delta(q, a) = r
$$

Luego, el \textbf{lenguaje aceptado} por el autómata $M$, denotado $\L(M)$ se puede definir como:
$$
    \L(M) := \{ w \in \Sigma^* \mid \exists q_f \in F : (q_0, w) \vdash_M^* (q_f, \lambda) \}
$$

\subsubsection{Propiedades}
\label{subsubsec-propiedades-rel-transicion}

La relación de transición $\vdash$ cumple las siguientes propiedades:

\begin{flalign*}
     &  & (q, \alpha) \vdash^* (r, \lambda) \land (q, \alpha) \vdash^* (s, \lambda) & \implies r = s                                                                                         &  & \textbf{(Determinismo)}              \\
     &  & (q, \alpha) \vdash^* (r, \lambda) \land (r, \beta) \vdash^* (p, \lambda)  & \implies (q, \alpha \cdot \beta) \vdash^* (p, \lambda)                                                 &  & \textbf{(Concatenación)}             \\
     &  & (q, \alpha \cdot \beta) \vdash^* (p, \lambda)                             & \implies \exists r \in Q \mid (q, \alpha) \vdash^* (r, \lambda) \land (r, \beta) \vdash^* (p, \lambda) &  & \textbf{(Siempre toma algún estado)} \\
     &  & (q, \alpha) \vdash^n (r, \lambda)                                         & \iff |\alpha| = n                                                                                      &  & \textbf{(Linealidad)}                \\
     &  & (q, \alpha) \vdash^* (q, \lambda)                                         & \implies \forall i \in \N,\ (q, \alpha^i) \vdash^* (q, \lambda)                                          &  & \textbf{(Invarianza)}
\end{flalign*}

\subsection{Función de Transición Extendida}

Una forma alternativa (y análoga) de dar el lenguaje aceptado es por medio de la \textbf{función de transición extendida}. Esta es una función $\hat \delta : Q \times \Sigma^* \to \mathcal P (Q)$ (toma cadenas en vez de símbolos) definida de la siguiente manera:
$$
    \begin{aligned}
        \hat \delta(q, \lambda)   & := q                              \\
        \hat \delta(q, w \cdot a) & := \delta (\hat \delta (q, w), a)
    \end{aligned}
$$
Donde $q \in Q, a \in \Sigma, w \in \Sigma^*$.

Bajo esta definición el lenguaje aceptado por un autómata $M$ es simplemente:
$$
    \L (M) := \{w \in \Sigma^* \mid \hat \delta (q_0, w) \in F \}
$$

\subsubsection{Propiedades}

La función de transición extendida cumple propiedades similares a las de la relación de transición:
\begin{flalign*}
     &  & \hat \delta(q, \alpha \cdot \beta) & = \hat \delta(\hat \delta(q, \alpha), \beta)             &  & \textbf{(Concatenación)} \\
     &  & \hat \delta(q, \alpha) = q         & \implies \forall i \in \N,\ \hat \delta(q, \alpha^i) = q &  & \textbf{(Invarianza)}
\end{flalign*}

\subsection{Autómatas Incompletos}

Según la \hyperref[definicion-afds]{definición anterior}, la función de transición $\delta$ debe definir un estado siguiente para cada combinación de estados y símbolos. Sin embargo, a veces resulta cómodo tener estados en los cuales recibir ciertos símbolos cause un rechazo inmediato de la cadena. Una forma de lograr eso es permitir que $\delta$ sea una \textbf{función parcial}, es decir, que se indefina para ciertas entradas. Un autómata cuya función de transición es parcial se denomina \textbf{incompleto}.

\begin{figure}[H]
    \centering
    \begin{tikzpicture}
        \node[state, initial] (0) {$q_0$};
        \node[state, right of=0] (1) {$q_1$};
        \node[state, accepting, right of=1] (2) {$q_2$};

        \draw   (0) edge[above] node{$a$} (1)
        (1) edge[above] node{$b$} (2);
    \end{tikzpicture}
    \caption*{Autómata incompleto que solamente acepta la cadena $ab$.}
\end{figure}

No obstante, la distinción entre autómatas completos e incompletos no es muy importante gracias al siguiente teorema:

\begin{theorem*}
    Para cualquier autómata incompleto $M$, existe un autómata completo $M'$ equivalente, es decir, $\L(M) = \L(M')$.
\end{theorem*}
\begin{proof}
    Si $M = \langle Q, \Sigma, \delta, q_0, F \rangle$, basta con tomar el autómata $M' = \langle Q \cup \{q_T\}, \Sigma, \delta', q_0, F \rangle$, donde $q_T$ es un nuevo estado, llamado \textbf{estado trampa}. La función de transición se extiende de la siguiente manera:
    $$
        \delta' (q, a) :=
        \begin{cases}
            \delta(q, a) & \si q \in Q \land \delta(q, a) \neq \indef \\
            q_T          & \si q = q_T \lor \delta(q, a) = \indef
        \end{cases}
    $$
    Veamos que $\L(M) = \L(M')$, demostrando $w \in \L(M) \iff w \in \L(M')$.

    $\implies$) Si $w \in \L(M)$, el autómata debe pasar por una serie de \textbf{transiciones definidas}, hasta llegar a un estado final. Luego, por la definición de $\delta'$, las transiciones entre esos estados son las mismas en $M'$, así que se llega al mismo estado, y por ende $w \in \L(M')$. Formalmente,
    $$
        \begin{aligned}
            a_1 \dots a_k \in \L(M) \implies & \exists q_f \in F : (q_0, a_1 \dots a_k) \vdash_M^* (q_f, \lambda)                                                         \\
            \iff                             & \exists q_1, \dots, q_k \in Q : (q_0, a_1 \dots a_k) \vdash_M (q_1, a_2 \dots a_k) \vdash_M \cdots \vdash_M (q_k, \lambda) \\
                                             & \land q_k \in F
        \end{aligned}
    $$
    Tomando la secuencia de estados $q_0, q_1 \dots q_k$, se tiene:
    $$
        \begin{aligned}
             & (q_i, a_{i+1} \dots a_k) \vdash_M (q_{i+1}, a_{i+2} \dots a_k) \iff \delta(q_i, a_{i+1}) = q_{i+1} \implies \delta'(q_i, a_{i+1}) = q_{i+1} \\
             & \implies (q_i, a_{i+1} \dots a_k) \vdash_{M'} (q_{i+1}, a_{i+2} \dots a_k)
        \end{aligned}
    $$
    Esto implica que $(q_0, w) \vdash_{M'}^* (q_k, \lambda)$, y entonces $w \in \L(M')$.

    $\impliedby$) Supongamos que $w \notin \L(M)$. Hay 2 posibilidades:
    \begin{itemize}
        \item Se tiene $(q_0, w) \vdash_M^* (q_k, \lambda)$ con $q_k \notin F$. En ese caso, el autómata nunca se indefine y, por un razonamiento análogo, $(q_0, w) \vdash_{M'}^* (q_k, \lambda)$, así que $w \notin \L(M')$.
        \item Por otro lado, puede suceder que el autómata se indefina durante la ejecución. En tal caso, se tiene una secuencia de estados $q_1, \dots, q_j$:
              $$(q_0, a_1 \dots a_k) \vdash_M (q_1, a_2 \dots a_k) \vdash_M \cdots \vdash_M (q_j, a_{j+1} \dots a_k)$$
              Con $\delta(q_j, a_{j+1}) = \indef$.

              En este caso, $\delta' (q_j, a_{j+1}) = q_T$ y, como $\delta'(q_T, a) = q_T\ \forall a \in \Sigma$, el autómata sigue en ese estado hasta consumir el resto de la cadena. Considerando que la ejecución se da de la misma forma en $M'$ antes de la indefinición, se tiene:
              $$(q_0, a_1 \dots a_k) \vdash_{M'}^+ (q_j, a_{j+1} \dots a_k) \vdash_{M'} (q_T, a_{j+2} \dots a_k) \vdash_{M'}^* (q_T, \lambda)$$
              Como $q_T \notin F$, $w \notin \L(M')$.
    \end{itemize}
\end{proof}

Este resultado implica que los autómatas incompletos tienen el mismo poder expresivo que los completos. Además, la construcción del autómata completo a partir de uno incompleto es computacionalmente barata: sólo requiere agregar un nuevo estado y conectarlo con a lo sumo $|Q|$ estados.

\begin{figure}[H]
    \centering
    \begin{tikzpicture}
        \node[state, initial] (0) {$q_0$};
        \node[state, right of=0] (1) {$q_1$};
        \node[state, accepting, right of=1] (2) {$q_2$};
        \node[state, below of=1] (3) {$q_T$};

        \draw   (0) edge[above] node{$a$} (1)
        (1) edge[above] node{$b$} (2)
        (0) edge[below, bend right] node{$b$} (3)
        (1) edge[right] node{$a$} (3)
        (2) edge[below, bend left] node{$a,b$} (3)
        (3) edge[loop below] node{$a,b$} (3);
    \end{tikzpicture}
    \caption*{Versión completa del autómata anterior, obtenida a partir de agregar un estado trampa.}
\end{figure}
