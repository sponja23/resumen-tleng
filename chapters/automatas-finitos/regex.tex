
\section{Expresiones Regulares}

Una \textbf{expresión regular} (o \textit{regex}) $r$ es una forma compacta de expresar lenguajes sobre un alfabeto $\Sigma$. Se definen de la siguiente manera:
\begin{itemize}
    \item ``$\emptyset$'' denota el lenguaje vacío.
    \item ``$\lambda$'' denota el lenguaje $\{\lambda\}$.
    \item Para cada símbolo $a \in \Sigma$, ``$a$'' denota el lenguaje $\{a\}$.
    \item Si $r$ y $s$ son expresiones regulares:
          \begin{itemize}
              \item $r|s$ denota la \textbf{unión} de sus lenguajes: $\L(r|s) := \L(r) \cup \L(s)$.
              \item $rs$ denota la \textbf{concatenación} de sus lenguajes: $\L(rs) := \L(r) \cdot \L(s)$.
              \item $r^*$ denota la \textbf{clausura de Kleene}: $\L(r^*) := \L(r)^*$.
              \item $r^+$ denota la \textbf{clausura positiva}: $\L(r^+) := \L(r)^+$.
          \end{itemize}
\end{itemize}

El \textbf{orden de precedencia} de estos operadores es $\square^*, \square^+, \cdot, |$. Se pueden asociar de otra forma por medio de paréntesis: por ejemplo, la expresión $ab^*$ denota todas las cadenas que empiezan con $a$ y terminan en 0 o más $b$s, mientras que la expresión $(ab)^*$ denota todas las cadenas formadas por $0$ o más repeticiones de $ab$.

\subsection{Equivalencia con Autómatas Finitos}

Los lenguajes reconocibles por expresiones regulares exactamente los mismos lenguajes reconocibles por autómatas finitos. Para demostrar esto, demostraremos que:
\begin{itemize}
    \item $\L$ es expresable por expresión regular $\implies$ $\L$ es expresable por autómata finito.
    \item $\L$ es expresable por autómata finito $\implies$ $\L$ es expresable por gramática regular.
    \item $\L$ es expresable por gramática regular $\implies$ $\L$ es expresable por expresión regular.
\end{itemize}
Es decir, como efecto colateral, también demostraremos que éstos son los lenguajes expresables por gramáticas regulares (los \textbf{lenguajes regulares}).

\subsubsection{Expresión Regular $\implies$ Autómata Finito}

Primero veamos que, dada una expresión regular, siempre se puede obtener un autómata que reconoce el mismo lenguaje

\begin{theorem*}
    Dada una expresión regular $r$, existe un AFND-$\lambda$ equivalente, es decir, uno para el cual $\L(r) = \L(M)$.
\end{theorem*}
\begin{proof}
    La demostración será por inducción estructural sobre $r$. Además, demostraremos que siempre se puede construir un autómata con un único estado final.

    Los casos base son:
    \begin{itemize}
        \item $r = \emptyset$. Se puede tomar el siguiente $M$:
              \begin{figure}[H]
                  \centering
                  \begin{tikzpicture}
                      \node[state, initial] (0) {$q_0$};
                      \node[state, accepting, right of=0] (1) {$q_1$};
                  \end{tikzpicture}
              \end{figure}

              Como el único estado final es inaccesible, ninguna cadena es aceptada.
        \item $r = \lambda$. Se puede tomar el siguiente $M$:
              \begin{figure}[H]
                  \centering
                  \begin{tikzpicture}
                      \node[state, initial, accepting] (0) {$q_0$};
                  \end{tikzpicture}
              \end{figure}

              En este caso, la cadena vacía es aceptada, pero la lectura de cualquier símbolo hace que el autómata se indefina, por lo cual las cadenas no vacías son rechazadas.
        \item $r = a$, con $a \in \Sigma$. Se puede tomar el siguiente $M$:
              \begin{figure}[H]
                  \centering
                  \begin{tikzpicture}
                      \node[state, initial] (0) {$q_0$};
                      \node[state, accepting, right of=0] (1) {$q_1$};

                      \draw   (0) edge[above] node{$a$} (1);
                  \end{tikzpicture}
              \end{figure}

              La única cadena que acepta es la que tiene solamente al símbolo $a$.
    \end{itemize}

    Por otro lado, Dadas dos expresiones $r$ y $s$ cuyos autómatas correspondientes son:
    \begin{itemize}
        \item $M_r = \langle Q_r, \Sigma, \delta_r, q_{0r}, \{q_{fr}\} \rangle$
        \item $M_s = \langle Q_s, \Sigma, \delta_s, q_{0s}, \{q_{fs}\} \rangle$
    \end{itemize}

    Entonces se pueden obtener autómatas para cada una de las operaciones:
    \begin{itemize}
        \item $p = r|s$. Se toma el autómata $M_p = \langle Q_r \cup Q_s \cup \{q_{0p}, q_{fp}\}, \Sigma, \delta_p, q_{0p}, \{q_{fp}\} \rangle$, con $\delta_p$ dada por:
              $$
                  \delta_p(q, a) :=
                  \begin{cases}
                      \delta_r(q, a)     & \si q \in Q_r \setminus \{q_{fr}\}             \\
                      \delta_s(q, a)     & \si q \in Q_s \setminus \{q_{fs}\}             \\
                      \{q_{0r}, q_{0s}\} & \si q = q_{0p} \land a = \lambda               \\
                      \{q_{fp}\}         & \si q \in \{q_{fr}, q_{fs}\} \land a = \lambda \\
                      \emptyset          & \ecc
                  \end{cases}
              $$

              Esto puede ser visualizado de la siguiente manera:
              \begin{figure}[H]
                  \centering
                  \begin{tikzpicture}
                      \tikzstyle{surround} = [fill=gray!20,dotted,draw=black,rounded corners=6mm, inner sep=10pt]

                      \node[state, initial] (0) {$q_{0p}$};
                      \node[state, above right of=0] (1) {$q_{0r}$};
                      \node[state, below right of=0] (3) {$q_{0r}$};
                      \node[state, right of=1] (2) {$q_{fr}$};
                      \node[state, right of=3] (4) {$q_{fs}$};
                      \node[state, accepting, below right of=2] (5) {$q_{fp}$};

                      \draw   (0) edge[above, bend left] node{$\lambda$} (1)
                      (0) edge[below, bend right] node{$\lambda$} (3)
                      (1) edge[dotted] node{} (2)
                      (3) edge[dotted] node{} (4)
                      (2) edge[above, bend left] node{$\lambda$} (5)
                      (4) edge[below, bend right] node{$\lambda$} (5);

                      \begin{pgfonlayer}{background}
                          \node[surround] (background) [fit=(1)(2), label=$M_r$] {};
                          \node[surround] (background) [fit=(3)(4), label=below:$M_s$] {};
                      \end{pgfonlayer}
                  \end{tikzpicture}
              \end{figure}

              Es claro que este autómata sólo acepta las cadenas que son aceptadas por $M_r$ o $M_s$.

        \item $p = rs$. Se toma el autómata $M_p = \langle Q_r \cup Q_s \cup \{q_{0p}, q_{fp}\}, \Sigma, \delta_p, q_{0p}, \{q_{fp}\} \rangle$, con $\delta_p$ dada por:
              $$
                  \delta_p(q, a) :=
                  \begin{cases}
                      \delta_r(q, a) & \si q \in Q_r \setminus \{q_{fr}\} \\
                      \delta_s(q, a) & \si q \in Q_s \setminus \{q_{fs}\} \\
                      \{q_{0r}\}     & \si q = q_{0p} \land a = \lambda   \\
                      \{q_{0s}\}     & \si q = q_{fr} \land a = \lambda   \\
                      \{q_{fp}\}     & \si q = q_{fs} \land a = \lambda   \\
                      \emptyset      & \ecc
                  \end{cases}
              $$

              Puede ser visualizado de la siguiente manera:
              \begin{figure}[H]
                  \centering
                  \resizebox{0.8\textwidth}{!}{
                      \begin{tikzpicture}
                          \tikzstyle{surround} = [fill=gray!20,dotted,draw=black,rounded corners=6mm, inner sep=10pt]

                          \node[state, initial] (0) {$q_{0p}$};
                          \node[state, right of=0] (1) {$q_{0r}$};
                          \node[state, right of=1] (2) {$q_{fr}$};
                          \node[state, right of=2] (3) {$q_{0s}$};
                          \node[state, right of=3] (4) {$q_{fs}$};
                          \node[state, right of=4, accepting] (5) {$q_{fp}$};

                          \draw   (0) edge[above] node{$\lambda$} (1)
                          (1) edge[dotted] node{} (2)
                          (2) edge[above] node{$\lambda$} (3)
                          (3) edge[dotted] node{} (4)
                          (4) edge[above] node{$\lambda$} (5);

                          \begin{pgfonlayer}{background}
                              \node[surround] (background) [fit=(1)(2), label=$M_r$] {};
                              \node[surround] (background) [fit=(3)(4), label=$M_s$] {};
                          \end{pgfonlayer}
                      \end{tikzpicture}
                  }
              \end{figure}

              En este caso, el autómata sólo acepta si lee una cadena que es aceptada por $M_r$, seguida por una cadena aceptada por $M_s$. Éstas son justamente las cadenas en $\L(M_r) \cdot \L(M_s)$.

        \item $p = r^*$. Se toma el autómata $M_p = \langle Q_r \cup \{q_{0p}\}, \Sigma, \delta_p, q_{0p}, \{q_{0p}\} \rangle$, con $\delta_p$ dado por:
              $$
                  \delta_p(q, a) :=
                  \begin{cases}
                      \delta_r(q, a) & \si q \in Q_r \setminus \{q_{fr}\} \\
                      \{q_{0r}\}     & \si q = q_{0p} \land a = \lambda   \\
                      \{q_{0r}\}     & \si q = q_{fr} \land a = \lambda   \\
                      \emptyset      & \ecc
                  \end{cases}
              $$

              Puede ser visualizado de la siguiente manera:
              \begin{figure}[H]
                  \centering
                  \begin{tikzpicture}
                      \tikzstyle{surround} = [fill=gray!20,dotted,draw=black,rounded corners=6mm, inner sep=10pt]

                      \node[state, initial, accepting] (0) {$q_{0p}$};
                      \node[state, right of=0] (1) {$q_{0r}$};
                      \node[state, right of=1] (2) {$q_{fr}$};

                      \draw   (0) edge[above] node{$\lambda$} (1)
                      (1) edge[dotted] node{} (2)
                      (2) edge[below, bend left] node{$\lambda$} (0);

                      \begin{pgfonlayer}{background}
                          \node[surround] (background) [fit=(1)(2), label=$M_r$] {};
                      \end{pgfonlayer}
                  \end{tikzpicture}
              \end{figure}

              Es fácil ver que el autómata sólo acepta cadenas de la forma $w^i$ con $i \geq 0$ y $w \in \L(M_r)$, que forman el lenguaje $\L(M_r)^*$.

        \item $p = r^+$. En este caso, se puede descomponer la expresión en $p = rr^*$.
    \end{itemize}
\end{proof}

La construcción realizada en la demostración se conoce como el \textbf{algoritmo de construcción de Thompson}. Para cada componente de la expresión, se agrega una cantidad constante de estados al autómata. Por ende, el autómata construido para una expresión $r$ tiene $\BigO{|r|}$ estados (y el algoritmo tiene esa misma complejidad temporal).

\subsubsection{Autómata Finito $\implies$ Gramática Regular}

\textbf{TODO}

\subsubsection{Gramática Regular $\implies$ Expresión Regular}

\textbf{TODO}
