\section{Autómatas de Pila}

Los \textbf{autómatas de pila} son similares a los autómatas finitos en que su ejecución se da mediante transiciones entre estados de un conjunto finito. La diferencia es que estos autómatas tienen acceso a una \textbf{pila}: las transiciones pueden apilar símbolos en ésta, y la transición tomada en cada paso puede depender del símbolo en el tope de la misma.

Formalmente, un autómata de pila $M$ es una tupla de la forma $\langle Q, \Sigma, \Gamma, \delta, q_0, Z_0, F \rangle$, donde:
\begin{itemize}
    \item $Q$ es el conjunto de estados.
    \item $\Sigma$ es el alfabeto de entrada.
    \item $\Gamma$ es el alfabeto de la pila.
    \item $\delta: Q \times (\Sigma \cup \{ \lambda \}) \times \Gamma \to \mathcal P (Q \times \Gamma^*)$ es la función de transición. Ésta recibe un estado, un símbolo de entrada (o $\lambda$) y un símbolo de tope de pila, y devuelve el conjunto de tuplas conformadas por los posibles estados siguientes junto con la cadena de símbolos que se apilan en cada caso.
    \item $q_0 \in Q$ es el estado inicial.
    \item $Z_0 \in \Gamma$ es el símbolo inicial de la pila.
    \item $F \subseteq Q$ es el conjunto de estados finales.
\end{itemize}

Al igual que los autómatas finitos, los autómatas de pila reciben una cadena $w \in \Sigma^*$ e iteran por sus símbolos. Se empieza en el estado inicial $q_0$ con una pila de $Z_0$, y se siguen las transiciones que indica $\delta$, modificando la pila según corresponda, hasta consumir toda la entrada. La aceptación se puede definir de 2 maneras:
\begin{enumerate}
    \item La cadena es aceptada cuando el último estado alcanzado está en $F$.
    \item La cadena es aceptada cuando, al terminar la ejecución, la pila está vacía.
\end{enumerate}

\subsection{Configuraciones Instantáneas}

En este caso, además de tener el estado actual y la cadena de entrada restante, las \textbf{configuraciones instantáneas} de estos autómatas deben contener los símbolos de la pila. Por ende, son tuplas $(q, \alpha, \pi) \in Q \times \Sigma^* \times \Gamma^*$. La \textbf{relación de transición} entre dichas configuraciones está dada por:
\begin{itemize}
    \item Si $(r, \tau) \in \delta(q, a, t)$, entonces $(q, a \alpha, t \pi) \vdash (r, \alpha, \tau \pi)$ (se consume $a$ de la entrada, se consume $t$ de la pila y se apila $\tau$).
    \item Si $(r, \tau) \in \delta(q, \lambda, t)$, entonces $(q, \alpha, t \pi) \vdash (r, \alpha, \tau \pi)$ (no se cambia la entrada, se consume $t$ de la pila y se apila $\tau$). 
\end{itemize}

Luego, si $M$ acepta por \textbf{estados finales}, su lenguaje aceptado es:
$$
    \L(M) := \{w \in \Sigma^* \mid \exists q_f \in F, \gamma \in \Gamma^* :  (q_0, w, Z_0) \vdash^* (q_f, \lambda, \gamma)\}
$$

Por otro lado, si $M$ acepta por \textbf{pila vacía}, su lenguaje aceptado es:
$$
    \L_\lambda(M) := \{w \in \Sigma^* \mid \exists r \in Q :  (q_0, w, Z_0) \vdash^* (r, \lambda, \lambda)\}
$$

\subsection{Equivalencia entre Tipos de Aceptación}

En el caso de los autómatas de pila generales, los 2 tipos de aceptación, por estados finales y por pila vacía, tienen el mismo poder expresivo.

\subsubsection{Estados Finales $\implies$ Pila Vacía}

\begin{theorem*}
    Dado un autómata de pila $M$, existe otro $M'$ tal que $\L(M) = \L_\lambda(M')$.
\end{theorem*}
\begin{proof}
    Sea $M = \langle Q, \Sigma, \Gamma, \delta, q_0, Z_0, F \rangle$. Luego, se puede construir el autómata $M' = \langle Q', \Sigma, \Gamma', \delta', q_0', X_0, \emptyset \rangle$, de la siguiente manera:
    \begin{itemize}
        \item $Q' = Q \cup \{q_0', q_\lambda\}$
        \item $\Gamma' = \Gamma \cup \{X_0\}$
        \item $\delta'$ tiene las todas transiciones de $\delta$, agregando:
        \begin{itemize}
            \item $\delta'(q_0', \lambda, X_0) = \{(q_0, Z_0 X_0)\}$ (se agrega un nuevo símbolo al fondo de la pila).
            \item $(q_\lambda, \lambda) \in \delta'(q_f, \lambda, t) \ \forall q_f \in F, t \in \Gamma$ (los estados finales se conectan a $q_\lambda$).
            \item $\delta'(q_\lambda, \lambda, t) = \{(q_\lambda, \lambda) \mid t \in \Gamma\}$ (el estado $q_\lambda$ vacía la pila).
        \end{itemize}
    \end{itemize}

    Veamos que este autómata acepta por pila vacía las mismas cadenas que el original acepta por estados finales. La idea es que el autómata apila un nuevo símbolo $X_0$ y \textbf{simula} al autómata original. Cuando se llega a un estado final, se pasa al estado $q_\lambda$, que tiene transiciones para vaciar la pila.

    \begin{itemize}
        \item $\L(M) \subseteq \L_\lambda(M')$: Supongamos que $w \in \L(M)$. Esto quiere decir que existe algún estado final $q_f$ y cadena de pila $\gamma$ tal que:
        $$
            (q_0, w, Z_0) \vdash_M^* (q_f, \lambda, \gamma)
        $$
        Además, como $\delta'(q_0', \alpha, X_0) \vdash_{M'} (q_0, \alpha, Z_0 X_0)$ para cualquier cadena $\alpha \in \Sigma^*$, y todas las transiciones de $\delta$ están incluidas en $\delta'$, $M'$ \textbf{simula} a $M$, es decir:
        $$
            (q_0', w, X_0) \vdash_{M'} (q_0, w, Z_0 X_0) \vdash_{M'}^* (q_f, \lambda, \gamma X_0)
        $$

        Una vez que el autómata llega a un estado final $q_f$, puede tomar la transición hacia $q_\lambda$, y luego vaciar el resto de la pila (denotamos $\gamma = t_1 t_2 \cdots t_k$):
        $$
            (q_f, \lambda, t_1 t_2 \cdots t_k X_0) \vdash_{M'} (q_\lambda, \lambda, t_2 \cdots t_k X_0)
        $$

        Luego, las transiciones de $q_\lambda$ permiten consumir todos los símbolos de la pila hasta vaciarla:
        $$
            (q_\lambda, \lambda, t_2 \cdots t_k X_0) \vdash_{M'} (q_\lambda, \lambda, t_3 \cdots t_k X_0) \vdash_{M'} \cdots \vdash_{M'} (q_\lambda, \lambda, X_0) \vdash_{M'} (q_\lambda, \lambda, \lambda)
        $$

        Juntando todas las cadenas de transiciones, se tiene:
        $$
            (q_0', w, X_0) \vdash_{M'}^* (q_\lambda, \lambda, \lambda) \implies w \in \L_\lambda(M')
        $$

        \item $\L_\lambda(M') \subseteq \L(M)$: Si $w \in \L_\lambda(M')$, es porque:
        $$
            (q_0', w, X_0) \vdash_{M'}^* (q_\lambda, \lambda, \lambda)
        $$

        Esto es porque la única transición que puede desapilar el símbolo inicial $X_0$ es la que está en $q_\lambda$. Más aún, estas transiciones deben ser de la forma:
        $$
            (q_0', w, X_0) \vdash_{M'} \underbrace{(q_0, w, Z_0 X_0) \vdash_{M'}^* (q_f, w, \gamma X_0)}_{\text{Simulación de } M} \vdash_{M'}^* (q_\lambda, \lambda, \lambda)
        $$

        Las transiciones que simulan a $M$ también están presentes en $M$, así que se tiene:
        $$
            (q_0, w, Z_0) \vdash_M^* (q_f, w, \gamma) \implies w \in \L(M)
        $$
    \end{itemize}
\end{proof}

\subsubsection{Pila Vacía $\implies$ Estados Finales}

\begin{theorem*}
    Dado un autómata de pila $M$, existe otro $M'$ tal que $\L(M') = L_\lambda(M)$.
\end{theorem*}
\begin{proof}
    Sea $M = \langle Q, \Sigma, \Gamma, \delta, q_0, Z_0, \emptyset \rangle$. Se define $M' = \langle Q', \Sigma, \Gamma', \delta', q_0', X_0, F \rangle$, donde:
    \begin{itemize}
        \item $Q' = Q \cup \{q_0', q_f\}$
        \item $\Gamma' = \Gamma \cup \{X_0\}$
        \item $\delta'$ tiene todas las transiciones de $\delta$, agregando:
        \begin{itemize}
            \item $\delta'(q_0', \lambda, X_0) = \{(q_0, Z_0 X_0)\}$ (se agrega un nuevo símbolo al fondo de la pila).
            \item $(q_f, \lambda) \in \delta'(q, \lambda, X_0)$ (todos los estados pueden pasar al estado final consumiendo el símbolo nuevo).
        \end{itemize}
        \item $F = \{q_f\}$
    \end{itemize}

    En este caso, el autómata nuevo también simula al original. Veamos que aceptan los mismos lenguajes:
    \begin{itemize}
        \item $\L_\lambda(M) \subseteq \L(M')$: Si $w \in \L_\lambda(M)$, entonces se tiene $(q_0, w, Z_0) \vdash_M (q, \lambda, \lambda)$ para algún $q \in Q$. Luego, por la definición de $\delta'$, también vale:
        $$
            (q_0', w, X_0) \vdash_{M'} \underbrace{(q_0, w, Z_0 X_0) \vdash_{M'}^* (q, \lambda, \lambda)}_{\text{Simulación de } M} \vdash_{M'} (q_f, \lambda, \lambda)
        $$

        Es decir,
        $$
            (q_0', w, X_0) \vdash_{M'}^* (q_f, \lambda, \lambda) \implies w \in \L(M')
        $$
        \item $\L(M') \subseteq \L_\lambda(M)$: Para que una cadena $w$ sea aceptada por $M'$, la ejecución debe comenzar por $q_0'$, pasar por transiciones de $M$ hasta consumir toda la cadena de entrada y llegar al símbolo $X_0$, para después pasar al nuevo estado final $q_f$. Es decir, la cadena de transiciones debe ser de la forma:
        $$
            (q_0', w, X_0) \vdash_{M'} \underbrace{(q_0, w, Z_0 X_0) \vdash_{M'}^* (q, \lambda, \lambda)}_{\text{Simulación de } M} \vdash_{M'} (q_f, \lambda, \lambda)
        $$

        Entonces, como las transiciones simuladas también están presentes en el autómata original, se tiene:
        $$
            (q_0, w, Z_0 X_0) \vdash_M^* (q, \lambda, \lambda) \implies w \in \L_\lambda(M)
        $$
    \end{itemize}
\end{proof}
