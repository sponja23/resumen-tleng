\section{Autómatas de Pila}

Los \textbf{autómatas de pila} son similares a los autómatas finitos en que su ejecución se da mediante transiciones entre estados de un conjunto finito. La diferencia es que estos autómatas tienen acceso a una \textbf{pila}: las transiciones pueden apilar símbolos en ésta, y la transición tomada en cada paso puede depender del símbolo en el tope de la misma.

Formalmente, un autómata de pila $M$ es una tupla de la forma $\langle Q, \Sigma, \Gamma, \delta, q_0, Z_0, F \rangle$, donde:
\begin{itemize}
    \item $Q$ es el conjunto de estados.
    \item $\Sigma$ es el alfabeto de entrada.
    \item $\Gamma$ es el alfabeto de la pila.
    \item $\delta: Q \times (\Sigma \cup \{ \lambda \}) \times \Gamma \to \mathcal P (Q \times \Gamma^*)$ es la función de transición. Ésta recibe un estado, un símbolo de entrada (o $\lambda$) y un símbolo de tope de pila, y devuelve el conjunto de tuplas conformadas por los posibles estados siguientes junto con la cadena de símbolos que se apilan en cada caso.
    \item $q_0 \in Q$ es el estado inicial.
    \item $Z_0 \in \Gamma$ es el símbolo inicial de la pila.
    \item $F \subseteq Q$ es el conjunto de estados finales.
\end{itemize}

Al igual que los autómatas finitos, los autómatas de pila reciben una cadena $w \in \Sigma^*$ e iteran por sus símbolos. Se empieza en el estado inicial $q_0$ con una pila de $Z_0$, y se siguen las transiciones que indica $\delta$, modificando la pila según corresponda, hasta consumir toda la entrada. La aceptación se puede definir de 2 maneras:
\begin{enumerate}
    \item La cadena es aceptada cuando el último estado alcanzado está en $F$.
    \item La cadena es aceptada cuando, al terminar la ejecución, la pila está vacía.
\end{enumerate}

\subsection{Configuraciones Instantáneas}

En este caso, además de tener el estado actual y la cadena de entrada restante, las \textbf{configuraciones instantáneas} de estos autómatas deben contener los símbolos de la pila. Por ende, son tuplas $(q, \alpha, \pi) \in Q \times \Sigma^* \times \Gamma^*$. La \textbf{relación de transición} entre dichas configuraciones está dada por:
\begin{itemize}
    \item Si $(r, \tau) \in \delta(q, a, t)$, entonces $(q, a \alpha, t \pi) \vdash (r, \alpha, \tau \pi)$ (se consume $a$ de la entrada, se consume $t$ de la pila y se apila $\tau$).
    \item Si $(r, \tau) \in \delta(q, \lambda, t)$, entonces $(q, \alpha, t \pi) \vdash (r, \alpha, \tau \pi)$ (no se cambia la entrada, se consume $t$ de la pila y se apila $\tau$). 
\end{itemize}

Luego, si $M$ acepta por \textbf{estados finales}, su lenguaje aceptado es:
$$
    \L(M) := \{w \in \Sigma^* \mid \exists q_f \in F, \gamma \in \Gamma^* :  (q_0, w, Z_0) \vdash^* (q_f, \lambda, \gamma)\}
$$

Por otro lado, si $M$ acepta por \textbf{pila vacía}, su lenguaje aceptado es:
$$
    \L_\lambda(M) := \{w \in \Sigma^* \mid \exists r \in Q :  (q_0, w, Z_0) \vdash^* (r, \lambda, \lambda)\}
$$

\subsection{Equivalencia entre Tipos de Aceptación}

En el caso de los autómatas de pila generales, los 2 tipos de aceptación, por estados finales y por pila vacía, tienen el mismo poder expresivo.

\subsubsection{Estados Finales $\implies$ Pila Vacía}

\begin{theorem*}
    Dado un autómata de pila $M$, existe otro $M'$ tal que $\L(M) = \L_\lambda(M')$.
\end{theorem*}
\begin{proof}
    Sea $M = \langle Q, \Sigma, \Gamma, \delta, q_0, Z_0, F \rangle$. Luego, se puede construir el autómata $M' = \langle Q', \Sigma, \Gamma', \delta', q_0', X_0 \rangle$, de la siguiente manera:
    \begin{itemize}
        \item $Q' = Q \cup \{q_0', q_\lambda\}$
        \item $\Gamma' = \Gamma \cup \{X_0\}$
        \item $\delta'$ tiene las todas transiciones de $\delta$, agregando:
        \begin{itemize}
            \item $\delta'(q_0', \lambda, X_0) = \{(q_0, Z_0 X_0)\}$ (se agrega un nuevo símbolo al fondo de la pila).
            \item $\delta'(q_f, \lambda, t) = \{(q_\lambda, \lambda)\}$ (los estados finales se conectan a $q_\lambda$).
            \item $\delta'(q_\lambda, \lambda, t) = \{(q_\lambda, \lambda)\}$ (el estado $q_\lambda$ vacía la pila).
        \end{itemize}
    \end{itemize}

    \textbf{TODO}
\end{proof}

\subsubsection{Pila Vacía $\implies$ Estados Finales}

\textbf{TODO}
