\section{Reescritura de Gramáticas Libres de Contexto}

En esta sección, analizaremos propiedades de las gramáticas libres de contexto y métodos para modificarlas.

\subsection{Símbolos Inútiles}

Sea $G = \langle V_N, V_T, P, S \rangle$. Un símbolo inútil $A \in V_N^*$ es uno para el cual no hay una derivación:
$$
    S \derivs w A y \derivs w x y
$$

Es decir, que no es alcanzable por derivaciones desde $S$ (es \textit{inaccesible}), o desde el cual no se puede llegar a una cadena de terminales. Para determinar la segunda propiedad, se puede un algoritmo similar al de \hyperref[glc-vacia2]{algoritmo de vacuidad} para obtener el conjunto de terminales que llegan a una cadena, mientras que la primera se puede verificar haciendo DFS sobre las producciones de la gramática, empezando por $S$.

Esto quiere decir que se puede obtener una gramática sin símbolos inútiles de la siguiente manera:
\begin{itemize}
    \item Computar el conjunto de símbolos inútiles $I$.
    \item Armar la gramática $G' = \langle V_N \setminus I, V_T, P', S \rangle$, donde $P' \subseteq P$ son las producciones en las que no participa ningún símbolo de $I$.
\end{itemize}

Claramente, $\L(G) = \L(G')$, porque ninguna derivación de $G$ incluía símbolos/producciones que no estén en $G'$.

\subsection{Gramáticas Propias}

Una producción-$\lambda$ es una de la forma $A \to \lambda$, y una GLC $G = \langle V_N, V_T, P, S \rangle$ es propia cuando no tiene ninguna producción-$\lambda$ excepto por $S \to \lambda$.

Por otro lado, un símbolo $A \in V_N$ es \textbf{anulable} cuando $A \derivs \lambda$. El conjunto de símbolos anulables $V_\lambda$ se puede computar mediante el siguiente algoritmo:

\begin{algorithm}[H]
    \caption{Cómputo del conjunto de símbolos anulables.}
    \label{simbolos-anulables}
    \begin{algorithmic}[1]
        \Procedure{SímbolosAnulables}{$G = \langle V_T, V_N, P, S \rangle$}
        \State $V_0 \gets \{A \mid A \to \lambda \in P\}$
        \State $i \gets 1$
        \Repeat
        \State $V_i \gets V_{i - 1} \cup \{A \mid \exists A \to X_1 \cdots X_k \in P : \forall 1 \leq i \leq k, \ X_i \in V_{i - 1} \}$
        \State $i \gets i + 1$
        \Until{$V_i = V_{i - 1}$}
        \State \Return $V_i$
        \EndProcedure
    \end{algorithmic}
\end{algorithm}

Luego, se puede obtener una gramática propia mediante el siguiente algoritmo:
\begin{algorithm}[H]
    \caption{Transformación a gramática propia.}
    \label{gramatica-propia}
    \begin{algorithmic}[1]
        \Procedure{GramáticaPropia}{$G = \langle V_T, V_N, P, S \rangle$}
        \State $V_\lambda \gets \textsc{SímbolosAnulables}(G)$
        \State $P' \gets \emptyset$
        \For{$A \to X_1 \cdots X_k \in P \mid X_1 \cdots X_k \neq \lambda$}
            \State $X_{i_1} \cdots X_{i_n} \gets X_1 \cdots X_k \cap V_\lambda$  \Comment{Símbolos anulables del cuerpo}
            \For{$X_{j_1} \cdots X_{j_m}$ subcadena de $X_{i_1} \cdots X_{i_n}$}
            \State $P' \gets P' \cup \{A \to (X_1 \cdots X_k \setminus \{X_{j_1}, \dots, X_{j_m}\})\}$
            \EndFor
        \EndFor
        \If{$\lambda \in \L(G)$}
            \State $P' \gets P' \cup \{S \to \lambda\}$
        \EndIf
        \State \Return $\langle V_N, V_T, P', S \rangle$
        \EndProcedure
    \end{algorithmic}
\end{algorithm}

El algoritmo itera por todas las producciones de $P$, descartando las que tienen cuerpo $\lambda$, y para las otras agrega todas las posibles cadenas que se forman al eliminar combinaciones de sus símbolos anulables. De esta forma, el lenguaje sigue siendo el mismo, pero la gramática es propia.

\subsection{Producciones Unitarias}

Una \textbf{producción unitaria} es una de la forma $A \to B$, con $A, B \in V_N$. Lo único que indica esta producción es que el símbolo $A$ se puede reemplazar por $B$ en cualquier lugar que se encuentre.

Para eliminar estas producciones, es conveniente considerar la relación de alcanzabilidad por producciones unitarias, dada por $A \prec^U B \iff A \derivs B$\footnote{La notación $\mathop{\prec^U}$ no es estándar.}. Esta relación se puede computar fácilmente por medio de DFS o BFS\footnote{Este algoritmo es sólo el cómputo de una clausura transitiva.}:
\begin{algorithm}[H]
    \caption{Cómputo de alcanzabilidad por producciones unitarias.}
    \label{alcanzabilidad-unitarias}
    \begin{algorithmic}[1]
        \Procedure{AlcanzabilidadProduccionesUnitarias}{$G = \langle V_T, V_N, P, S \rangle$}
        \State $\mathop{\prec^U_0} \gets \{(A, A) \mid A \in V_N\}$
        \State $i \gets 1$
        \Repeat
        \State $\mathop{\prec^U_i} \gets \mathop{\prec^U_0} \cup \{(X, Z) \in V_N \times V_N \mid \exists Y \in V_N : X \to Y \in P \land Y \prec^U_{i - 1} Z\}$
        \State $i \gets i + 1$
        \Until{$\mathop{\prec^U_i} = \mathop{\prec^U_{i - 1}}$}
        \State \Return $V_i$
        \EndProcedure
    \end{algorithmic}
\end{algorithm}

Luego, para obtener una gramática sin producciones unitarias, se puede usar el siguiente algoritmo:

\begin{algorithm}[H]
    \caption{Eliminación de producciones unitarias.}
    \label{producciones-unitarias}
    \begin{algorithmic}[1]
        \Procedure{SinProduccionesUnitarias}{$G = \langle V_T, V_N, P, S \rangle$}
        \State $\prec^U \gets \textsc{AlcanzabilidadProduccionesUnitarias}$
        \State $P' \gets \emptyset$
        \For{$A, B \in V_N \mid A \prec^U B$}
        \State $P' \gets P' \cup \{A \to \alpha \mid B \to \alpha \in P\} \setminus \{A \to B\}$
        \EndFor
        \State \Return $\langle V_N, V_T, P', S \rangle$
        \EndProcedure
    \end{algorithmic}
\end{algorithm}

Notemos que, como $X \prec^U X \forall X \in V_N$, $P'$ tiene todas las producciones de $P$ (excepto por las unitarias), y se agregan nuevas producciones $A \to \alpha$ para todos los símbolos que alcanzan a otros por producciones unitarias ($A \prec^U B$).

\subsection{Gramática sin Ciclos}

Un \textbf{ciclo} en una gramática $G$ se da cuando un símbolo se puede derivar a sí mismo: $A \derivp A$. Por ende, una \textbf{gramática sin ciclos} es una en la que no hay ninguna de esas derivaciones.

En las gramáticas libres de contexto, la existencia de un ciclo requiere producciones-$\lambda$ o producciones unitarias (es fácil de demostrar por inducción en la cantidad de derivaciones de un ciclo). Por ende, si se aplican los métodos para eliminar producciones-$\lambda$ y producciones unitarias, la gramática resultantes es sin ciclos.

\subsection{Gramática Recursiva a Izquierda/Derecha}

Como se \hyperref[recursiva-a-izquierda-cota]{mencionó antes}, Una gramática $G = \langle V_N, V_T, P, S \rangle$ libre de contexto es \textbf{recursiva a izquierda} cuando tiene algún $A \in V_N$ que cumple:
$$
    A \derivp A \alpha
$$

Además, una producción de la forma $A \to A \alpha$ se denomina \textbf{recursión inmediata} (por izquierda).

Análogamente, la recursión por derecha se da cuando $A \derivp \alpha A$.

Para cualquier gramática $G$, se puede obtener una gramática equivalente sin recursión a izquierda.

\subsubsection{Eliminación de Recursión Inmediata}

Sea $G = \langle V_N, V_T, P, S \rangle$ una gramática con producciones de la forma:
$$
    A \to A \alpha_1 \mid \cdots \mid A \alpha_k \mid \beta_1 \mid \cdots \mid \beta_n
$$

Donde las cadenas $\beta_i$ no empiezan con $A$.

Luego, se puede intercambiar la recursión por izquierda inmediata de esta gramática por recursión por derecha. Para lograrlo, se agrega un nuevo símbolo $A'$, y se reemplazan las producciones anteriores por\footnote{Para que esto funcione, no debe haber ninguna producción de la forma $A \to A$ (igualmente, esas se pueden eliminar sin cambiar la gramática).}:
$$
\begin{aligned}
    A & \to \beta_1 \mid \cdots \mid \beta_n \mid \beta_1 A' \mid \cdots \mid \beta_n A' \\
    A' & \to \alpha_1 \mid \cdots \mid \alpha_k \mid \alpha_1 A' \mid \cdots \mid \alpha_k A'
\end{aligned}
$$

Luego, para cualquier derivación de la gramática anterior con la forma:
$$
    A \to A \alpha_{i_1} \to A \alpha_{i_2} \alpha_{i_1} \to \cdots \to \beta_j \alpha_{i_m} \cdots \alpha_{i_1}
$$

La gramática nueva $G = \langle V_N \cup A', V_T, P', S \rangle$ puede llegar a la misma cadena por medio de:
$$
    A \to \beta_j A' \to \beta_j \alpha_{i_m} A' \to \cdots \to \beta_j \alpha_{i_m} \cdots \alpha_{i_1}
$$

Notemos que, si $P$ tenía $k$ producciones con cabeza $A$, se agregan $k$ producciones a $P'$.

\subsubsection{Eliminación de Recursión No-Inmediata}

Para ver cómo eliminar la recursión no-inmediata, se puede tomar como ejemplo el siguiente conjunto de producciones:
$$
\begin{aligned}
    A & \to BC \mid a \\
    B & \to CA \mid Ab
\end{aligned}
$$

La gramática es recursiva a izquierda ya que, por ejemplo:
$$
    A \deriv BC \deriv AbC
$$

Para arreglarla, se pueden reemplazar las ocurrencias de $A$ al principio del cuerpo de las producciones por los símbolos en los que se deriva:
$$
\begin{aligned}
    A & \to BC \mid a \\
    B & \to CA \mid BCb \mid ab
\end{aligned}
$$

Ahora la gramática solamente tiene recursión por izquierda inmediata, que se puede eliminar con el método anterior.

Generalizando esto, la idea es ordenar los símbolos de alguna forma (arbitraria) $A_1, \dots, A_n$ e ir modificando iterativamente la gramática: en cada paso, las producciones con cabeza $A_i$ se transforman para que, si empiezan con un símbolo no terminal, sea uno $A_j$, con $j > i$, y se elimina cualquier recursión inmediata que surja. Al terminar el algoritmo, no puede haber ninguna instancia de recursión no-inmediata, porque para ello se requiere que:
$$
    A_i \deriv A_{j_1} \alpha_1 \deriv A_{j_2} \alpha_2 \deriv \cdots \deriv A_{j_{k - 1}} \alpha_{j_{k - 1}}\deriv A_i \alpha_{j_k}
$$

Pero, como $A_i \derivp A_j \alpha$ sólo cuando $i < j$, se tendría que $i < i_1$ e $i_1 < i$ (absurdo).

El algoritmo es el siguiente\footnote{La gramática de entrada no debería tener ciclos (aunque el algoritmo puede ser modificado para detectarlos y eliminarlos), producciones unitarias ni producciones-$\lambda$.}:

\begin{algorithm}[H]
    \caption{Eliminación de recursión por izquierda.}
    \label{eliminacion-recursion-izquierda}
    \begin{algorithmic}[1]
        \Procedure{SinRecursiónPorIzquierda}{$G = \langle V_T, V_N, P, S \rangle$}
        \State $A_1, \dots, A_n \gets V_N$
        \For{$i \in \{1, \dots, n\}$}
            \For{$j \in \{1, \dots, i - 1\}$}
                \State Sean $A_j \to \delta_1 \mid \delta_2 \mid \cdots \mid \delta_m$ las producciones de $A_j$.
                \For{$A_i \to A_j \gamma_k \in P$}
                    \State Reemplazar $A_i \to A_j \gamma_k$ por $A_i \to \delta_1 \gamma_k \mid \delta_2 \gamma_k \mid \cdots \mid \delta_m \gamma_k$
                \EndFor
            \EndFor
        \State Eliminar recursión inmediata para $A_i$
        \EndFor
        \EndProcedure
    \end{algorithmic}
\end{algorithm}

El ciclo más interno se puede eliminar por medio de \textbf{factorización por izquierda}: cuando se tienen varias producciones de la forma:
$$
    X \to Y \alpha_1 \mid \cdots \mid Y \alpha_n \mid \beta_1 \mid \cdots \mid \beta_m
$$

Se pueden reemplazar por:
$$
\begin{aligned}
    X & \to Y Y' \mid \beta_1 \mid \cdots \mid \beta_m \\
    Y' & \to \alpha_1 \mid \cdots \mid \alpha_n
\end{aligned}
$$

De esta forma, $X$ tiene a lo sumo 1 producción que empieza con $Y$ para cada par de símbolos no terminales $X, Y \in V_N$.

\subsubsection{Complejidad del Algoritmo}

Analicemos la complejidad de peor caso del algoritmo de eliminación de recursión a izquierda. Si $c$ es la máxima cantidad de producciones con una misma cabeza, el peor caso se da cuando todas las producciones son de la forma $X \to A_1 A_2 \cdots A_n$:
\begin{itemize}
    \item Todas las producciones con cabeza $A_1$ tienen recursión inmediata. Esto implica que quedan $2c$ producciones con cabeza $A_1$.
    \item Todas las producciones con cabeza $A_2$ tienen un cuerpo que empieza con $A_1$. Para cada una de ellas, hay que agregar las $2c$ producciones que quedaron para de $A_1$, y luego eliminar la recursión inmediata (porque éstas empiezan con $A_2$). Se tienen $2(2c) * c = (2c)^2$ producciones con cabeza $A_2$.
    \item En el último paso, todas las producciones con cabeza $A_n$ empiezan con $A_1$, y se tiene una cantidad de producciones dada por:
    $$
        2c \prod_{i = 0}^{n - 1} (2c)^{2^i} = 2c (2c)^{\sum_{i = 0}^{n - 1} 2^i} = 2c (2c)^{2^n - 1} = (2c)^{2^n}
    $$
\end{itemize}

En el peor caso, se agregan menos de $2(2c)^{2n}$. Sin embargo, el tiempo de ejecución depende fuertemente del orden en el que se recorren los símbolos.

\subsection{Forma Normal de Chomsky}

Una gramática libre de contexto $G = \langle V_N, V_T, P, S \rangle$ está en \textbf{forma normal de Chomsky} cuando todas sus producciones son de la forma:
$$
    A \to BC \\
    A \to a
$$

Con $A, B, C \in V_N$ y $a \in V_T$. Además, se puede agregar $S \to \lambda$.

Todo lenguaje libre de contexto tiene una gramática en FNC. Para pasar una gramática cualquiera $G = \langle V_N, V_T, P, S \rangle$ a una $G' = \langle V_N', V_T, P', S' \rangle$ en FNC, primero se deben eliminar producciones-$\lambda$ y unitarias. Luego, el conjunto de producciones $P'$ se debe construir de la siguiente manera:
\begin{itemize}
    \item Agregar $S' \to S$ y, si $\lambda \in \L(G)$, agregar $S' \to \lambda$.
    \item Agregar todas las producciones que ya están en FNC (todas las $A \to a$ y $A \to BC$ con $a \in V_T,$ y $A,B,C \in V_N$).
    \item Para cada producción $A \to X_1 X_2$ donde alguno/s de $X_1$ y $X_2$ es/son terminal/es, agregar:
    $$
        A \to X_1' X_2'
    $$
    \item Para cada producción $A \to X_1 \cdots X_k$ con $k > 2$, poner las producciones:
    $$
    \begin{aligned}
        A & \to X_1' \langle X_2 \cdots X_k \rangle \\
        \langle X_2 \cdots X_k \rangle & \to X_2' \langle X_3 \cdots X_k \rangle \\
        & \vdots \\
        \langle X_{k - 1} X_k \rangle & \to X_{k - 1}' X_k
    \end{aligned}
    $$
    \item Para cada símbolo $X_i'$ de los casos anteriores,
    \begin{itemize}
        \item Si $X_i \in V_T$, $X_i'$ es un símbolo de $V_N$, y se debe agregar $X_i' \to X_i$.
        \item Si $X_i \in V_N$, $X_i' = X_i$.
    \end{itemize}
    \item Cada $\langle X_j \cdots X_k \rangle$ de los casos anteriores es un símbolo nuevo de $V_N$.
\end{itemize}

Este método resulta en una gramática con $|V_N'| \leq |V_N| + |V_T| + \ell |P|$ ($\ell$ es el largo máximo de los cuerpos de producciones en $P$), y corre en un tiempo lineal en $|P|$.

\subsection{Forma Normal de Greibach}

Una gramática libre de contexto $G = \langle V_N, V_T, P, S \rangle$ está en \textbf{forma normal de Greibach} cuando todas sus producciones son de la forma:
$$
    A \to a \alpha
$$

Con $a \in V_T$ y $\alpha \in V_N^*$.

Todo lenguaje libre de contexto tiene una gramática en FNG. Para pasar una gramática cualquiera $G = \langle V_N, V_T, P, S \rangle$ a una $G' = \langle V_N', V_T, P', S' \rangle$ en FNG, primero se deben eliminar producciones-$\lambda$ y la recursión a izquierda. Luego, se puede definir un orden total sobre los símbolos $A_1 < A_2 < \cdots < A_n$ de forma tal que en cada producción $A \to \alpha$, el cuerpo $\alpha$ empieza con un símbolo terminal o un no-terminal $B$ que cumple $A < B$\footnote{Esto sucede por la falta de recursión a izquierda: la relación $A \prec B \iff (\exists \alpha \in (V_N \cup V_T)^* \mid A \derivp B \alpha)$ debe ser de orden, porque claramente es transitiva y, si no fuera antisimétrica, se tendría que $A \derivp A \alpha$ para algún $\alpha$.}. El algoritmo es el siguiente:

\begin{algorithm}[H]
    \caption{Forma Normal de Greibach.}
    \label{forma-normal-de-greibach}
    \begin{algorithmic}[1]
        \Procedure{FNG}{$G = \langle V_T, V_N, P, S \rangle$}
        \State Eliminar producciones-$\lambda$ y recursión por izquierda de $G$
        \State Obtener el orden $A_1 < A_2 < \cdots < A_n$ con $A_i \not\to A_j \alpha\ \forall j < i$
        \For{$i = n - 1, \dots, 1$}
            \For{$A_i \to A_j \alpha \mid j > i$}
                \State Sean $A_j \to \beta_1 \mid \cdots \mid \beta_m$ las producciones con cabeza $A_j$
                \State Reemplazar $A_i \to A_j \alpha$ por $A_i \to \beta_1 \alpha \mid \cdots \mid \beta_m \alpha$
            \EndFor
            \For{$A \to a X_1 \cdots X_k$} \Comment{Todas las producciones de $A$ deben tener esta estructura.}
                \For{$X_i \in \{X_1, \dots, X_k\}$}
                    \State Si $X_i \in V_N$, $X_i' = X_i$.
                    \State Si $X_i \in V_T$, agregar un nuevo símbolo $X_i'$ junto con la producción $X_i' \to X_i$.
                \EndFor
                \State Reemplazar $A \to a X_1 \cdots X_k$ por $A \to a X_1' \cdots X_k'$.
            \EndFor
        \EndFor
        \EndProcedure
    \end{algorithmic}
\end{algorithm}

En cada paso del ciclo externo del algoritmo, se transforman todas las producciones con símbolos $A_i$ para que queden en FNG (empiezan con terminal y siguen con no-terminales).
