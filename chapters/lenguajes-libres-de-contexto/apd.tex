\section{Autómatas de Pila Determinísticos}

Un autómata de pila $M = \langle Q, \Sigma, \Gamma, \delta, q_0, Z_0, F \rangle$ se considera \textbf{determinístico} cuando, para cada $a \in \Sigma, q \in Q, Z \in \Gamma$, se cumple alguna de:
\begin{itemize}
    \item $|\delta(q, a, Z)| \leq 1$, y $|\delta(q, \lambda, Z)| = 0$.
    \item $|\delta(q, a, Z)| = 0$ para todo $a$, y $|\delta(q, \lambda, Z)| \leq 1$.
\end{itemize}

Es decir, cada par (estado, símbolo de pila) puede o bien tener transiciones en algunos símbolos de entrada, o no tener ninguna de esas transiciones, teniendo en cambio alguna transición-$\lambda$.

\hyperref[equivalencia-afd-afnd]{A diferencia de los autómatas finitos}, los APDs no tienen el mismo poder expresivo que los autómatas de pila generales. Por ejemplo, el lenguaje $\mathcal L = \{ww^R \mid w \in \Sigma^*\}$ es libre de contexto\footnote{Se puede construir un autómata de pila no-determinístico que lo reconoce: este autómata tendría un alfabeto de pila $\Gamma$ que representa símbolos de $\Sigma$. Luego, debería apilar los símbolos correspondientes a los de $w$ a medida que lee la cadena, y transicionar no-determinísticamente a desapilarlos, validando que es equivalente al que se está leyendo de la entrada.}, pero no puede ser aceptado por un autómata de pila determinístico. Esto se puede ver considerando la cadena $0^n 11 0^n$: para que sea aceptada por un APD, éste debe usar la pila para validar que hay tantos $0$s antes del $11$ como después. Sin embargo, si luego recibiera la entrada $0^m 11 0^m$, ya no podría verificar si $n = m$, porque la pila se vació en el paso anterior.

\subsection{Clausura por Complemento}

A diferencia de los lenguajes libres de contexto generales, los LLC determinísticos están clausurados por complemento:

\begin{theorem*}
    Sea $\L$ un lenguaje libre de contexto determinístico. Luego, $\L^c$ también es libre de contexto determinístico.
\end{theorem*}

Sea $M = \langle Q, \Sigma, \Gamma, \delta, q_0, Z_0, F\rangle$ un APD tal que $\L(M) = \L$. En este caso, no basta con complementar el conjunto de estados finales del autómata, ya que podría pasar que:
\begin{enumerate}
    \item $M$ sea incompleto, y por ende no acepta una cadena por toma alguna transición vacía. En ese caso, el autómata con estados finales complementados también rechazaría esa cadena.
    \item $M$ entre en un ciclo infinito de transiciones-$\lambda$ para alguna cadena. En este caso, la cadena tampoco se acepta en el autómata con estados finales complementados, porque nunca se llega a consumir por completo.
    \item $M$ consume toda la cadena de entrada, pero llega a un punto en el que se pueden alcanzar estados finales y no-finales por medio de transiciones-$\lambda$. En este caso, el autómata con estados finales complementados también aceptaría la cadena.
\end{enumerate}

Debemos mitigar todos estos problemas para encontrar un autómata que acepte exactamente el lenguaje complementado $\L^c$. 

\subsubsection{Configuraciones Ciclantes}

Una configuración $(q, w, \alpha)$ \textbf{cicla} cuando existes secuencias infinitas de estados $p_1, p_2, \dots$ y cadenas $\beta_1, \beta_2, ...$ con $|\beta_i| \geq \alpha$\footnote{Si se tiene algún $\beta_j$ con longitud menor, se podría tomar como el nuevo principio de la cadena de configuraciones ciclantes. Como la longitud está acotada por debajo (por $0$), hay alguna $\beta_m$ en la cadena con longitud mínima.} tales que:
$$
    (q, w, \alpha) \vdash (p_1, w, \beta_1) \vdash (p_2, w, \beta_2) \vdash \cdots \vdash (p_i, w, \beta_i) \vdash \cdots
$$

Es decir, es una secuencia de configuraciones en la cual nunca se leen símbolos de entrada, y la pila mantiene su longitud inicial.

Se puede ver que existe una cota en la cantidad máxima de transiciones que puede hacer $M$ sin leer de la entrada y sin repetir el estado ni el tope de la pila. Para encontrarla, denotemos como $\ell$ al número máximo de símbolos que se escriben en la pila para alguna transición:
$$
    \ell := \max \{|\tau| \mid \exists q, p \in Q, a \in \Sigma, t \in \Gamma : (p, \tau) \in \delta(q, a, t)\}
$$

Luego, el autómata puede escribir hasta $|Q| |\Gamma| \ell$ símbolos en la pila antes de repetir el estado o el tope de la pila. Por ende, la cantidad máxima de configuraciones distintas por las que se pueden pasar sin leer ningún símbolo de la cadena es:
$$
\begin{aligned}
    K & := |Q| \times |\{\tau \in \Gamma^* \mid |\tau| \leq |Q| |\Gamma| \ell\}| \\
    & = |Q| \sum_{i=0}^{|Q| |\Gamma| \ell} |\Gamma|^i \\
    & = |Q| \frac{|\Gamma|^{|Q| |\Gamma| \ell - 1} - 1}{|\Gamma| - 1}
\end{aligned}
$$

Esto implica que es posible detectar cuándo un autómata entra en una configuración ciclante: luego de ejecutarlo por $K + 1$ pasos, o bien repetió alguna configuración (y en tal caso entró en un ciclo infinito), o debe consumir algún símbolo de la entrada.

\subsubsection{APD Continuo}

Un autómata de pila determinístico $M = \langle Q, \Sigma, \Gamma, \delta, q_0, Z_0, F \rangle$ es \textbf{continuo} cuando ``siempre lee toda la cadena de entrada'', es decir, para toda cadena $w \in \Sigma^*$ existen $p \in Q$ y $\alpha \in \Gamma^*$ tales que $(q_0, w, Z_0) \vdash^* (p, \lambda, \alpha)$.

\begin{lemma*}
    Para todo APD $M = \langle Q, \Sigma, \Gamma, \delta, q_0, Z_0, F \rangle$ existe otro APD equivalente y continuo.
\end{lemma*}
\begin{proof}
    Primero, notemos que se puede asumir que $M$ es completo: si tuviera estados que se indefinen para algún símbolo de entrada/pila, se podría agregar un estado trampa al que transicionan.
    
    Se puede tomar el autómata de pila $M' = \langle Q \cup \{f, t\}, \Sigma, \Gamma, \delta', q_0, Z_0, F \cup \{f\} \rangle$, donde la función de transición $\delta'$ ``arregla'' los casos de transiciones ciclantes o incompletas. Para definirla, consideremos los conjuntos:
    $$
    \begin{aligned}
        C_1 & := \{(q, Z) \in Q \times \Gamma \mid (q, \lambda, Z) \text{ es ciclante } \land \nexists g \in F : (\exists \alpha \in \Gamma^* : (q, \lambda, Z) \vdash^* (g, \lambda, \alpha)) \} \\
        C_2 & := \{(q, Z) \in Q \times \Gamma \mid (q, \lambda, Z) \text{ es ciclante } \land \exists g \in F, \alpha \in \Gamma^* : (q, \lambda, Z) \vdash^* (g, \lambda, \alpha)) \}
    \end{aligned}
    $$

    Es decir, $C_1$ es el conjunto de los estados que pertenecen a una secuencia ciclante sin ningún estado final, y $C_2$ es el conjunto de los estados que pertenecen a una secuencia ciclante en la que sí hay algún estado final.

    Luego, podemos definir $\delta'$ de la siguiente manera:
    \begin{itemize}
        \item $\forall (q, Z) \notin (C_1 \cup C_2), \delta'(q, a, Z) = \delta(q, a, Z)\ \forall a \in \Sigma \cup \{\lambda\}$
        \item $\forall (q, Z) \in C_1, \delta'(q, \lambda, Z) = (t, Z)$
        \item $\forall (q, Z) \in C_2, \delta'(q, \lambda, Z) = (f, Z)$
        \item $\forall a \in \Sigma, Z \in \Gamma, \delta'(f, a, Z) = (t, Z) \land \delta'(t, a, Z) = (t, Z)$
    \end{itemize}
\end{proof}

\subsubsection{Demostración de Clausura}

Para demostrar que cualquier LLCD $\L$ está clausurado por complemento, podemos construir un autómata que acepta $\L^c$:

\begin{proof}
    Sea $M = \langle Q, \Sigma, \Gamma, \delta, q_0, Z_0, F \rangle$ un APD continuo tal que $\L(M) = \L$. Luego, podemos tomar el autómata $M' = \langle Q', \Sigma, \Gamma, \delta', q_0', Z_0, F' \rangle$. El conjunto de estados $Q'$ está definido como:
    $$
        Q' = \{[q, k] \mid q \in Q \land k \in \{0, 1, 2\}\}
    $$

    El propósito del número $k$ es detectar si el autómata pasó o no por un estado final desde la última vez que consumió un símbolo. Específicamente:
    \begin{itemize}
        \item Los estados $[q, 0]$ indican que $M$ no pasó por ningún estado final desde la última vez que consumió un símbolo.
        \item Los estados $[q, 1]$ indican que $M$ pasó por algún estado final desde la última vez que consumió un símbolo.
        \item Los estados $[q, 2]$ son los finales, y el aútomata se mueve a éstos luego de pasar por una serie de transiciones-$\lambda$ sin estados finales (es decir, por estados de la forma $[p, 0]$).
    \end{itemize}

    Cuando $M$ está en un $[q, 0]$ y realiza una transición que lee de la entrada, pasa a un $[q, 2]$. Si la ejecución termina ahí, la cadena es aceptada, porque el autómata acaba de pasar por una serie de transiciones-$\lambda$ sin ningún estado final. Por otro lado, si se lee algún símbolo de la entrada, se pasa al estado $[p, 0]$ o $[p, 1]$ según la pertenencia de $p$ a $F$, y se realizan las transiciones-$\lambda$ correspondientes: los estados $[q, 1]$ siempre pasan a otros $[p, 1]$, pero los $[q, 0]$ pasan a $[p, 0]$ cuando $p \notin F$ y $[p, 1] \in F$.

    Formalmente, la función de transición está definida por:
    \begin{itemize}
        \item Si $\delta(q, a, Z) = \{(p, \gamma)\}$, entonces:
        $$
            \delta'([q, 1], a, Z) = \delta'([q, 2], a, Z) =
            \begin{cases}
                \{([p, 0], \gamma)\} & \si p \notin F \\
                \{([p, 1], \gamma)\} & \si p \in F
            \end{cases}
        $$

        Por otro lado, como $\delta(q, a, Z) = \{(p, \gamma)\}$ y $M$ es determinístico, sabemos que $\delta(q, \lambda, Z) = \emptyset$, así que se puede definir la siguiente transición sin violar el determinismo:
        $$
            \delta'([q, 0], \lambda, Z) = \{([p, 2], Z)\}
        $$

        \item Si $\delta(q, \lambda, Z) = \{(p, \gamma)\}$, entonces:
        $$
        \begin{aligned}
            \delta'([q, 2], \lambda, Z) & = \emptyset \\
            \delta'([q, 1], \lambda, Z) & = \{([p, 1], \gamma)\} \\
            \delta'([q, 0], \lambda, Z) & =
            \begin{cases}
                \{([p, 0], \gamma)\} & \si p \notin F \\
                \{([p, 1], \gamma)\} & \si p \in F
            \end{cases}
        \end{aligned}
        $$

        Además, los estados finales y el inicial se definen como:
        $$
        \begin{aligned}
            F & = \{[q, 2] \mid q \notin F\} \\
            q_0' & =
            \begin{cases}
                [q_0, 0] & \si q_0 \notin F \\
                [q_0, 1] & \si q_0 \in F
            \end{cases}
        \end{aligned}
        $$
    \end{itemize}
\end{proof}

\subsection{Otras Propiedades de Clausura}

Los lenguajes libres de contexto determinístico están clausurados por:

\begin{itemize}
    \item \textbf{Complemento}.
    \item $Pre(\L)$ (las cadenas que tienen un prefijo en $\L$).
    \item $Min(\L)$ (las cadenas de $\L$ que \textbf{no tienen} prefijo propio en $\L$).
    \item $Max(\L)$ (las cadenas de $\L$ que no son prefijos de otra cadena en $\L$).
    \item Intersección con lenguaje regular (misma demostración que para LLC general).
    \item Concatenación por derecha con lenguaje regular (se construye el autómata).
\end{itemize}

\subsection{No-Clausura}

Los lenguajes libres de contexto determinístico \textbf{no están} clausurados por:

\begin{itemize}
    \item Intersección (misma demostración que para LLC general).
    \item Unión (si estuvieran, como están clausurados por complemento, $\left(\L_1^c \cup L_2^c\right)^c = \L_1 \cap \L_2$ sería LLCD).
    \item Reverso.
    \item Concatenación.
    \item Clausura de Kleene.
\end{itemize}

\subsection{Algoritmos de Decisión}

Existen algoritmos para:

\begin{itemize}
    \item $\L \overset{?}{=} \emptyset$ (probar las cadenas de longitud menor a la constante de pumping, o computar el conjunto de símbolos de la gramática que derivan en cadenas)
    \item $\L$ es finito? (probar las cadenas con longitud entre $p$ y $2p$)
    \item $\L$ es infinito? (inverso al anterior)
    \item $\L$ es co-finito? (es lo mismo que ``$\L^c$ es finito?'')
    \item $\L = \Sigma^*$ (es lo mismo que $\L^c \overset{?}{=} \emptyset$)
    \item $\L_1 \overset{?}{=} \L_2$
    \item $w \in \L$ (en tiempo lineal)
    \item $\L \overset{?}{=} R$
    \item $\L \overset{?}{\subseteq} R$
\end{itemize}

No hay algoritmos para:
\begin{itemize}
    \item $\L_1 \overset{?}{\subseteq} \L_2$
    \item $\L_1 \cap \L_2 \overset{?}{=} \emptyset$
\end{itemize}
