\section{Propiedades de Clausura de Lenguajes Libres de Contexto}

\subsection{Unión}

\begin{theorem*}
    Sean $\L_1, \L_2 \subseteq V_T^*$ lenguajes libres de contexto. Su unión $\L_1 \cup \L_2$ es libre de contexto.
\end{theorem*}
\begin{proof}
    Sean $G_1 = \langle V_T, V_{N_1}, P_1, S_1 \rangle$ y $G_2 = \langle V_T, V_{N_2}, P_2, S_2 \rangle$ gramáticas libres de contexto para los lenguajes $\L_1$ y $\L_2$. Luego, se puede tomar la gramática $G = \langle V_T, \{S\} \cup V_{N_1} \cup V_{N_2}, P_1 \cup P_2 \cup \{S \to S_1, S \to S_2\}, S \rangle$. Veamos que $w \in \L(G) \iff w \in \L_1 \lor w \in \L_2$:
    \begin{itemize}
        \item $\implies$) Si $w \in \L(G)$, entonces debe existir alguna derivación $S \underset{G}{\derivp} w$. Como las únicas 2 derivaciones con cabeza $S$ son $S \to S_1$ y $S \to S_2$, se debe dar $S_1 \underset{G}{\derivp} w \lor S_2 \underset{G}{\derivp} w$. Además, como $S_1$ solo participa en derivaciones de $P_1$, y lo mismo sucede para $S_2$ esto quiere decir que $S_1 \underset{G_1}{\derivp} w$ o $S_2 \underset{G_2}{\derivp} w$. Por ende, $w \in \L_1 \lor w \in \L_2$.
        \item $\impliedby$) Si $w \in \L_1 \lor w \in L_2$, se tiene que $S_1 \underset{G_1}{\derivp} w$ o $S_2 \underset{G_2}{\derivp} w$. Como las producciones de ambas gramáticas están contenidas en las producciones de $G$, también se tiene que $S_1 \underset{G}{\derivp} w \lor S_2 \underset{G}{\derivp} w$. Gracias a las producciones $S \to S_1$ y $S \mid S_2$, en ambos casos se tiene $S \underset{G}{\derivp} w$.
    \end{itemize}
\end{proof}

\subsection{Concatenación}

\begin{theorem*}
    Sean $\L_1, \L_2 \subseteq V_T^*$ lenguajes libres de contexto. Su concatenación $\L_1 \cdot \L_2$ es libre de contexto.
\end{theorem*}
\begin{proof}
    La demostración es análoga a la de la unión: tomamos las gramáticas $G_1 = \langle V_T, V_{N_1}, P_1, S_1 \rangle$ y $G_2 = \langle V_T, V_{N_2}, P_2, S_2 \rangle$ para los lenguajes $\L_1$ y $\L_2$, y construimos una nueva gramática $G = \langle V_T, \{S\} \cup V_{N_1} \cup V_{N_2}, P_1 \cup P_2 \cup \{S \to S_1 S_2\}, S \rangle$, que es libre de contexto y cumple que $w \in \L(G) \iff w \in \L_1 \cdot \L_2$.
\end{proof}

\subsection{Clausura de Kleene}

\begin{theorem*}
    Sea $\L \subseteq V_T^*$ un lenguaje libre de contexto. Su clausura de Kleene $\L^*$ es libre de contexto.
\end{theorem*}
\begin{proof}
    Al igual que en los teoremas anteriores, se demuestra tomando una gramática libre de contexto $G = \langle V_T, V_N, P, S \rangle$ para $\L$ y construyendo una de la forma $G' = \langle V_T, \{S'\} \cup V_N, P \cup \{S' \to S S', S' \to \lambda\}, S' \rangle$. Esta nueva gramática cumple que $w \in \L(G') \iff w \in \L^*$.
\end{proof}

\subsection{Reverso}

\begin{theorem*}
    Sea $\L \subseteq V_T^*$ un lenguaje libre de contexto. Su reverso $\L^*$ es libre de contexto.
\end{theorem*}
\begin{proof}
    Dada una gramática libre de contexto $G = \langle V_T, V_N, P, S \rangle$ para $\L$, se puede tomar la gramática $G' = \langle V_T, V_N, P^R, S \rangle$, donde $P^R = \{A \to \beta^R \mid A \to \beta \in P\}$.
\end{proof}

\subsection{Intersección con Lenguaje Regular}

\begin{theorem*}
    Sea $\L \subseteq \Sigma^*$ libre de contexto y $R \subseteq \Sigma^*$ regular. Luego, la intersección $\L \cap R$ es libre de contexto
\end{theorem*}
\begin{proof}
    Sea $M_L = \langle Q_L, \Sigma, \Gamma, \delta_L, q_{L_0}, Z_0, F_L \rangle$ un autómata de pila para $\L$, y sea $M_R = \langle Q_R, \Sigma, \delta_R, q_{R_0}, F_R \rangle$ un AFND para $R$. Luego, se puede tomar el autómata de pila definido por $M' = \langle Q', \Sigma, \Gamma, \delta', q_0', Z_0, F' \rangle$, donde:
    \begin{itemize}
        \item $Q' = Q_L \times Q_R$
        \item $\delta'((q_L, q_R), a, t) = \{((p_L, p_R), s) \mid (p_L, s) \in \delta_L(q_L, a, t) \land p_R \in \delta_R(q_R, a)\}$
        \item $F' = \{(q_L, q_R) \mid q_L \in F_L \land q_R \in F_R\}$
    \end{itemize}

    Luego, $w \in \L(M') \iff w \in \L \cap R$, así que $\L \cap R$ es libre de contexto.
\end{proof}

\subsection{No-clausura}

Se puede verificar que los lenguajes libres de contexto \textbf{no} están clausurados por ciertas operaciones.

\subsubsection{Intersección}

Para refutar la clausura por intersección, se pueden considerar los lenguajes $\L_1 = \{a^i b^j c^j \mid a,b,c \in \Sigma^*, i,j \geq 0\}$ y $\L_2 = \{a^i b^i c^j \mid a,b,c \in \Sigma^*, i,j \geq 0\}$. Ambos son libres de contexto: son concatenaciones entre lenguajes de la forma $\{x^i y^i\}$ (que son libres de contexto) y lenguajes $z^*$ (que son regulares, así que también son libres de contexto).

Por otro lado, su intersección $\L_1 \cap \L_2 = \{a^i b^i c^i \mid a,b,c \in \Sigma^*, i \geq 0\}$ no puede ser libre de contexto, debido al lema de pumping: tomando la cadena $w = a^p b^p c^p$, cualquier descomposición $w = rxyzs$ tendrá a $xyz$ como una cadena que contiene sólo $a$s y $b$s o sólo $b$s y $c$s (porque tiene longitud máxima $p$). Luego, al bombearla, se agregarían sólo 2 de los 3 tipos de símbolos, así que la nueva cadena no estaría en $\L_1 \cap \L_2$ (así que el lenguaje no es libre de contexto).

\subsubsection{Complemento}

Es fácil ver que los lenguajes libres de contexto no están clausurados por complemento: si lo estuvieran, entonces para dos lenguajes LCs $\L_1, \L_2$ se tendría que $\left(\L_1^c \cup \L_2^c\right)^c = \L_1 \cap \L_2$ es libre de contexto (recién vimos que esto es falso).

\subsubsection{Diferencia}

De nuevo, se puede refutar en base a propiedades anteriores: si los LLCs estuvieran cerrados por diferencia, para cualquier LLC $\L$, el lenguaje $\Sigma^* \setminus \L = \L^c$ también sería LC (absurdo, por la propiedad anterior).
