\section{Gramáticas Libres de Contexto}

Como se mencionó anteriormente, una \textbf{gramática libre de contexto} $G = \langle V_N, V_T, P, S \rangle$ en la cual todas las producciones son de la forma:
$$
    A \to \alpha, \text{ con } A \in V_N, \alpha \in (V_N \cup V_T)^*
$$

La relación de derivación $\deriv: (V_N \cup V_T)^* \times (V_N \cup V_T)^*$ de una gramática $G$ está definida en base a las transiciones: si $\alpha, \beta, \gamma_1, \gamma_2 \in (V_N \cup V_T)^*$ con $\alpha \to \beta \in P$, se tiene
$$
    \gamma_1 \alpha \gamma_2 \deriv \gamma_1 \beta \gamma_2
$$

Una \textbf{forma sentencial} de $G$ es una cadena $\alpha \in (V_N \cup V_T)^*$ tal que $S \derivs \alpha$. Luego, el \textbf{lenguaje aceptado} por $G$ se define como:
$$
    \L(G) := \{\alpha \in V_T^* \mid S \derivs \alpha\}
$$

\subsection{Equivalencia con Autómatas de Pila}

Veamos que las gramáticas libres de contexto tienen el mismo poder expresivo que los autómatas de pila.

\subsubsection{GLC $\implies$ Autómata de Pila}

\begin{theorem*}
    Para cada gramática libre de contexto $G = \langle V_N, V_T, P, S \rangle$, existe un autómata de pila $M$ tal que $\L(G) = \L_\lambda(M)$.
\end{theorem*}
\begin{proof}
    Tomemos el autómata $M = \langle Q, \Sigma, \Gamma, \delta, q_0, Z_0 \rangle$, donde:
    \begin{itemize}
        \item $Q = \{q\}$
        \item $\Sigma = V_T$
        \item $\Gamma = (V_T \cup V_N)$
        \item $\delta$ está dada por las siguientes reglas:
        \begin{itemize}
            \item $\forall (A \to \alpha) \in P, \ (q, \alpha) \in \delta(q, \lambda, A)$
            \item $\forall a \in V_T,\ \delta(q, a, a) = \{(q, \lambda)\}$
        \end{itemize}
        \item $q_0 = q$
        \item $Z_0 = S$
    \end{itemize}

    Como resultado intermedio, veamos que $A \derivs w \iff (q, w, A) \overset{*}{\vdash} (q, \lambda, \lambda)$, por inducción completa en la cantidad de derivaciones $k$:
    \begin{itemize}
        \item \textbf{Caso Base}: en $k = 1$, denotando w = $a_1 \cdots a_k$, $A \deriv w$ si y sólo si $A \to w \in P$, y esto a su vez se da si y sólo si $(q, a_1 \cdots a_k, A) \vdash (q, a_1 \cdots a_k, a_1 \cdots a_k) \overset{k}{\vdash} (q, \lambda, \lambda)$.
        \item \textbf{Paso Inductivo}: en $k > 1$, nuestra hipótesis inductiva será que la propiedad vale para todo $j < k$.
        
            Luego, $A \overset{k}{\deriv} w \iff A \to X_1 \cdots X_n \in P$, con $X_i \overset{k_i}{\deriv} x_i$ tal que $w = x_1 \cdots x_n$ y $\sum_{i = 1}^n k_i = k$. Además, alguno de los $k_i$ es al menos $1$ (porque $k > 1$), así que $k_i < k\ \forall i$.

            Por la definición de $M$, $A \to X_1 \cdots X_n \in P \iff (q, w, A) \vdash (q, w, X_1 \cdots X_n)$. Consideremos cada símbolo $X_i$:
            \begin{itemize}
                \item Si $X_i \in V_N$, entonces la hipótesis inductiva indica que $X_i \to x_i \in P \iff (q, x_i, X_i) \overset{*}{\vdash} (q, \lambda, \lambda)$.
                \item Si $X_i = x_i \in V_T^*$, entonces $(q, x_i, X_i) = (q, x_i, x_i) \overset{*}{\vdash} (q, \lambda, \lambda)$.
            \end{itemize}

            Por ende, se tiene que:
            $$
            \begin{aligned}
                A \to X_1 \cdots X_n \in P & \iff (q, w, A) \vdash (q, x_1 \cdots x_n, X_1 \cdots X_n) \\
                & \overset{*}{\vdash} (q, x_2 \cdots x_n, X_2 \cdots X_n) \\
                & \qquad \vdots \\
                & \overset{*}{\vdash} (q, \lambda, \lambda)
            \end{aligned}
            $$
    \end{itemize}

    En particular, se tiene que $w \in \L(G) \iff S \derivp w \iff (q, w, S) \overset{*}{\vdash} (q, \lambda, \lambda) \iff w \in \L(M)$.
\end{proof}

\subsubsection{Autómata de Pila $\implies$ GLC}

\begin{theorem*}
    Para todo autómata de pila $M = \langle Q, \Sigma, \Gamma, \delta, q_0, Z_0 \rangle$, el lenguaje $\L_\lambda(M)$ es libre de contexto.
\end{theorem*}
\begin{proof}
    Se toma la gramática $G = \langle V_N, V_T, P, S \rangle$, donde:
    \begin{itemize}
        \item $S$ es un símbolo nuevo.
        \item $V_N := \{[q, A, p] \mid q, p \in Q, A \in \Gamma\} \cup \{S\}$.
        \item $V_T := \Sigma$.
        \item $P$ está definido por las siguientes reglas:
        \begin{itemize}
            \item $\forall q \in Q$:
            $$
                S \to [q_0, Z_0, q] \in P
            $$
            \item $\forall q, p \in Q, a \in \Sigma \cup \{\lambda\}, A \in \Gamma$:
            $$
                [q, A, p] \to a \in P \iff (p, \lambda) \in \delta(q, a, A)
            $$
            \item $\forall q, q_1, \dots, q_{m + 1} \in Q, a \in \Sigma \cup \{\lambda\}, A \in \Gamma$:
            $$
                [q, A, q_{m+1}] \to a[q_1, B_1, q_2] \cdots [q_m, B_m, q_{m + 1}] \in P \iff (q_1, B_1 \cdots B_m) \in \delta(q, a, A)
            $$
        \end{itemize}
    \end{itemize}

    La idea es que cada derivación más a la izquierda de $G$ es una \textbf{simulación} de la ejecución de $M$. Veamos que, para todo $q, p \in Q, A \in \Gamma, x \in \Sigma^*$, se tiene:
    $$
        (q, x, A) \vdash^* (p, \lambda, \lambda) \iff [q, A, p] \derivs x
    $$

    \begin{itemize}
        \item $\implies$) La demostración será por inducción completa en la cantidad de transiciones de $M$:
        \begin{itemize}
            \item \textbf{Caso Base}: Si $(q, x, A) \vdash (p, \lambda, \lambda)$, es porque $|x| = 1$ o $|x| = 0$:
            \begin{itemize}
                \item $|x| = 0$: en tal caso, $x = \lambda$, y por ende $(q, \lambda, A) \vdash (p, \lambda, \lambda) \iff (p, \lambda) \in \delta(q, \lambda, A) \iff [q, A, p] \to \lambda \in P$. Luego, se tiene $[q, A, p] \deriv \lambda$.
                \item $|x| = 1$: ídem anterior, $x = a$ implica que $(q, a, A) \vdash (p, \lambda, \lambda) \iff (p, \lambda) \in \delta(q, a, A) \iff [q, A, p] \to a \in P$, y entonces $[q, A, p] \deriv a$.
            \end{itemize}
            \item \textbf{Paso Inductivo}: asumamos que la implicación vale para todo $j < i$. Luego, se tiene $x = a \cdot y$, donde $a \in \Sigma \cup \{\lambda\}$ e $y \in \Sigma^*$, y se cumple $(q, x, A) \vdash^i (p, \lambda, \lambda)$. La ejecución se da de la siguiente manera:
            $$
                (q, ay, A) \vdash (q_1, y, B_1 \cdots B_m) \vdash^{i - 1} (p, \lambda, \lambda)
            $$

            Por otro lado, la cadena $y$ se puede descomponer en subcadenas $y = y_1 \cdots y_k$ donde para cada $j$, la cadena $y_1 \cdots y_j$ deja al símbolo $B_j$ en el tope de la pila. Estas cadenas van pasando por los estados intermedios $q_2, \dots, q_{m + 1}$ de forma tal que:
            $$
                (q_j, y_j, B_j) \vdash^{i_j} (q_{j + 1}, \lambda, \lambda)
            $$

            Como $\sum_{j = 1}^m i_j = i - 1$, todos los $i_j$ son menores que $i$, y entonces por hipótesis inductiva se tiene que:
            $$
                (q_j, y_j, B_j) \vdash^{i_j} (q_{j + 1}, \lambda, \lambda) \implies [q_j, B_j, q_{j + 1}] \derivs y_j
            $$

            Tomando la producción $[q, A, q_{m + 1}] \to a [q_1, B_1, q_2] \cdots [q_m, B_m, B_{m + 1}]$, se puede ver que:
            $$
            \begin{aligned}\relax
                [q, A, q_{m + 1}] & \deriv a [q_1, B_1, q_2] \cdots [q_m, B_m, q_{m + 1}] \\
                & \derivs a y_1 [q_2, B_2, q_3] \cdots [q_m, B_m, q_{m + 1}] \\
                & \qquad \qquad \vdots \\
                & \derivs a y_1 \cdots y_m = a y = x
            \end{aligned}
            $$
        \end{itemize}
        \item $\impliedby$) La demostración será por inducción completa en la cantidad de derivaciones:
        \begin{itemize}
            \item \textbf{Caso Base}: Si $[q, A, p] \deriv x$, entonces $x = a$ para algún $a \in \Sigma \cup \{\lambda\}$, y se tiene $[q, A, p] \to a \in P$ que, por la definición de $G$, implica que $(p, \lambda) \in \delta(q, a, A)$. Esto quiere decir que $(q, a, A) \vdash (p, \lambda, \lambda)$.
            \item \textbf{Paso Inductivo}: Supongamos que la implicación vale para cualquier $j < i$, y que $[q, A, p] \overset{i}{\deriv} x$. Luego, como es más de una derivación, la primera derivación utilizada debe ser de la tercera forma, es decir,
            $$
                [q, A, p] \deriv a [q_1, B_1, q_2] \cdots [q_m, B_m, p] \overset{i - 1}{\deriv} x
            $$
            La cadena $x$ se puede descomponer en las subcadenas $x = a x_1 \cdots x_m$ de forma tal que para cada $j$, se tiene $[q_j, B_j, q_{j + 1}] \overset{i_j}{\deriv} x_j$ con $i_j < i$ (porque $\sum_{j = 1}^m i_j = i$). Por hipótesis inductiva, esto implica que $(q_j, x_j, B_j) \vdash^* (q_{j + 1}, \lambda, \lambda)$. Además, como siempre podemos agregar símbolos al fondo de la pila y agregar símbolos al final de la entrada sin cambiar el funcionamiento, se tiene que:
            $$
                (q_j, x_j \cdots x_m, B_j \cdots B_m) \vdash^* (q_j, x_{j + 1} \cdots x_m, B_{j + 1} \cdots B_m)
            $$

            Luego, sabiendo que $[q, A, p] \deriv a [q_1, B_1, q_2] \cdots [q_m, B_m, p] \in P$, se debe tener una transición $(q_1, B_1 \cdots B_m) \in \delta(q, a, A)$ (por definición de $G$), y por ende:
            $$
                \begin{aligned}
                    (q, x, A) & \vdash (q_1, x_1 \cdots x_m, B_1 \cdots B_m) \\
                    & \vdash^* (q_2, x_2 \cdots x_m, B_2 \cdots B_m) \\
                    & \qquad \qquad \vdots \\
                    & \vdash^* (p, \lambda, \lambda)
                \end{aligned}
            $$
        \end{itemize}
    \end{itemize}

    Sabiendo ese lema, se puede ver que, en particular:
    $$
        x \in \L_\lambda(M) \iff (q_0, x, Z_0) \vdash^* (p, \lambda, \lambda) \iff [q_0, Z_0, p] \derivs x
    $$

    Como además $S \to [q_0, Z_0, p]$, se tiene:
    $$
        x \in \L_\lambda(M) \iff S \derivs x \iff x \in \L(G)
    $$
\end{proof}
