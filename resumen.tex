\documentclass[a4paper]{report}
\usepackage{amsmath, amscd, amssymb, amsthm, latexsym}
\usepackage[spanish]{babel}
\usepackage{enumerate}
\usepackage{float}
\usepackage[table]{xcolor}
\usepackage{geometry}
\usepackage{url}
\usepackage{booktabs}
\usepackage{tabularx}
\usepackage{graphicx}
\usepackage{caption}
\usepackage{hyperref}
\usepackage{graphics}
\usepackage{tikz}
\usepackage{multirow}
\usepackage{environ}
\usepackage{algpseudocode}
\usepackage{algorithm}
\usepackage[skip=10pt,indent=10pt]{parskip}

% Setup de tikz
\usetikzlibrary{automata, positioning, arrows, backgrounds, patterns, fit}

\tikzset{
    ->, % makes the edges directed
    >=stealth, % makes the arrow heads bold
    node distance=3cm, % specifies the minimum distance between two nodes. Change if necessary.
    every state/.style={thick, fill=gray!10}, % sets the properties for each ’state’ node
    initial text=$ $, % sets the text that appears on the start arrow
}

% Corregir título de algoritmos
\floatname{algorithm}{Algoritmo}

% Imágenes
\graphicspath{{img/}}

% Captions
\captionsetup[figure]{%
    font=small,%
    justification=centering,%
    width=0.6\textwidth
}

% Links
\hypersetup{
    colorlinks=true,%
    linkcolor=blue!80!red,%
    urlcolor=green!70!black
}

% Márgenes
\geometry{
    a4paper,
    left=30mm,
    right=30mm,
    top=30mm,
    bottom=30mm,
}


% \captionsetup[figure]{%
%     font=small,%
%     justification=centering,%
%     width=0.6\textwidth
% }

% Comandos caseros
\newcommand{\BigO}[1]{\ensuremath{\mathcal{O}(#1)}}
\newcommand{\BigOmega}[1]{\ensuremath{\Omega(#1)}}
\newcommand{\BigTheta}[1]{\ensuremath{\Theta(#1)}}

\newcommand{\code}{\texttt}

\newcommand{\si}{\text{si }}
\newcommand{\ecc}{\text{en caso contrario}}

\newcommand{\N}{\mathbb{N}}
\newcommand{\Z}{\mathbb{Z}}
\newcommand{\R}{\mathbb{R}}

% Comandos tleng
\renewcommand{\L}{\mathcal{L}}
\newcommand{\deriv}{\Rightarrow}
\newcommand{\derivs}{\overset{*}{\Rightarrow}}
\newcommand{\derivp}{\overset{+}{\Rightarrow}}
\newcommand{\indef}{\mathrm \perp}

% Teoremas
\newtheorem*{theorem*}{Teorema}
\newtheorem*{lemma*}{Lema}


\title{%
    Resumen para Teoría de Lenguajes
}
\author{Tomás Spognardi}
\date{\today}

\begin{document}

\maketitle

\tableofcontents

\newpage

\chapter{Introducción a los Lenguajes}

\section{Alfabetos y Cadenas}

Primero, algunas definiciones básicas:

\begin{itemize}
    \item Un \textbf{alfabeto} $\Sigma$ es un conjunto finito, no vacío, de \textbf{símbolos}.
    \item Una \textbf{cadena} $\alpha$ es una secuencia finita de símbolos de algún alfabeto: $\alpha = a_1 a_2 ... a_k$.
    \begin{itemize}
        \item La cadena vacía, denotada por $\lambda$, no tiene ningún símbolo.
    \end{itemize}
\end{itemize}

\subsection{Operaciones de Cadenas}

\subsubsection{Concatenación}

Una cadena $\alpha = a_1 a_2 ... a_k$ se puede \textbf{concatenar} con un símbolo $b$, formando una nueva cadena $b \cdot \alpha = b a_1 a_2 ... a_k$. A veces se omite el símbolo $\cdot$, dejando sólo la expresión $b \alpha$.

Se puede extender esta definición para permitir concatenar cadenas con \textbf{otras cadenas}:
$$
\begin{aligned}
    \alpha \cdot \lambda & := \alpha \\
    \alpha \cdot (b \cdot \beta) & := (\alpha \cdot b) \cdot \beta
\end{aligned}
$$

Esto último nos permite definir la \textbf{repetición}/\textbf{potencia} de cadenas:
$$
\begin{aligned}
    \alpha^0 & := \lambda \\
    \alpha^n & := \alpha \cdot \alpha^{n-1}
\end{aligned}
$$

\subsubsection{Longitud}

La \textbf{longitud} de una cadena $\alpha = a_1 a_2 ... a_k$ es la cantidad de símbolos que tiene: $|\alpha| = k$. Se puede definir de forma inductiva:
$$
\begin{aligned}
    |\lambda| & := 0 \\
    |a \cdot \alpha| & := 1 + |\alpha|
\end{aligned}
$$

Claramente, la longitud de la concatenación entre dos cadenas es la suma de sus longitudes:
$$|\alpha \cdot \beta| = |\alpha| + |\beta|$$

\subsubsection{Reverso}

Dada una cadena $\alpha = a_1 a_2 ... a_k$, su \textbf{reverso} es la cadena formada por los símbolos de $\alpha$ en el orden inverso: $\alpha^r = a_k a_{k-1} ... a_1$. Se puede definir de forma inductiva:
$$
\begin{aligned}
    \lambda^r & := \lambda \\
    (a \cdot \alpha)^r & := \alpha^r \cdot a
\end{aligned}
$$

Tiene las siguientes propiedades:
\begin{itemize}
    \item Es \textbf{involutiva}: $(\alpha^r)^r = \alpha$.
    \item Es \textbf{antidistributiva} con respecto a la concatenación: $(\alpha \cdot \beta)^r = \beta^r \cdot \alpha^r$.
    \item $|\alpha^r| = |\alpha|$
\end{itemize}

\subsection{Clausura de Kleene}

Dado un alfabeto $\Sigma$, su \textbf{clausura de Kleene}, denotada $\Sigma^*$, es el conjunto de todas las cadenas formadas por símbolos de $\Sigma$. Se puede definir inductivamente de la siguiente manera:
$$
\begin{aligned}
    & \lambda \in \Sigma^* \\
    \forall a \in \Sigma, \alpha \in \Sigma^*,\ & (\alpha \in \Sigma^* \implies a \alpha \in \Sigma^*)
\end{aligned}
$$
Por otro lado, la \textbf{clausura positiva} de $\Sigma$, denotada por $\Sigma^+$, es el conjunto de todas las cadenas \textbf{no vacías} con símbolos de $\Sigma$. Es decir, $\Sigma^+ = \Sigma^* \setminus \{\lambda\}$.

\section{Lenguajes}

Dado un alfabeto $\Sigma$, un \textbf{lenguaje} $\L$ sobre ese alfabeto es simplemente un conjunto de cadenas de $\Sigma^*$ ($\L \subseteq \Sigma^*$).

\subsection{Operaciones entre Lenguajes}
\label{subsec-operaciones-lenguajes}

Si $\L$ y $\L'$ son dos lenguajes definidos sobre el mismo alfabeto $\Sigma^*$, se pueden realizar una serie de operaciones sobre los mismos.

\subsubsection{Operaciones de Conjuntos}

Las operaciones de \textbf{unión}, \textbf{intersección} y \textbf{complemento} son las mismas que se definen para cualquier conjunto:
$$
\begin{aligned}
    \L^c & := \{\alpha \in \Sigma^* \mid \alpha \notin \L\} \\
    \L \cup \L' & := \{\alpha \in \Sigma^* \mid \alpha \in \L \lor \alpha \in \L'\} \\
    \L \cap \L' & := \{\alpha \in \Sigma^* \mid \alpha \in \L \land \alpha \in \L'\} \\
\end{aligned}
$$

\subsubsection{Concatenación}

La \textbf{concatenación} entre los lenguajes $\L$ y $\L'$, denotada $\L \cdot \L'$ o simplemente $\L\L'$, es el lenguaje formado por las cadenas resultantes de concatenar un elemento de $\L$ con uno de $\L'$:
$$\L \cdot \L' := \{\alpha \cdot \beta \mid \alpha \in \L, \beta \in \L'\}$$

Tiene las siguientes propiedades:
\begin{itemize}
    \item Es \textbf{asociativa}: $\L \cdot (\L' \cdot \L'') = (\L \cdot \L') \cdot \L''$
    \item Es \textbf{distributiva} con respecto a la unión: $\L \cdot (\L' \cup \L'') = \L \cdot \L' \cup \L \cdot \L''$
\end{itemize}

La definición de \textbf{repetición}/\textbf{potencia} de cadenas se puede extender a lenguajes:
$$
\begin{aligned}
    \L^0 & := \{\lambda\} \text{ (excepto para $\L = \emptyset$)}\\
    \L^n & := \L \cdot \L^{n - 1} = \{\alpha_1 \cdot \alpha_2 \cdots \alpha_n \mid \alpha_i \in \L\ \forall i \in \{1, ..., n\}\}
\end{aligned}
$$

\subsubsection{Clausura de Kleene}

La \textbf{clausura de Kleene} y la \textbf{clausura positiva} también se pueden extender a lenguajes:
$$
\begin{aligned}
    \L^* & := \bigcup_{i=0}^\infty \L^i = \{\alpha_1 \cdot \alpha_2 \cdots \alpha_n \mid n \in \N,\ \alpha_i \in \L\ \forall i \in \{1, ..., n\}\} \\
    \L^+ & := \bigcup_{i=1}^\infty \L^i = \L \cdot \L^*
\end{aligned}
$$
Es decir, la clausura de Kleene de un lenguaje son todas las cadenas que se pueden formar concatenando 0 o más de sus elementos.

\subsubsection{Reverso}

Por último, también se puede extender la operación de \textbf{reverso}: para un lenguaje $\L$, su reverso $\L^r$ tiene todas las cadenas que son reversos de elementos del lenguaje original. Es decir:
$$\L^r := \{\alpha^r \mid \alpha \in \L\}$$

\subsection{Gramáticas}

Una \textbf{gramática} $G$ es una forma concisa de definir un lenguaje. Concretamente, es una 4-upla $\langle V_N, V_T, P, S \rangle$, cuyos componentes representan:
\begin{itemize}
    \item $V_N$ es el alfabeto de \textbf{símbolos no-terminales}: éstos se pueden considerar ``variables sintácticas'', y deben ser reemplazados por medio de producciones para generar una cadena. Se suelen denotar por letras en mayúscula ($A, B, C, ...$).
    \item $V_T$ es el alfabeto de \textbf{símbolos terminales}: éstos son los que forman parte de las cadenas producidas por la gramática.
    \item $P$ es el conjunto de \textbf{producciones}, que son de la forma $\alpha \to \beta$ donde, $\alpha, \beta \in (V_N \cup V_T)^*$\footnote{En realidad, $\alpha$ debe tener al menos un símbolo no-terminal, así que pertenece al conjunto $(V_N \cup V_T)^* V_N (V_N \cup V_T)^*$}. $\alpha$ es denominada la \textbf{cabeza} y $\beta$ es el \textbf{cuerpo}.
    \item $S \in V_N$ es el símbolo inicial de la gramática.
\end{itemize}

El \textbf{lenguaje generado} por la gramática $\L(G) \subseteq V_T^*$ es el conjunto de cadenas que se forman al empezar por $S$ y aplicando una serie de \textbf{derivaciones} hasta llegar a una cadena de símbolos terminales.

\subsubsection{Derivaciones}

Dada una gramática $G$, la derivación $\underset{G}{\Rightarrow}$ es una relación entre cadenas de $(V_N \cup V_T)^*$\footnote{Como las cabezas de las producciones deben tener al menos un símbolo no-terminal, $\underset{G}{\Rightarrow} \subseteq (V_N \cup V_T)^* V_N (V_N \cup V_T)^* \times (V_N \cup V_T)^*$.}, y está definida por:
$$\gamma_1 \alpha \gamma_2 \underset{G}{\Rightarrow} \gamma_1 \beta \gamma_2 \iff \alpha \to \beta \in P$$
La clausura transitiva de esta (o cualquier) relación es la mínima\footnote{Mínima en términos de inclusión.} relación que contiene a $\underset{G}{\Rightarrow}$ y es transitiva: se denota $\overset{+}{\underset{G}{\Rightarrow}}$. Por otro lado, la clausura reflexo-transitiva $\overset{*}{\underset{G}{\Rightarrow}}$ está dada por $\mathrm{\overset{*}{\underset{G}{\Rightarrow}}} = \mathrm{\overset{+}{\underset{G}{\Rightarrow}}} \cup \{(\alpha, \alpha) \mid \alpha \in (V_N \cup V_T)^*\}$. Concretamente,
$$
\begin{aligned}
    \alpha \overset{+}{\underset{G}{\Rightarrow}} \beta & \iff \exists \gamma_1, ..., \gamma_k \in (V_N \cup V_T)^* \mid \alpha \underset{G}{\Rightarrow} \gamma_1 \underset{G}{\Rightarrow} \cdots \underset{G}{\Rightarrow} \gamma_k \underset{G}{\Rightarrow} \beta \\
    \alpha \overset{*}{\underset{G}{\Rightarrow}} \beta & \iff \alpha \overset{+}{\underset{G}{\Rightarrow}} \beta \lor \alpha = \beta
\end{aligned}
$$
Por ende, la gramática generada por el lenguaje $G$ se puede expresar como:
$$\L(G) := \{\alpha \in V_T^* \mid S \overset{+}{\underset{G}{\Rightarrow}} \alpha\}$$

\subsection{Jerarquía de Chomsky}

El espacio de todos los lenguajes definidos sobre un alfabeto dado se puede particionar en la llamada \textbf{jerarquía de Chomsky}. Está compuesta por 4 niveles, cada uno contenido en el siguiente.

\begin{figure}[H]
    \centering
    \includegraphics*[width=0.6\textwidth]{jerarquía-chomsky.png}
    \caption*{Diagrama de Venn de la jerarquía de Chomsky.}
\end{figure}

\subsubsection{Lenguajes Regulares}
\label{lenguaje-regular}

Los \textbf{lenguajes regulares} son aquellos que son generados por las gramáticas en las que todas las producciones son de la forma:
$$
\begin{aligned}
    A & \to b \\
    A & \to b A \\
    A & \to \lambda
\end{aligned}
$$
Donde $A \in V_N, b \in V_T$.

Las gramáticas que cumplen dichas reglas se denominan \textbf{regulares a derecha}: si, en lugar de permitir $A \to b A$, se permiten las producciones $A \to A b$, se obtienen las gramáticas \textbf{regulares a izquierda}. Ambos tipos de gramática son capaces de expresar cualquier lenguaje regular (y por ende tienen el mismo poder expresivo).

% TODO: Referencia
Como se explica \textbf{más adelante}, estos son exactamente los lenguajes que se pueden reconocer por autómatas finitos.

\subsubsection{Lenguajes Libres de Contexto}

Los \textbf{lenguajes libres de contexto} son los generados por gramáticas donde todas las producciones son de la forma:
$$A \to \beta$$
Con $A \in V_N, \beta \in (V_N \cup V_T)^*$.

% TODO: Referencia
Como se explica \textbf{más adelante}, estos son exactamente los lenguajes que se pueden reconocer por \textbf{autómatas de pila}.

\subsubsection{Lenguajes Sensitivos al Contexto}

Los \textbf{lenguajes sensitivos al contexto} son los generados por grámaticas donde todas las producciones son de la forma:
$$\alpha A \beta \to \alpha \gamma \beta$$
Donde $A \in V_N, \gamma \in (V_N \cup V_T)^+$ ($\gamma$ no puede ser la cadena vacía). Además, se permite la producción $S \to \lambda$.

Estos lenguajes son exactamente los lenguajes que se pueden reconocer por \textbf{máquinas de Turing que están limitadas linealmente en espacio} (en vez de tener una cinta infinita, sólo pueden usar una porción dada por una función lineal de la entrada).

\subsubsection{Lenguajes Recursivamente Enumerables}

Los \textbf{lenguajes recursivamente enumerables} incluyen a cualquier lenguaje que se puede expresar en términos de una gramática. Corresponden a los conjuntos que reconoce alguna \textbf{máquina de Turing}.


\chapter{Lenguajes Regulares}

\section{Autómatas Finitos Determinísticos}
\label{definicion-afds}

Un \textbf{autómata finito determinístico} (AFD) $M$ es una 5-upla $\langle Q, \Sigma, \delta, q_0, F \rangle$ donde:
\begin{itemize}
    \item $Q$ es el conjunto finito de \textbf{estados del autómata}.
    \item $\Sigma$ es el \textbf{alfabeto de entrada}.
    \item $\delta: Q \times \Sigma \to Q$ es la \textbf{función de transición}: si el autómata se encuentra en el estado $q \in Q$ y lee el símbolo $a \in \Sigma$ de la entrada, pasa al estado $\delta(q, a)$.
    \item $q_0 \in Q$ es el \textbf{estado inicial}.
    \item $F \subseteq Q$ es el conjunto de \textbf{estados finales}/\textbf{estados de aceptación}.
\end{itemize}


Estos autómatas toman una cadena de entrada $w \in \Sigma^*$ e iteran por los símbolos de la misma. Empezando en el estado inicial, cada símbolo indica el siguiente estado a tomar. Si, al terminar de leer la entrada, el autómata se encuentra en un estado final, la cadena es \textbf{aceptada}. De lo contrario, es \textbf{rechazada}.

\subsubsection{Visualización}

Estos autómatas se suelen representar como \textbf{grafos} $G = \langle V, E \rangle$, donde:
\begin{itemize}
    \item Los \textbf{nodos} representan los estados del autómata ($V = Q$).
    \item Las \textbf{aristas} están dadas por la función de transición: $(v, w) \in E \iff \exists a \in \Sigma \mid \delta(v, a) = w$. Además, las aristas se etiquetan con los símbolos por los cuales se puede tomar la transición.
    \item El estado inicial se representa con una arista entrante que no viene de otro estado.
    \item Los estados finales se respresentan con doble borde.
\end{itemize}

\begin{figure}[H]
    \centering
    \begin{tikzpicture}
        \node[state, initial] (0) {$q_0$};
        \node[state, accepting, right of=0] (1) {$q_1$};
        \node[state, below right of=0] (2) {$q_2$};

        \draw   (0) edge[above] node{$a$} (1)
        (0) edge[above] node{$b$} (2)
        (1) edge[below, bend left, right=0.3] node{$a,b$} (2)
        (2) edge[above, bend left, left=0.3] node{$b$} (1)
        (2) edge[loop right] node{$a$} (2);
    \end{tikzpicture}
    \caption*{Visualización del autómata $M = \langle \{q_0, q_1, q_2\}, \{a, b\}, \delta, q_0, \{q_1\} \rangle$, donde la función de transición $\delta$ está dada por las flechas entre los estados.}
\end{figure}

\subsection{Configuraciones Instantáneas}

La ejecución de los AFDs se puede formalizar usando el concepto de las \textbf{configuraciones instantáneas}: $\langle q, s \rangle \in Q \times \Sigma^*$ representa el punto de la ejecución del autómata en el que se encuentra en el estado $q$ y tiene a $s$ como entrada restante.

La función de transición $\delta$ de un autómata $M$ se puede adaptar a una relación entre configuraciones instantáneas, la \textbf{relación de transición} $\vdash_M$:
$$
    (q, a \cdot w) \vdash_M (r, w) \iff \delta(q, a) = r
$$

Luego, el \textbf{lenguaje aceptado} por el autómata $M$, denotado $\L(M)$ se puede definir como:
$$
    \L(M) := \{ w \in \Sigma^* \mid \exists q_f \in F : (q_0, w) \vdash_M^* (q_f, \lambda) \}
$$

\subsubsection{Propiedades}
\label{subsubsec-propiedades-rel-transicion}

La relación de transición $\vdash$ cumple las siguientes propiedades:

\begin{flalign*}
     &  & (q, \alpha) \vdash^* (r, \lambda) \land (q, \alpha) \vdash^* (s, \lambda) & \implies r = s                                                                                         &  & \textbf{(Determinismo)}              \\
     &  & (q, \alpha) \vdash^* (r, \lambda) \land (r, \beta) \vdash^* (p, \lambda)  & \implies (q, \alpha \cdot \beta) \vdash^* (p, \lambda)                                                 &  & \textbf{(Concatenación)}             \\
     &  & (q, \alpha \cdot \beta) \vdash^* (p, \lambda)                             & \implies \exists r \in Q \mid (q, \alpha) \vdash^* (r, \lambda) \land (r, \beta) \vdash^* (p, \lambda) &  & \textbf{(Siempre toma algún estado)} \\
     &  & (q, \alpha) \vdash^n (r, \lambda)                                         & \iff |\alpha| = n                                                                                      &  & \textbf{(Linealidad)}                \\
     &  & (q, \alpha) \vdash^* (q, \lambda)                                         & \implies \forall i \in \N,\ (q, \alpha^i) \vdash^* (q, \lambda)                                          &  & \textbf{(Invarianza)}
\end{flalign*}

\subsection{Función de Transición Extendida}

Una forma alternativa (y análoga) de dar el lenguaje aceptado es por medio de la \textbf{función de transición extendida}. Esta es una función $\hat \delta : Q \times \Sigma^* \to \mathcal P (Q)$ (toma cadenas en vez de símbolos) definida de la siguiente manera:
$$
    \begin{aligned}
        \hat \delta(q, \lambda)   & := q                              \\
        \hat \delta(q, w \cdot a) & := \delta (\hat \delta (q, w), a)
    \end{aligned}
$$
Donde $q \in Q, a \in \Sigma, w \in \Sigma^*$.

Bajo esta definición el lenguaje aceptado por un autómata $M$ es simplemente:
$$
    \L (M) := \{w \in \Sigma^* \mid \hat \delta (q_0, w) \in F \}
$$

\subsubsection{Propiedades}

La función de transición extendida cumple propiedades similares a las de la relación de transición:
\begin{flalign*}
     &  & \hat \delta(q, \alpha \cdot \beta) & = \hat \delta(\hat \delta(q, \alpha), \beta)             &  & \textbf{(Concatenación)} \\
     &  & \hat \delta(q, \alpha) = q         & \implies \forall i \in \N,\ \hat \delta(q, \alpha^i) = q &  & \textbf{(Invarianza)}
\end{flalign*}

\subsection{Autómatas Incompletos}

Según la \hyperref[definicion-afds]{definición anterior}, la función de transición $\delta$ debe definir un estado siguiente para cada combinación de estados y símbolos. Sin embargo, a veces resulta cómodo tener estados en los cuales recibir ciertos símbolos cause un rechazo inmediato de la cadena. Una forma de lograr eso es permitir que $\delta$ sea una \textbf{función parcial}, es decir, que se indefina para ciertas entradas. Un autómata cuya función de transición es parcial se denomina \textbf{incompleto}.

\begin{figure}[H]
    \centering
    \begin{tikzpicture}
        \node[state, initial] (0) {$q_0$};
        \node[state, right of=0] (1) {$q_1$};
        \node[state, accepting, right of=1] (2) {$q_2$};

        \draw   (0) edge[above] node{$a$} (1)
        (1) edge[above] node{$b$} (2);
    \end{tikzpicture}
    \caption*{Autómata incompleto que solamente acepta la cadena $ab$.}
\end{figure}

No obstante, la distinción entre autómatas completos e incompletos no es muy importante gracias al siguiente teorema:

\begin{theorem*}
    Para cualquier autómata incompleto $M$, existe un autómata completo $M'$ equivalente, es decir, $\L(M) = \L(M')$.
\end{theorem*}
\begin{proof}
    Si $M = \langle Q, \Sigma, \delta, q_0, F \rangle$, basta con tomar el autómata $M' = \langle Q \cup \{q_T\}, \Sigma, \delta', q_0, F \rangle$, donde $q_T$ es un nuevo estado, llamado \textbf{estado trampa}. La función de transición se extiende de la siguiente manera:
    $$
        \delta' (q, a) :=
        \begin{cases}
            \delta(q, a) & \si q \in Q \land \delta(q, a) \neq \indef \\
            q_T          & \si q = q_T \lor \delta(q, a) = \indef
        \end{cases}
    $$
    Veamos que $\L(M) = \L(M')$, demostrando $w \in \L(M) \iff w \in \L(M')$.

    $\implies$) Si $w \in \L(M)$, el autómata debe pasar por una serie de \textbf{transiciones definidas}, hasta llegar a un estado final. Luego, por la definición de $\delta'$, las transiciones entre esos estados son las mismas en $M'$, así que se llega al mismo estado, y por ende $w \in \L(M')$. Formalmente,
    $$
        \begin{aligned}
            a_1 \dots a_k \in \L(M) \implies & \exists q_f \in F : (q_0, a_1 \dots a_k) \vdash_M^* (q_f, \lambda)                                                         \\
            \iff                             & \exists q_1, \dots, q_k \in Q : (q_0, a_1 \dots a_k) \vdash_M (q_1, a_2 \dots a_k) \vdash_M \cdots \vdash_M (q_k, \lambda) \\
                                             & \land q_k \in F
        \end{aligned}
    $$
    Tomando la secuencia de estados $q_0, q_1 \dots q_k$, se tiene:
    $$
        \begin{aligned}
             & (q_i, a_{i+1} \dots a_k) \vdash_M (q_{i+1}, a_{i+2} \dots a_k) \iff \delta(q_i, a_{i+1}) = q_{i+1} \implies \delta'(q_i, a_{i+1}) = q_{i+1} \\
             & \implies (q_i, a_{i+1} \dots a_k) \vdash_{M'} (q_{i+1}, a_{i+2} \dots a_k)
        \end{aligned}
    $$
    Esto implica que $(q_0, w) \vdash_{M'}^* (q_k, \lambda)$, y entonces $w \in \L(M')$.

    $\impliedby$) Supongamos que $w \notin \L(M)$. Hay 2 posibilidades:
    \begin{itemize}
        \item Se tiene $(q_0, w) \vdash_M^* (q_k, \lambda)$ con $q_k \notin F$. En ese caso, el autómata nunca se indefine y, por un razonamiento análogo, $(q_0, w) \vdash_{M'}^* (q_k, \lambda)$, así que $w \notin \L(M')$.
        \item Por otro lado, puede suceder que el autómata se indefina durante la ejecución. En tal caso, se tiene una secuencia de estados $q_1, \dots, q_j$:
              $$(q_0, a_1 \dots a_k) \vdash_M (q_1, a_2 \dots a_k) \vdash_M \cdots \vdash_M (q_j, a_{j+1} \dots a_k)$$
              Con $\delta(q_j, a_{j+1}) = \indef$.

              En este caso, $\delta' (q_j, a_{j+1}) = q_T$ y, como $\delta'(q_T, a) = q_T\ \forall a \in \Sigma$, el autómata sigue en ese estado hasta consumir el resto de la cadena. Considerando que la ejecución se da de la misma forma en $M'$ antes de la indefinición, se tiene:
              $$(q_0, a_1 \dots a_k) \vdash_{M'}^+ (q_j, a_{j+1} \dots a_k) \vdash_{M'} (q_T, a_{j+2} \dots a_k) \vdash_{M'}^* (q_T, \lambda)$$
              Como $q_T \notin F$, $w \notin \L(M')$.
    \end{itemize}
\end{proof}

Este resultado implica que los autómatas incompletos tienen el mismo poder expresivo que los completos. Además, la construcción del autómata completo a partir de uno incompleto es computacionalmente barata: sólo requiere agregar un nuevo estado y conectarlo con a lo sumo $|Q|$ estados.

\begin{figure}[H]
    \centering
    \begin{tikzpicture}
        \node[state, initial] (0) {$q_0$};
        \node[state, right of=0] (1) {$q_1$};
        \node[state, accepting, right of=1] (2) {$q_2$};
        \node[state, below of=1] (3) {$q_T$};

        \draw   (0) edge[above] node{$a$} (1)
        (1) edge[above] node{$b$} (2)
        (0) edge[below, bend right] node{$b$} (3)
        (1) edge[right] node{$a$} (3)
        (2) edge[below, bend left] node{$a,b$} (3)
        (3) edge[loop below] node{$a,b$} (3);
    \end{tikzpicture}
    \caption*{Versión completa del autómata anterior, obtenida a partir de agregar un estado trampa.}
\end{figure}


\section{Autómatas Finitos No Determinísticos}

Un \textbf{autómata finito no determinístico} (AFND) $M$ es una 5-upla $\langle Q, \Sigma, \delta, q_0, F \rangle$. Sus componentes tienen la misma semántica que para los AFDs, pero en este caso se tiene $\delta : Q \times \Sigma \to \mathcal P (Q)$, es decir, el codominio de $\delta$ son los subconjuntos de $Q$. Esto es porque, cuando el autómata está en el estado $q$ y lee el símbolo $a$, es capaz tomar alguna de varias transiciones posibles.

\begin{figure}[H]
    \centering
    \begin{tikzpicture}
        \node[state, initial] (0) {$q_0$};
        \node[state, right of=0] (1) {$q_1$};
        \node[state, accepting, right of=1] (2) {$q_2$};

        \draw   (0) edge[above, loop above] node{$a,b$} (0)
        (0) edge[above] node{$a$} (1)
        (1) edge[above] node{$b$} (2);
    \end{tikzpicture}
    \caption*{AFND que acepta cualquier cadena que termina en $ab$. \\ Se puede ver que $\delta(q_0, a) = \{q_0, q_1\}$.}
\end{figure}

Este cambio requiere re-definir la \textbf{relación de transición} entre configuraciones instantáneas $\vdash_M$ (ésta deja de ser una función):
$$
    (q, a \cdot w) \vdash_M (r, w) \iff r \in \delta(q, a)
$$

La relación de transición cumple las mismas \hyperref[subsubsec-propiedades-rel-transicion]{propiedades} que las del caso determinístico (excepto por la de determinismo, obviamente).

También se debe volver a definir la \textbf{función de transición extendida} $\hat \delta : Q \times \Sigma^* \to \mathcal P (Q)$:
$$
    \begin{aligned}
        \hat \delta(q, \lambda)   & := \{q\}                                                                                                                    \\
        \hat \delta(q, w \cdot a) & := \{p \in Q \mid \exists r \in \hat \delta (q, w) : p \in \delta (r, a)\} = \bigcup_{r \in \hat \delta(q, w)} \delta(r, a)
    \end{aligned}
$$

El lenguaje aceptado por el autómata $M$ es el conjunto de cadenas en las que \textbf{alguno} de todos los posibles caminos de ejecución termina en un estado final. Es decir,
$$
    w \in \L(M) \iff \hat \delta(q_0, w) \cap F \neq \emptyset \iff \exists q_f \in F \mid (q_0, w) \vdash_M^* (q_f, \lambda)
$$

Notemos que esta definición automáticamente permite tener autómatas incompletos: basta con tomar $\delta(q, a) = \emptyset$.

\subsection{Transiciones Lambda}

Un autómata finito no determinístico con transiticiones-$\lambda$ (AFND-$\lambda$) es uno en el cual la función de transición es de la forma $delta : Q \times (\Sigma \cup \{\lambda\}) \to \mathcal P (Q)$. Se agrega un ``símbolo'' $\lambda$ que se puede tomar \textbf{sin consumir ningún símbolo} de la entrada.

\begin{figure}[H]
    \centering
    \begin{tikzpicture}
        \node[state, initial] (0) {$q_0$};
        \node[state, accepting, above right of=0] (1) {$q_1$};
        \node[state, right of=1] (2) {$q_2$};
        \node[state, accepting, below right of=0] (3) {$q_1$};
        \node[state, right of=3] (4) {$q_2$};

        \draw   (0) edge[above, bend left] node{$\lambda$} (1)
        (1) edge[loop above] node{$b$} (1)
        (1) edge[above, bend left] node{$a$} (2)
        (2) edge[below, bend left] node{$a$} (1)
        (2) edge[loop above] node{$b$} (2)
        (0) edge[below, bend right] node{$\lambda$} (3)
        (3) edge[loop above] node{$a$} (3)
        (3) edge[above, bend left] node{$b$} (4)
        (4) edge[below, bend left] node{$b$} (3)
        (4) edge[loop above] node{$a$} (4);
    \end{tikzpicture}
    \caption*{AFND que acepta las cadenas que tienen cantidad par de $a$s o cantidad par de $b$s.}
\end{figure}

Esto se puede formalizar por medio de la \textbf{relación de transición} entre configuraciones instantáneas:
$$
    (q, w) \vdash_M (r, w') \iff (w = a \cdot w' \land r \in \delta(q, a)) \lor (w = w' \land r \in \delta(q, \lambda))
$$

Esta modificación hace que deje de valer la \hyperref[subsubsec-propiedades-rel-transicion]{propiedad de linealidad}.

Para definir la función de transición extendida, primero conviene introducir la \textbf{clausura lambda}: $Cl_\lambda(q)$ es el conjunto de estados que son alcanzables desde $q$ por medio de transiciones-$\lambda$. Si se define la relación:
$$q \sim_\lambda r \iff r \in \delta(q, \lambda)$$
Entonces $r \in Cl_\lambda(q) \iff q \sim_\lambda^* r$. También se puede extender la definición para considerar conjuntos de estados: $Cl_\lambda(S) := \bigcup_{q \in S} Cl_\lambda(q)$.

La \textbf{función de transición extendida} $\hat \delta : Q \times (\Sigma \cup \{\lambda\}) \to \mathcal P (Q)$ pasa a ser:
$$
    \begin{aligned}
        \hat \delta (q, \lambda)   & := Cl_\lambda(q)                                                          \\
        \hat \delta (q, w \cdot a) & := Cl_\lambda\left(\bigcup_{r \in \hat \delta(q, w)} \delta(r, a) \right)
    \end{aligned}
$$

\subsubsection{Equivalencia con AFNDs}

Obviamente, cualquier AFND puede ser interpretado como un AFND-$\lambda$. Lo sorprendente es que esto también vale para el otro lado: resulta que agregar transiciones-$\lambda$ no aumenta el poder expresivo de los AFNDs.

\begin{theorem*}
    Dado un AFND-$\lambda$ $M = \langle Q, \Sigma, \delta, q_0, F \rangle$, existe un AFND sin transiciones lambda equivalente.
\end{theorem*}
\begin{proof}
    Hay igual que antes, la demostración es constructiva: exhibimos un autómata $M' = \langle Q, \Sigma, \delta', q_0, F' \rangle$ que cumple $\L(M) = \L(M')$. El conjunto de estados y el estado inicial son los mismos, pero la función de transición y el conjunto de estados finales se reemplazan por:
    $$
        \begin{aligned}
            \delta'(q, a) & := \hat \delta (q, a) = Cl_\lambda\left(\bigcup_{r \in Cl_\lambda(q)} \delta(r, a)\right) \\
            F'            & := \begin{cases}
                                   F              & \si Cl_\lambda (q_0) \cap F = \emptyset \\
                                   F \cup \{q_0\} & \ecc
                               \end{cases}
        \end{aligned}
    $$
    En tal caso, la función de transición extendida del nuevo autómata es:
    $$
        \begin{aligned}
            \hat \delta'(q, \lambda)   & := \{q\}                                              \\
            \hat \delta'(q, w \cdot a) & := \bigcup_{r \in \hat \delta'(q, w)} \delta'(r, a) =
        \end{aligned}
    $$
    Veamos que $\hat \delta'(q, w) = \hat \delta(q, w) \ \forall w \in \Sigma^+$ por inducción en la longitud de las cadenas:
    \begin{itemize}
        \item Cuando $|w| = 1$, se tiene $w = a \in \Sigma$, y luego:
              $$
                  \hat \delta'(q, \lambda \cdot a)
                  = \bigcup_{r \in \underbrace{\hat \delta'(q, \lambda)}_{\{q\}} } \delta'(r, a)
                  = \delta'(q, a) = \hat \delta(q, a)
              $$
        \item Suponiendo que vale para cualquier cadena de longitud $n$, y siendo $w$ una de esas cadenas:
              $$
                  \begin{aligned}
                      \hat \delta'(q, w \cdot a)
                       & = \bigcup_{r \in \hat \delta'(q, w)} \delta'(r, a)
                      \overset{HI}{=} \bigcup_{r \in \hat \delta(q, w)} \delta'(r, a)
                      = \bigcup_{r \in \hat \delta(q, w)} \hat \delta(r, a)                                                                                                                                                                                              \\
                       & = \bigcup_{r \in \hat \delta(q, w)} Cl_\lambda\left(\bigcup_{r' \in Cl_\lambda(r)} \delta(r', a)\right) = \bigcup_{r \in \hat \delta(q, w)} \bigcup_{r' \in \underbrace{Cl_\lambda(r)}_{\subseteq \hat \delta(q, w)}} Cl_\lambda(\delta(r', a)) \\
                       & = \bigcup_{r \in \hat \delta(q, w)} Cl_\lambda(\delta(r, a)) = Cl_\lambda\left(\bigcup_{r \in \hat \delta(q, w)} \delta(r, a)\right) = \hat \delta (q, w \cdot a)
                  \end{aligned}
              $$
    \end{itemize}

    Esto implica que, para cualquier cadena $w \in \Sigma^+$, se tiene:
    $$w \in \L(M) \iff \hat \delta(q_0, w) \in F \iff \hat \delta'(q_0, w) \cap F \neq \emptyset$$

    Entonces:
    \begin{itemize}
        \item Si $w \in \L(M)$ y $\hat \delta'(q_0, w) \cap F \neq \emptyset$, como $F \subseteq F'$, $\hat \delta'(q_0, w) \cap F' \neq \emptyset \implies w \in \L(M')$.
        \item Si $w \notin \L(M)$ y $\hat \delta'(q_0, w) \cap F = \emptyset$, la única forma de que $\hat \delta'(q_0, w) \cap F' \neq \emptyset$ es que $q_0 \in F' \cap \hat \delta'(q_0, w)$. Sin embargo, cuando vale $q_0 \in F'$ es porque $Cl_\lambda(q_0) \cap F \neq \emptyset$, lo cual implica que existe alguna secuencia de transiciones lambda de la forma:
              $$(q_0, \lambda) \vdash_M \dots \vdash (q_f, \lambda)$$
              Con $q_f \in F'$. Como $q_0 \in \hat \delta'(q_0, w)$, también vale que $(q_0, w) \vdash^* (q_0, \lambda)$. Por ende, se tiene:
              $$(q_0, w) \vdash_M^* (q_0, \lambda) \vdash_M^* (q_f, \lambda) \implies w \in \L(M)$$
              Pero, por hipótesis, $w \notin \L(M)$. Llegamos a un \textbf{absurdo}, que vino de suponer que $\hat \delta'(q_0, w) \cap F' \neq \emptyset$. Por ende, $\hat \delta'(q_0, w) \cap F' = \emptyset \implies w \notin \L(M')$.
    \end{itemize}

    En ambos casos, se tiene:
    $$w \in \L(M) \iff w \in \L(M')$$
\end{proof}

La construcción realizada en la demostración puede ser adaptada fácilmente a un algoritmo que elimina las transiciones lambda de un autómata dado. Una implementación \textit{naïve}, que recorre todos los estados y explora las clausuras lambda correspondientes, tendría una complejidad temporal acotada por $\BigO{|Q|^2}$.

\subsection{Equivalencia con AFDs}

Es fácil ver que a partir de un AFD $M$ se puede construir un AFND $M'$ equivalente tomando $\delta'(q, a) = \{\delta(q, a)\}$. A continuación, veremos que a partir de un AFND se puede construir un AFD equivalente:

\begin{theorem*}
    Dado un AFND $M = \langle Q, \Sigma, \delta, q_0, F \rangle$, existe un AFD $M'$ tal que $\L(M) = \L(M')$.
\end{theorem*}
\begin{proof}
    Nuevamente, la demostración será constructiva: tomamos $M' = \langle Q', \Sigma, \delta', q_0', F' \rangle$, donde
    \begin{itemize}
        \item $Q' = \mathcal P (Q)$.
        \item $F' = \{S \in Q' \mid S \cap F \neq 0\}$.
        \item $q_0' = \{q_0\}$
        \item $\delta'(S, a) = \bigcup_{q \in S} \delta(q, a)$
    \end{itemize}

    Antes de demostrar que este autómata es equivalente, veamos que, para cualquier cadena $w \in \Sigma^*$,
    $$\hat \delta'(q_0', w) = \hat \delta(q_0, w)$$

    La demostración será por inducción en $|w|$:
    \begin{itemize}
        \item Si $|w| = 0 \iff w = \lambda$, se tiene:
              $$\hat \delta'(q_0', \lambda) = q_0' = \{q_0\} = \hat \delta(q_0, \lambda)$$
        \item Si vale para $w$ con $|w| = n$,
              $$\hat \delta'(q_0', w \cdot a) = \delta'(\hat \delta(q_0', w), a) = \bigcup_{q \in \hat \delta'(q_0', w)} \delta(q, a) \overset{HI}{=} \bigcup_{q \in \hat \delta(q_0, w)} \delta(q, a) = \hat \delta(q_0, w \cdot a)$$
    \end{itemize}

    Teniendo este resultado, la demostración de equivalencia es bastante directa:
    $$w \in \L(M) \iff \hat \delta(q_0, w) \cap F \neq \emptyset \iff \hat \delta'(q_0', w) \cap F \neq \emptyset \iff \hat \delta'(q_0', w) \in F' \iff w \in \L(M')$$
\end{proof}

Al igual que en los casos anteriores, esta demostración se puede adaptar a un algoritmo que \textbf{determiniza} un AFND. Sin embargo, a diferencia de las equivalencias anteiores, este procedimiento es más costoso: el nuevo AFD tiene $2^{|Q|}$ estados\footnote{La cantidad de estados se puede disminuir si se descartan los \textbf{estados inaccesibles} (aquellos para los que no hay un camino desde $q_0$). Aún así, es posible que el autómata quede con $\BigO{2^{|Q|}}$ estados.}.



\section{Expresiones Regulares}

Una \textbf{expresión regular} (o \textit{regex}) $r$ es una forma compacta de expresar lenguajes sobre un alfabeto $\Sigma$. Se definen de la siguiente manera:
\begin{itemize}
    \item ``$\emptyset$'' denota el lenguaje vacío.
    \item ``$\lambda$'' denota el lenguaje $\{\lambda\}$.
    \item Para cada símbolo $a \in \Sigma$, ``$a$'' denota el lenguaje $\{a\}$.
    \item Si $r$ y $s$ son expresiones regulares:
          \begin{itemize}
              \item $r|s$ denota la \textbf{unión} de sus lenguajes: $\L(r|s) := \L(r) \cup \L(s)$.
              \item $rs$ denota la \textbf{concatenación} de sus lenguajes: $\L(rs) := \L(r) \cdot \L(s)$.
              \item $r^*$ denota la \textbf{clausura de Kleene}: $\L(r^*) := \L(r)^*$.
              \item $r^+$ denota la \textbf{clausura positiva}: $\L(r^+) := \L(r)^+$.
          \end{itemize}
\end{itemize}

El \textbf{orden de precedencia} de estos operadores es $\square^*, \square^+, \cdot, |$. Se pueden asociar de otra forma por medio de paréntesis: por ejemplo, la expresión $ab^*$ denota todas las cadenas que empiezan con $a$ y terminan en 0 o más $b$s, mientras que la expresión $(ab)^*$ denota todas las cadenas formadas por $0$ o más repeticiones de $ab$.

\subsection{Equivalencia con Autómatas Finitos}

Los lenguajes reconocibles por expresiones regulares exactamente los mismos lenguajes reconocibles por autómatas finitos. Para demostrar esto, demostraremos que:
\begin{itemize}
    \item $\L$ es expresable por expresión regular $\implies$ $\L$ es expresable por autómata finito.
    \item $\L$ es expresable por autómata finito $\implies$ $\L$ es expresable por gramática regular.
    \item $\L$ es expresable por gramática regular $\implies$ $\L$ es expresable por expresión regular.
\end{itemize}
Es decir, como efecto colateral, también demostraremos que éstos son los lenguajes expresables por gramáticas regulares (los \textbf{lenguajes regulares}).

\subsubsection{Expresión Regular $\implies$ Autómata Finito}

Primero veamos que, dada una expresión regular, siempre se puede obtener un autómata que reconoce el mismo lenguaje

\begin{theorem*}
    Dada una expresión regular $r$, existe un AFND-$\lambda$ equivalente, es decir, uno para el cual $\L(r) = \L(M)$.
\end{theorem*}
\begin{proof}
    La demostración será por inducción estructural sobre $r$. Además, demostraremos que siempre se puede construir un autómata con un único estado final.

    Los casos base son:
    \begin{itemize}
        \item $r = \emptyset$. Se puede tomar el siguiente $M$:
              \begin{figure}[H]
                  \centering
                  \begin{tikzpicture}
                      \node[state, initial] (0) {$q_0$};
                      \node[state, accepting, right of=0] (1) {$q_1$};
                  \end{tikzpicture}
              \end{figure}

              Como el único estado final es inaccesible, ninguna cadena es aceptada.
        \item $r = \lambda$. Se puede tomar el siguiente $M$:
              \begin{figure}[H]
                  \centering
                  \begin{tikzpicture}
                      \node[state, initial, accepting] (0) {$q_0$};
                  \end{tikzpicture}
              \end{figure}

              En este caso, la cadena vacía es aceptada, pero la lectura de cualquier símbolo hace que el autómata se indefina, por lo cual las cadenas no vacías son rechazadas.
        \item $r = a$, con $a \in \Sigma$. Se puede tomar el siguiente $M$:
              \begin{figure}[H]
                  \centering
                  \begin{tikzpicture}
                      \node[state, initial] (0) {$q_0$};
                      \node[state, accepting, right of=0] (1) {$q_1$};

                      \draw   (0) edge[above] node{$a$} (1);
                  \end{tikzpicture}
              \end{figure}

              La única cadena que acepta es la que tiene solamente al símbolo $a$.
    \end{itemize}

    Por otro lado, Dadas dos expresiones $r$ y $s$ cuyos autómatas correspondientes son:
    \begin{itemize}
        \item $M_r = \langle Q_r, \Sigma, \delta_r, q_{0r}, \{q_{fr}\} \rangle$
        \item $M_s = \langle Q_s, \Sigma, \delta_s, q_{0s}, \{q_{fs}\} \rangle$
    \end{itemize}

    Entonces se pueden obtener autómatas para cada una de las operaciones:
    \begin{itemize}
        \item $p = r|s$. Se toma el autómata $M_p = \langle Q_r \cup Q_s \cup \{q_{0p}, q_{fp}\}, \Sigma, \delta_p, q_{0p}, \{q_{fp}\} \rangle$, con $\delta_p$ dada por:
              $$
                  \delta_p(q, a) :=
                  \begin{cases}
                      \delta_r(q, a)     & \si q \in Q_r \setminus \{q_{fr}\}             \\
                      \delta_s(q, a)     & \si q \in Q_s \setminus \{q_{fs}\}             \\
                      \{q_{0r}, q_{0s}\} & \si q = q_{0p} \land a = \lambda               \\
                      \{q_{fp}\}         & \si q \in \{q_{fr}, q_{fs}\} \land a = \lambda \\
                      \emptyset          & \ecc
                  \end{cases}
              $$

              Esto puede ser visualizado de la siguiente manera:
              \begin{figure}[H]
                  \centering
                  \begin{tikzpicture}
                      \tikzstyle{surround} = [fill=gray!20,dotted,draw=black,rounded corners=6mm, inner sep=10pt]

                      \node[state, initial] (0) {$q_{0p}$};
                      \node[state, above right of=0] (1) {$q_{0r}$};
                      \node[state, below right of=0] (3) {$q_{0r}$};
                      \node[state, right of=1] (2) {$q_{fr}$};
                      \node[state, right of=3] (4) {$q_{fs}$};
                      \node[state, accepting, below right of=2] (5) {$q_{fp}$};

                      \draw   (0) edge[above, bend left] node{$\lambda$} (1)
                      (0) edge[below, bend right] node{$\lambda$} (3)
                      (1) edge[dotted] node{} (2)
                      (3) edge[dotted] node{} (4)
                      (2) edge[above, bend left] node{$\lambda$} (5)
                      (4) edge[below, bend right] node{$\lambda$} (5);

                      \begin{pgfonlayer}{background}
                          \node[surround] (background) [fit=(1)(2), label=$M_r$] {};
                          \node[surround] (background) [fit=(3)(4), label=below:$M_s$] {};
                      \end{pgfonlayer}
                  \end{tikzpicture}
              \end{figure}

              Es claro que este autómata sólo acepta las cadenas que son aceptadas por $M_r$ o $M_s$.

        \item $p = rs$. Se toma el autómata $M_p = \langle Q_r \cup Q_s \cup \{q_{0p}, q_{fp}\}, \Sigma, \delta_p, q_{0p}, \{q_{fp}\} \rangle$, con $\delta_p$ dada por:
              $$
                  \delta_p(q, a) :=
                  \begin{cases}
                      \delta_r(q, a) & \si q \in Q_r \setminus \{q_{fr}\} \\
                      \delta_s(q, a) & \si q \in Q_s \setminus \{q_{fs}\} \\
                      \{q_{0r}\}     & \si q = q_{0p} \land a = \lambda   \\
                      \{q_{0s}\}     & \si q = q_{fr} \land a = \lambda   \\
                      \{q_{fp}\}     & \si q = q_{fs} \land a = \lambda   \\
                      \emptyset      & \ecc
                  \end{cases}
              $$

              Puede ser visualizado de la siguiente manera:
              \begin{figure}[H]
                  \centering
                  \resizebox{0.8\textwidth}{!}{
                      \begin{tikzpicture}
                          \tikzstyle{surround} = [fill=gray!20,dotted,draw=black,rounded corners=6mm, inner sep=10pt]

                          \node[state, initial] (0) {$q_{0p}$};
                          \node[state, right of=0] (1) {$q_{0r}$};
                          \node[state, right of=1] (2) {$q_{fr}$};
                          \node[state, right of=2] (3) {$q_{0s}$};
                          \node[state, right of=3] (4) {$q_{fs}$};
                          \node[state, right of=4, accepting] (5) {$q_{fp}$};

                          \draw   (0) edge[above] node{$\lambda$} (1)
                          (1) edge[dotted] node{} (2)
                          (2) edge[above] node{$\lambda$} (3)
                          (3) edge[dotted] node{} (4)
                          (4) edge[above] node{$\lambda$} (5);

                          \begin{pgfonlayer}{background}
                              \node[surround] (background) [fit=(1)(2), label=$M_r$] {};
                              \node[surround] (background) [fit=(3)(4), label=$M_s$] {};
                          \end{pgfonlayer}
                      \end{tikzpicture}
                  }
              \end{figure}

              En este caso, el autómata sólo acepta si lee una cadena que es aceptada por $M_r$, seguida por una cadena aceptada por $M_s$. Éstas son justamente las cadenas en $\L(M_r) \cdot \L(M_s)$.

        \item $p = r^*$. Se toma el autómata $M_p = \langle Q_r \cup \{q_{0p}\}, \Sigma, \delta_p, q_{0p}, \{q_{0p}\} \rangle$, con $\delta_p$ dado por:
              $$
                  \delta_p(q, a) :=
                  \begin{cases}
                      \delta_r(q, a) & \si q \in Q_r \setminus \{q_{fr}\} \\
                      \{q_{0r}\}     & \si q = q_{0p} \land a = \lambda   \\
                      \{q_{0r}\}     & \si q = q_{fr} \land a = \lambda   \\
                      \emptyset      & \ecc
                  \end{cases}
              $$

              Puede ser visualizado de la siguiente manera:
              \begin{figure}[H]
                  \centering
                  \begin{tikzpicture}
                      \tikzstyle{surround} = [fill=gray!20,dotted,draw=black,rounded corners=6mm, inner sep=10pt]

                      \node[state, initial, accepting] (0) {$q_{0p}$};
                      \node[state, right of=0] (1) {$q_{0r}$};
                      \node[state, right of=1] (2) {$q_{fr}$};

                      \draw   (0) edge[above] node{$\lambda$} (1)
                      (1) edge[dotted] node{} (2)
                      (2) edge[below, bend left] node{$\lambda$} (0);

                      \begin{pgfonlayer}{background}
                          \node[surround] (background) [fit=(1)(2), label=$M_r$] {};
                      \end{pgfonlayer}
                  \end{tikzpicture}
              \end{figure}

              Es fácil ver que el autómata sólo acepta cadenas de la forma $w^i$ con $i \geq 0$ y $w \in \L(M_r)$, que forman el lenguaje $\L(M_r)^*$.

        \item $p = r^+$. En este caso, se puede descomponer la expresión en $p = rr^*$.
    \end{itemize}
\end{proof}

La construcción realizada en la demostración se conoce como el \textbf{algoritmo de construcción de Thompson}. Para cada componente de la expresión, se agrega una cantidad constante de estados al autómata. Por ende, el autómata construido para una expresión $r$ tiene $\BigO{|r|}$ estados (y el algoritmo tiene esa misma complejidad temporal).

\subsubsection{Autómata Finito $\implies$ Gramática Regular}

\textbf{TODO}

\subsubsection{Gramática Regular $\implies$ Expresión Regular}

\textbf{TODO}


\section{Propiedades de Clausura de Lenguajes Regulares}

Ahora que sabemos que los lenguajes regulares pueden ser generados por tanto expresiones regulares como autómatas finitos, podemos demostrar que esta clase de lenguajes está \textbf{cerrada} bajo ciertas operaciones, es decir, que aplicar éstas a operandos regulares resultará en otro lenguaje regular.

\subsection{Concatenación}

\begin{theorem*}
    Sean $\L_1, \L_2 \subseteq \Sigma^*$ lenguajes regulares. Su concatenación $\L_1 \cdot \L_2$ es regular.
\end{theorem*}
\begin{proof}
    Es trivial: sean $r_1, r_2$ expresiones regulares para $\L_1$ y $\L_2$. Entonces, la expresión $r_1 r_2$ reconoce al lenguaje $\L_1 \cdot \L_2$ y, por ende, éste es regular.
\end{proof}

\subsection{Unión}

\begin{theorem*}
    Sean $\L_1, \L_2 \subseteq \Sigma^*$ lenguajes regulares. Su unión $\L_1 \cup \L_2$ es regular.
\end{theorem*}
\begin{proof}
    La demostración es análoga al caso anterior: sean $r_1, r_2$ expresiones regulares para $\L_1$ y $\L_2$. Entonces, la expresión $r_1|r_2$ reconoce al lenguaje $\L_1 \cup \L_2$ y, por ende, éste es regular.
\end{proof}

\subsection{Clausura de Kleene/Positiva}

\begin{theorem*}
    Sea $\L \subseteq \Sigma^*$ un lenguaje regular. Tanto $\L^*$ como $\L^+$ son regulares.
\end{theorem*}
\begin{proof}
    Sea $r$ una expresión regular que reconoce $\L$. Entonces, $r^*$ y $r^+$ reconocen $\L^*$ y $\L^+$ respectivamente y, por ende, ambos lenguajes son regulares.
\end{proof}

\subsection{Complemento}

\begin{theorem*}
    Sea $\L \subseteq \Sigma^*$ un lenguaje regular. Su complemento $\L^c$ es regular.
\end{theorem*}
\begin{proof}
    Sea $M = \langle Q, \Sigma, \delta, q_0, F \rangle$ un autómata finito tal que $\L(M) = \L$. Luego, tomemos $M' = \langle Q, \Sigma, \delta, q_0, Q \setminus F \rangle$, es decir, un autómata idéntico con los estados finales invertidos. Entonces, una cadena es aceptada por $M'$ cuando:
    $$
        w \in \L(M') \iff \hat \delta(q_0, w) \in Q \setminus F \iff \delta(q_0, w) \notin F \iff w \notin \L(M)
    $$
    Por ende, el lenguaje aceptado por $M'$ es $\L^c$, y esto implica que ese lenguaje es regular (porque se puede construir un autómata finito que lo acepta).
\end{proof}

\subsection{Intersección}

\begin{theorem*}
    Sean $\L_1, \L_2 \subseteq \Sigma^*$ lenguajes regulares. Su intersección $\L_1 \cap \L_2$ es regular.
\end{theorem*}
\begin{proof}
    Según las Leyes de De Morgan, $\L_1 \cap \L_2 = \left(\L_1^c \cap \L_2^c\right)^c$. Como la clase de lenguajes regulares está cerrada por unión y complemento, este lenguaje también es regular.
\end{proof}

\subsection{Reverso}

\begin{theorem*}
    Sea $\L \subseteq \Sigma^*$ un lenguaje regular. Su reverso $\L^r$ es regular.
\end{theorem*}
\begin{proof}
    Tomemos un autómata finito determinístico $M = \langle Q, \Sigma, \delta, q_0, F \rangle$ tal que $\L(M) = \L$. Luego, construyamos el autómata no determinístico $M' = \langle Q', \Sigma, \delta', q_0', F' \rangle$, donde:
    \begin{itemize}
        \item El conjunto de estados es $Q' := Q \cup \{q_0'\}$, es decir, se agrega un nuevo estado $q_0'$.
        \item La función de transición está dada por:
              $$
                  \delta'(q, a) :=
                  \begin{cases}
                      \{r \mid \delta(r, a) = q\} & \si q \neq q_0'                \\
                      F                           & \si q = q_0' \land a = \lambda
                  \end{cases}
              $$
              Es decir, se \textbf{revierten} todas las transiciones de los estados originales, mientras que el nuevo estado está conectado mediante transiciones lambda con todos los estados finales anteriores.
        \item El estado inicial es el nuevo estado $q_0'$.
        \item El conjunto de estados finales es $F' = \{q_0\}$, es decir, tiene sólo al estado inicial original.
    \end{itemize}

    Luego, veamos que la relación de transición del nuevo autómata cumple $(q, w) \vdash_{M'}^* (r, \lambda) \iff (r, w^r) \vdash_M^* (q, \lambda)$ para los estados originales $q, r \in Q$. La demostración será por inducción en la longitud de $w$:
    \begin{itemize}
        \item \textbf{Caso Base}: Si $|w| = 0$, entonces $w = \lambda$. Como $M'$ no tiene ninguna transición lambda entre los estados originales, la única forma de que se cumpla $(q, \lambda) \vdash_{M'}^* (r, \lambda)$ es que $q = r$ y, por ende, también se cumple $(r, \lambda^r) = (r, \lambda) \vdash_M^* (q, \lambda)$. Lo mismo vale para la implicación inversa: el autómata original no tiene transiciones lambda porque es determinístico.
        \item \textbf{Paso Inductivo}: Si $|w| > 0$, entonces $w = a \cdot w'$. Por hipótesis inductiva, sabemos que $(q, w') \vdash_{M'}^* (r, \lambda) \iff (r, w'^r) \vdash_M^* (q, \lambda)$ para cualquier par de estados originales $q, r \in Q$. Demostremos que $(q, a \cdot w') \vdash_{M'}^* (r, \lambda) \iff (r, w'^r \cdot a) \vdash_M^* (q, \lambda)$.
              \begin{itemize}
                  \item $\implies$) Si $(q, a \cdot w') \vdash_{M'}^* (r, \lambda)$ entonces, por la \hyperref[subsubsec-propiedades-rel-transicion]{propiedad de la relación de transición} de siempre tomar algún estado, sabemos que existe algún estado $s \in Q$ tal que:
                        $$
                            (q, a) \vdash_{M'}^* (s, \lambda) \land (s, w') \vdash_{M'}^* (r, \lambda)
                        $$
                        Como el autómata no tiene transiciones lambda entre los estados originales, $(q, a) \vdash_{M'}^* (s, \lambda)$ sólo puede valer cuando $(q, a) \vdash_{M'} (s, \lambda)$, y esto a su vez implica que $s \in \delta'(q, a)$. Por la definición de $\delta'$, tenemos que $s \in \delta'(q, a) \implies \delta(s, a) = q \implies (s, a) \vdash_M (q, \lambda)$. Por otro lado, la hipótesis inductiva garantiza que $(s, w') \vdash_{M'}^* (r, \lambda)$ implica $(r, w'^r) \vdash_M^* (s, \lambda)$.

                        Por la propiedad de concatenación de la relación de transición, se tiene que:
                        $$
                            (r, w'^r) \vdash_M^* (s, \lambda) \land (s, a) \vdash_M^* (q, \lambda) \implies (r, w'^r \cdot a) \vdash_M^* (q, \lambda)
                        $$
                  \item $\impliedby$) La demostración es análoga: asumiendo $(r, w'^r \cdot a) \vdash_M^* (q, \lambda)$, podemos usar la propiedad de existencia de estados intermedios para obtener un estado $s \in Q$ tal que:
                        $$
                            (r, w'^r) \vdash_M^* (s, \lambda) \land (s, a) \vdash_M^* (q, \lambda)
                        $$
                        Nuevamente, sabemos que $(s, a) \vdash_M^* (q, \lambda) \implies (s, a) \vdash_M (q, \lambda) \implies \delta(s, a) = q$ y, por la definición de $\delta'$, esto implica que $s \in \delta'(q, a) \iff (q, a) \vdash_{M'} (s, \lambda)$. Por otro lado, la hipótesis inductiva implica que $(r, w'^r) \vdash_M^* (s, \lambda) \implies (s, w') \vdash_{M'} (r, \lambda)$.

                        Por la propiedad de concatenación de la relación de transición, se tiene que:
                        $$
                            (q, a) \vdash_{M'}^* (s, \lambda) \land (s, w') \vdash_{M'}^* (r, \lambda) \implies (q, a \cdot w') \vdash_{M'}^* (r, \lambda)
                        $$
              \end{itemize}
    \end{itemize}

    Ahora que sabemos que $(q, w) \vdash_{M'}^* (r, \lambda) \iff (r, w^r) \vdash_M^* (q, \lambda)$ para estados originales $q, r \in Q$, analicemos las cadenas aceptadas por $M'$:
    $$
        w \in \L(M') \iff (q_0', w) \vdash_{M'}^* (q_0, \lambda)
    $$
    El único estado final es $q_0$, así que la ejecución del autómata debe terminar en ese estado. Por otro lado, las únicas transiciones salientes de $q_0'$ son las transiciones lambda hacia los estados finales, así que se tiene:
    $$
        (q_0', w) \vdash_{M'}^* (q_0, \lambda) \iff \exists q_f \in F \mid (q_f, w) \vdash_{M'}^* (q_0, \lambda)
    $$
    Entonces, por la propiedad demostrada anteriormente, se tiene que:
    $$
    \begin{aligned}        
        w \in \L(M') & \iff \exists q_f \in F \mid (q_f, w) \vdash_{M'}^* (q_0, \lambda) \\
        & \iff \exists q_f \in F \mid (q_0, w^r) \vdash_{M}^* (q_f, \lambda) \iff w^r \in \L(M)
    \end{aligned}
    $$
    Es decir, $\L(M') = {w^r \mid w \in \L(M)} = (\L(M))^r = \L^r$, y por ende, $\L^r$ es regular.
\end{proof}


\section{Lema de Pumping}
\label{pumping-regulares}

El \textbf{lema de pumping} es una propiedad que cumplen todos los lenguajes regulares. Se puede enunciar de la siguiente manera:

\begin{theorem*}
    Sea $\L$ un lenguaje regular. Luego, existe una longitud $p \in \N$ tal que cualquier cadena $z$ con longitud $|z| \geq p$ se puede \textbf{factorizar} de la siguiente manera:
    $$
        z = u v w
    $$
    Donde las subcadenas cumplen las siguientes propiedades:
    $$
        \begin{aligned}
            |uv|    & \leq p                    \\
            |v|     & \geq 1                    \\
            u v^i w & \in \L \ \forall i \geq 0
        \end{aligned}
    $$
\end{theorem*}

En otras palabras, para cualquier lenguaje regular existe una longitud (denominada \textbf{constante de pumping}\footnote{Es importante notar que, bajo esta definición, la constante de pumping no es única: es fácil ver que, si $p$ es constante de pumping para un lenguaje $\L$, cualquier $m > p$ también lo es. Lo que sí es único para cada lenguaje es su \textbf{mínima} constante de pumping.}) a partir de la cual las cadenas se pueden ``bombear'', es decir, descomponer en prefijo $\cdot$ ``parte bombeable'' $\cdot$ sufijo, y cualquier cadena que resulte de repetir la parte bombeable también pertenecerá al lenguaje.

\subsection{Demostración}

\begin{proof}
    Para demostrar el lema de pumping, es convienente tomar un autómata finito determinístico $M = \langle Q, \Sigma, \delta, q_0, F \rangle$ que reconoce $\L$, es decir, que cumple $\L(M) = \L$. Demostraremos que una posible constante de pumping es precisamente la cantidad de estados del autómata $p = |Q|$.

    Sea $z \in \L(M)$ una cadena con longitud $m \geq p$. Recordemos que una forma de definir el lenguaje aceptado por $M$ es:
    $$
        z \in \L(M) \iff \exists q_f \in F, (q_0, z) \vdash^* (q_f, \lambda)
    $$

    Por ende, sea $q_f$ el estado final en el que termina $M$ al consumir $z$. Por otro lado, según la \hyperref[subsubsec-propiedades-rel-transicion]{propiedad de linealidad} de la relación de transición:
    $$
        |z| = m \implies (q_0, z) \vdash^m (q_f, \lambda)
    $$

    Es decir, el autómata debe pasar por $m$ transiciones hasta llegar al estado final, lo cual implica que pasa por $m + 1$ estados. Como hay sólo $p < m + 1$ estados, el autómata debe pasar 2 veces por algún estado (debido al Principio del Palomar). Más aún, consideremos el prefijo $\alpha$ de $z$ de longitud $p$ ($z = \alpha \cdot \beta$): su ejecución por el autómata también debe repetir algún estado, porque pasa por $p + 1$ estados.

    Tomemos la cadena de configuraciones instantáneas empezando en $(q_0, \alpha)$, con la notación $\alpha = a_0 \cdots a_p$:
    $$
        (q_0, a_0 \cdots a_p) \vdash (r_1, a_1 \cdots a_p) \vdash \cdots \vdash (r_i, a_i \cdots a_p) \vdash \cdots \vdash (r_i, a_j \cdots a_p) \vdash \cdots \vdash (r_k, \lambda)
    $$

    Como mencionamos, el autómata debe pasar 2 veces por algún estado: en este caso ese estado se denota $r_i$. Luego, llamemos:
    $$
        \begin{aligned}
            u & := a_0 \cdots a_{i - 1}       \\
            v & := a_i \cdots a_{j - 1}       \\
            w & := a_j \cdots a_p \cdot \beta
        \end{aligned}
    $$

    Se puede ver que $uvw = a_0 \cdots a_p \cdot \beta = \alpha \cdot \beta = z$. Ahora, demostremos las propiedades del lema de pumping:
    \begin{itemize}
        \item $|uv| = j \leq p$: Trivial, por construcción.
        \item $|v| \geq 1$: No se podría tener $|v| = 0$, porque la cadena de transiciones $(r_i, a_i \cdots a_p) \vdash \cdots \vdash (r_i, a_j \cdots a_p)$ no puede estar vacía.
        \item $u v^i w \in \L \ \forall i \geq 0$: Gracias a la propiedad de invarianza, se tiene que:
              $$
                  (r_i, v) \vdash (r_i, \lambda) \implies (r_i, v^m) \vdash (r_i, \lambda) \ \forall m \geq 0
              $$
              Por ende, para cualquier natural $m \geq 0$, se tiene:
              $$
                  (q_0, u \cdot v^m \cdot w) \vdash^* (r_i, v^m \cdot w) \vdash^* (r_i, w) \vdash^* (q_f, \lambda)
              $$
              Es decir, $u v^m w \in \L(M)$.
    \end{itemize}
\end{proof}

\subsection{Refutando Regularidad}

Una de las aplicaciones del Lema de Pumping es en demostrar que un lenguaje particular no puede ser regular: como todos los lenguajes regulares cumplen este lema, si un lenguaje no lo cumple entonces no es regular.

Tomemos como ejemplo el lenguaje $\L = \{a^n b^n \mid n \in \N\}$.

\begin{theorem*}
    El lenguaje $\L = \{a^n b^n \mid n \in \N\}$ no es regular.
\end{theorem*}
\begin{proof}
    Lo demostraremos por el absurdo: si $\L$ fuera regular, entonces existiría una constante de pumping $p \in \N$. Luego, tomemos la cadena $z = a^p b^p$. Es claro que:
    \begin{itemize}
        \item $z \in \L$
        \item $|z| = 2p \geq p$
    \end{itemize}

    Por ende, esta cadena se puede factorizar como $z = uvw$, con $|uv| = m \leq p$ y $|v| \geq 1$. Como $uv$ es prefijo de $z$ con longitud a lo sumo $p$, debe estar compuesta únicamente por el símbolo $a$ (porque $z$ empieza con $p$ $a$s). Esto a su vez implica que $v = a^k$ para algún $k \geq 1$. Además, sabemos que:
    $$uv^2w = a^{m-k}a^{2k}a^{p - m}b^p = a^{p + k}b^p \in \L$$

    Pero esto es absurdo, porque $p + k \geq p + 1 > p \implies p + k \neq p$.
\end{proof}



\chapter{Lenguajes Libres de Contexto}

\section{Autómatas de Pila}

Los \textbf{autómatas de pila} son similares a los autómatas finitos en que su ejecución se da mediante transiciones entre estados de un conjunto finito. La diferencia es que estos autómatas tienen acceso a una \textbf{pila}: las transiciones pueden apilar símbolos en ésta, y la transición tomada en cada paso puede depender del símbolo en el tope de la misma.

Formalmente, un autómata de pila $M$ es una tupla de la forma $\langle Q, \Sigma, \Gamma, \delta, q_0, Z_0, F \rangle$, donde:
\begin{itemize}
    \item $Q$ es el conjunto de estados.
    \item $\Sigma$ es el alfabeto de entrada.
    \item $\Gamma$ es el alfabeto de la pila.
    \item $\delta: Q \times (\Sigma \cup \{ \lambda \}) \times \Gamma \to \mathcal P (Q \times \Gamma^*)$ es la función de transición. Ésta recibe un estado, un símbolo de entrada (o $\lambda$) y un símbolo de tope de pila, y devuelve el conjunto de tuplas conformadas por los posibles estados siguientes junto con la cadena de símbolos que se apilan en cada caso.
    \item $q_0 \in Q$ es el estado inicial.
    \item $Z_0 \in \Gamma$ es el símbolo inicial de la pila.
    \item $F \subseteq Q$ es el conjunto de estados finales.
\end{itemize}

Al igual que los autómatas finitos, los autómatas de pila reciben una cadena $w \in \Sigma^*$ e iteran por sus símbolos. Se empieza en el estado inicial $q_0$ con una pila de $Z_0$, y se siguen las transiciones que indica $\delta$, modificando la pila según corresponda, hasta consumir toda la entrada. La aceptación se puede definir de 2 maneras:
\begin{enumerate}
    \item La cadena es aceptada cuando el último estado alcanzado está en $F$.
    \item La cadena es aceptada cuando, al terminar la ejecución, la pila está vacía.
\end{enumerate}

\subsection{Configuraciones Instantáneas}

En este caso, además de tener el estado actual y la cadena de entrada restante, las \textbf{configuraciones instantáneas} de estos autómatas deben contener los símbolos de la pila. Por ende, son tuplas $(q, \alpha, \pi) \in Q \times \Sigma^* \times \Gamma^*$. La \textbf{relación de transición} entre dichas configuraciones está dada por:
\begin{itemize}
    \item Si $(r, \tau) \in \delta(q, a, t)$, entonces $(q, a \alpha, t \pi) \vdash (r, \alpha, \tau \pi)$ (se consume $a$ de la entrada, se consume $t$ de la pila y se apila $\tau$).
    \item Si $(r, \tau) \in \delta(q, \lambda, t)$, entonces $(q, \alpha, t \pi) \vdash (r, \alpha, \tau \pi)$ (no se cambia la entrada, se consume $t$ de la pila y se apila $\tau$). 
\end{itemize}

Luego, si $M$ acepta por \textbf{estados finales}, su lenguaje aceptado es:
$$
    \L(M) := \{w \in \Sigma^* \mid \exists q_f \in F, \gamma \in \Gamma^* :  (q_0, w, Z_0) \vdash^* (q_f, \lambda, \gamma)\}
$$

Por otro lado, si $M$ acepta por \textbf{pila vacía}, su lenguaje aceptado es:
$$
    \L_\lambda(M) := \{w \in \Sigma^* \mid \exists r \in Q :  (q_0, w, Z_0) \vdash^* (r, \lambda, \lambda)\}
$$

\subsection{Equivalencia entre Tipos de Aceptación}

En el caso de los autómatas de pila generales, los 2 tipos de aceptación, por estados finales y por pila vacía, tienen el mismo poder expresivo.

\subsubsection{Estados Finales $\implies$ Pila Vacía}

\begin{theorem*}
    Dado un autómata de pila $M$, existe otro $M'$ tal que $\L(M) = \L_\lambda(M')$.
\end{theorem*}
\begin{proof}
    Sea $M = \langle Q, \Sigma, \Gamma, \delta, q_0, Z_0, F \rangle$. Luego, se puede construir el autómata $M' = \langle Q', \Sigma, \Gamma', \delta', q_0', X_0, \emptyset \rangle$, de la siguiente manera:
    \begin{itemize}
        \item $Q' = Q \cup \{q_0', q_\lambda\}$
        \item $\Gamma' = \Gamma \cup \{X_0\}$
        \item $\delta'$ tiene las todas transiciones de $\delta$, agregando:
        \begin{itemize}
            \item $\delta'(q_0', \lambda, X_0) = \{(q_0, Z_0 X_0)\}$ (se agrega un nuevo símbolo al fondo de la pila).
            \item $(q_\lambda, \lambda) \in \delta'(q_f, \lambda, t) \ \forall q_f \in F, t \in \Gamma$ (los estados finales se conectan a $q_\lambda$).
            \item $\delta'(q_\lambda, \lambda, t) = \{(q_\lambda, \lambda) \mid t \in \Gamma\}$ (el estado $q_\lambda$ vacía la pila).
        \end{itemize}
    \end{itemize}

    Veamos que este autómata acepta por pila vacía las mismas cadenas que el original acepta por estados finales. La idea es que el autómata apila un nuevo símbolo $X_0$ y \textbf{simula} al autómata original. Cuando se llega a un estado final, se pasa al estado $q_\lambda$, que tiene transiciones para vaciar la pila.

    \begin{itemize}
        \item $\L(M) \subseteq \L_\lambda(M')$: Supongamos que $w \in \L(M)$. Esto quiere decir que existe algún estado final $q_f$ y cadena de pila $\gamma$ tal que:
        $$
            (q_0, w, Z_0) \vdash_M^* (q_f, \lambda, \gamma)
        $$
        Además, como $\delta'(q_0', \alpha, X_0) \vdash_{M'} (q_0, \alpha, Z_0 X_0)$ para cualquier cadena $\alpha \in \Sigma^*$, y todas las transiciones de $\delta$ están incluidas en $\delta'$, $M'$ \textbf{simula} a $M$, es decir:
        $$
            (q_0', w, X_0) \vdash_{M'} (q_0, w, Z_0 X_0) \vdash_{M'}^* (q_f, \lambda, \gamma X_0)
        $$

        Una vez que el autómata llega a un estado final $q_f$, puede tomar la transición hacia $q_\lambda$, y luego vaciar el resto de la pila (denotamos $\gamma = t_1 t_2 \cdots t_k$):
        $$
            (q_f, \lambda, t_1 t_2 \cdots t_k X_0) \vdash_{M'} (q_\lambda, \lambda, t_2 \cdots t_k X_0)
        $$

        Luego, las transiciones de $q_\lambda$ permiten consumir todos los símbolos de la pila hasta vaciarla:
        $$
            (q_\lambda, \lambda, t_2 \cdots t_k X_0) \vdash_{M'} (q_\lambda, \lambda, t_3 \cdots t_k X_0) \vdash_{M'} \cdots \vdash_{M'} (q_\lambda, \lambda, X_0) \vdash_{M'} (q_\lambda, \lambda, \lambda)
        $$

        Juntando todas las cadenas de transiciones, se tiene:
        $$
            (q_0', w, X_0) \vdash_{M'}^* (q_\lambda, \lambda, \lambda) \implies w \in \L_\lambda(M')
        $$

        \item $\L_\lambda(M') \subseteq \L(M)$: Si $w \in \L_\lambda(M')$, es porque:
        $$
            (q_0', w, X_0) \vdash_{M'}^* (q_\lambda, \lambda, \lambda)
        $$

        Esto es porque la única transición que puede desapilar el símbolo inicial $X_0$ es la que está en $q_\lambda$. Más aún, estas transiciones deben ser de la forma:
        $$
            (q_0', w, X_0) \vdash_{M'} \underbrace{(q_0, w, Z_0 X_0) \vdash_{M'}^* (q_f, w, \gamma X_0)}_{\text{Simulación de } M} \vdash_{M'}^* (q_\lambda, \lambda, \lambda)
        $$

        Las transiciones que simulan a $M$ también están presentes en $M$, así que se tiene:
        $$
            (q_0, w, Z_0) \vdash_M^* (q_f, w, \gamma) \implies w \in \L(M)
        $$
    \end{itemize}
\end{proof}

\subsubsection{Pila Vacía $\implies$ Estados Finales}

\begin{theorem*}
    Dado un autómata de pila $M$, existe otro $M'$ tal que $\L(M') = L_\lambda(M)$.
\end{theorem*}
\begin{proof}
    Sea $M = \langle Q, \Sigma, \Gamma, \delta, q_0, Z_0, \emptyset \rangle$. Se define $M' = \langle Q', \Sigma, \Gamma', \delta', q_0', X_0, F \rangle$, donde:
    \begin{itemize}
        \item $Q' = Q \cup \{q_0', q_f\}$
        \item $\Gamma' = \Gamma \cup \{X_0\}$
        \item $\delta'$ tiene todas las transiciones de $\delta$, agregando:
        \begin{itemize}
            \item $\delta'(q_0', \lambda, X_0) = \{(q_0, Z_0 X_0)\}$ (se agrega un nuevo símbolo al fondo de la pila).
            \item $(q_f, \lambda) \in \delta'(q, \lambda, X_0)$ (todos los estados pueden pasar al estado final consumiendo el símbolo nuevo).
        \end{itemize}
        \item $F = \{q_f\}$
    \end{itemize}

    En este caso, el autómata nuevo también simula al original. Veamos que aceptan los mismos lenguajes:
    \begin{itemize}
        \item $\L_\lambda(M) \subseteq \L(M')$: Si $w \in \L_\lambda(M)$, entonces se tiene $(q_0, w, Z_0) \vdash_M (q, \lambda, \lambda)$ para algún $q \in Q$. Luego, por la definición de $\delta'$, también vale:
        $$
            (q_0', w, X_0) \vdash_{M'} \underbrace{(q_0, w, Z_0 X_0) \vdash_{M'}^* (q, \lambda, \lambda)}_{\text{Simulación de } M} \vdash_{M'} (q_f, \lambda, \lambda)
        $$

        Es decir,
        $$
            (q_0', w, X_0) \vdash_{M'}^* (q_f, \lambda, \lambda) \implies w \in \L(M')
        $$
        \item $\L(M') \subseteq \L_\lambda(M)$: Para que una cadena $w$ sea aceptada por $M'$, la ejecución debe comenzar por $q_0'$, pasar por transiciones de $M$ hasta consumir toda la cadena de entrada y llegar al símbolo $X_0$, para después pasar al nuevo estado final $q_f$. Es decir, la cadena de transiciones debe ser de la forma:
        $$
            (q_0', w, X_0) \vdash_{M'} \underbrace{(q_0, w, Z_0 X_0) \vdash_{M'}^* (q, \lambda, \lambda)}_{\text{Simulación de } M} \vdash_{M'} (q_f, \lambda, \lambda)
        $$

        Entonces, como las transiciones simuladas también están presentes en el autómata original, se tiene:
        $$
            (q_0, w, Z_0 X_0) \vdash_M^* (q, \lambda, \lambda) \implies w \in \L_\lambda(M)
        $$
    \end{itemize}
\end{proof}

\textbf{TODO}


\section{Gramáticas Libres de Contexto}

Como se mencionó anteriormente, una \textbf{gramática libre de contexto} $G = \langle V_N, V_T, P, S \rangle$ en la cual todas las producciones son de la forma:
$$
    A \to \alpha, \text{ con } A \in V_N, \alpha \in (V_N \cup V_T)^*
$$

La relación de derivación $\deriv: (V_N \cup V_T)^* \times (V_N \cup V_T)^*$ de una gramática $G$ está definida en base a las transiciones: si $\alpha, \beta, \gamma_1, \gamma_2 \in (V_N \cup V_T)^*$ con $\alpha \to \beta \in P$, se tiene
$$
    \gamma_1 \alpha \gamma_2 \deriv \gamma_1 \beta \gamma_2
$$

Una \textbf{forma sentencial} de $G$ es una cadena $\alpha \in (V_N \cup V_T)^*$ tal que $S \derivs \alpha$. Luego, el \textbf{lenguaje aceptado} por $G$ se define como:
$$
    \L(G) := \{\alpha \in V_T^* \mid S \derivs \alpha\}
$$

\subsection{Equivalencia con Autómatas de Pila}

Veamos que las gramáticas libres de contexto tienen el mismo poder expresivo que los autómatas de pila.

\subsubsection{GLC $\implies$ Autómata de Pila}

\begin{theorem*}
    Para cada gramática libre de contexto $G = \langle V_N, V_T, P, S \rangle$, existe un autómata de pila $M$ tal que $\L(G) = \L_\lambda(M)$.
\end{theorem*}
\begin{proof}
    Tomemos el autómata $M = \langle Q, \Sigma, \Gamma, \delta, q_0, Z_0 \rangle$, donde:
    \begin{itemize}
        \item $Q = \{q\}$
        \item $\Sigma = V_T$
        \item $\Gamma = (V_T \cup V_N)$
        \item $\delta$ está dada por las siguientes reglas:
        \begin{itemize}
            \item $\forall (A \to \alpha) \in P, \ (q, \alpha) \in \delta(q, \lambda, A)$
            \item $\forall a \in V_T,\ \delta(q, a, a) = \{(q, \lambda)\}$
        \end{itemize}
        \item $q_0 = q$
        \item $Z_0 = S$
    \end{itemize}

    Como resultado intermedio, veamos que $A \derivs w \iff (q, w, A) \overset{*}{\vdash} (q, \lambda, \lambda)$, por inducción completa en la cantidad de derivaciones $k$:
    \begin{itemize}
        \item \textbf{Caso Base}: en $k = 1$, denotando w = $a_1 \cdots a_k$, $A \deriv w$ si y sólo si $A \to w \in P$, y esto a su vez se da si y sólo si $(q, a_1 \cdots a_k, A) \vdash (q, a_1 \cdots a_k, a_1 \cdots a_k) \overset{k}{\vdash} (q, \lambda, \lambda)$.
        \item \textbf{Paso Inductivo}: en $k > 1$, nuestra hipótesis inductiva será que la propiedad vale para todo $j < k$.
        
            Luego, $A \overset{k}{\deriv} w \iff A \to X_1 \cdots X_n \in P$, con $X_i \overset{k_i}{\deriv} x_i$ tal que $w = x_1 \cdots x_n$ y $\sum_{i = 1}^n k_i = k$. Además, alguno de los $k_i$ es al menos $1$ (porque $k > 1$), así que $k_i < k\ \forall i$.

            Por la definición de $M$, $A \to X_1 \cdots X_n \in P \iff (q, w, A) \vdash (q, w, X_1 \cdots X_n)$. Consideremos cada símbolo $X_i$:
            \begin{itemize}
                \item Si $X_i \in V_N$, entonces la hipótesis inductiva indica que $X_i \to x_i \in P \iff (q, x_i, X_i) \overset{*}{\vdash} (q, \lambda, \lambda)$.
                \item Si $X_i = x_i \in V_T^*$, entonces $(q, x_i, X_i) = (q, x_i, x_i) \overset{*}{\vdash} (q, \lambda, \lambda)$.
            \end{itemize}

            Por ende, se tiene que:
            $$
            \begin{aligned}
                A \to X_1 \cdots X_n \in P & \iff (q, w, A) \vdash (q, x_1 \cdots x_n, X_1 \cdots X_n) \\
                & \overset{*}{\vdash} (q, x_2 \cdots x_n, X_2 \cdots X_n) \\
                & \qquad \vdots \\
                & \overset{*}{\vdash} (q, \lambda, \lambda)
            \end{aligned}
            $$
    \end{itemize}

    En particular, se tiene que $w \in \L(G) \iff S \derivp w \iff (q, w, S) \overset{*}{\vdash} (q, \lambda, \lambda) \iff w \in \L(M)$.
\end{proof}

\subsubsection{Autómata de Pila $\implies$ GLC}

\textbf{TODO}


\section{Lema de Pumping Libre de Contexto}

Existe una versión análoga al \hyperref[pumping-regulares]{lema de pumping de los lenguajes regulares}, esta vez para los lenguajes libres de contexto:

\begin{theorem*}
    Para cualquier lenguaje $L \subseteq \Sigma^*$ libre de contexto, existe una constante de pumping $p > 0$ tal que, para toda cadena $\alpha \in \Sigma^*$ con $|\alpha| \geq p$, se tiene una factorización $\alpha = r x y z s$ tal que:
    \begin{itemize}
        \item $|xyz| \leq n$
        \item $|xz| \geq 1$
        \item $\forall i \geq 0,\ r x^i y z^i s \in L$
    \end{itemize}
\end{theorem*}

Demostrar esta versión del lema de pumping es más complicado que la de los lenguajes regulares, y requiere entender el concepto de \textbf{árbol de derivación}.

\subsection{Árboles de Derivación}

Dada una gramática libre de contexto $G = \langle V_N, V_T, P, S \rangle$ y una cadena $\alpha \in V_T^*$, un \textbf{árbol de derivación} es un árbol tal que:
\begin{itemize}
    \item Cada vértice está etiquetado con un elemento de $V_N \cup V_T \cup \lambda$.
    \item La raíz tiene como etiqueta a $S$.
    \item Los vértices interiores están etiquetados con un símbolo de $V_N$.
    \item Para todo vértice interior con etiqueta $A$ e hijos con etiquetas $X_1, \dots, X_k$, la gramática debe tener una producción $A \to X_1 \cdots X_k \in P$.
    \item Las hojas están etiquetadas con símbolos de $V_T \cup \lambda$.
    \item Los vértices con etiqueta $\lambda$ son el único hijo de un padre con etiqueta $A$, y por la regla anterior se debe tener $A \to \lambda$.
    \item Un recorrido in-order por las hojas $h_1, \dots, h_m$ del árbol debe res ultar en la cadena $h_1 \cdots h_m = \alpha$.
\end{itemize}

Por ejemplo, tomemos la gramática $G = \langle \{\{E, T, F\}, \{\mathop{+}, \mathop{*}, \textbf{id}, \textbf{const}, \mathrm{(}, \mathrm{)}\}, P, E\}$ con $P$ dado por:
$$
    \begin{aligned}
        E & \to E + T \mid T                             \\
        T & \to T * F \mid F                             \\
        F & \to \textbf{id} \mid \textbf{const} \mid (E)
    \end{aligned}
$$

Un árbol de derivación para la cadena $\alpha = \textbf{id} + \textbf{id} * \textbf{id}$ sería (en este caso, es único):
\begin{figure}[H]
    \centering
    \includegraphics*[width=0.4\textwidth]{árbol-de-derivación.png}
    \caption*{Árbol de derivación para $\textbf{id} + \textbf{id} * \textbf{id}$.}
\end{figure}

Un \textbf{camino} de $X$ en un árbol con raíz en $A$, denotado $\mathcal T(A)$ es una secuencia de símbolos $A, X_1, \dots, X_k, X$ que corresponde a las etiquetas de una rama de $\mathcal T(A)$. Luego la \textbf{altura} de $\mathcal T(A)$ está definida como:
$$
    \max \{|\alpha| \mid \exists x \in V_T : \alpha x \text{ es camino de } \mathcal T(A)\}
$$

\subsubsection{Cota Basada en Altura de Árbol de Derivación}

El siguiente lema establece una relación entre la longitud de una cadena y la altura máxima de un árbol de derivación para la misma:

\begin{lemma*}
    Sea $G = \langle V_N, V_T, P, S \rangle$ una gramática libre de contexto, y $\alpha \in (V_N \cup V_T)^*$ tal que $\alpha \in \L(G)$. Para todo árbol de derivación $\mathcal T(S)$ que genera a $\alpha$ con altura $h$, sea:
    $$
        a = \max\{k \mid (\exists A \to \beta \in P : \beta \neq \lambda \land k = |\beta|) \lor (\exists A \to \lambda \in P \land k = 1)\}
    $$

    Luego, $|\alpha| \leq a^h$.
\end{lemma*}
\begin{proof}
    La constante $a$ es básicamente la cantidad máxima de símbolos que puede tener el cuerpo de una producción de $G$. Por la construcción de los árboles de derivación, este es el grado máximo que puede tener cada nodo interior. Como un árbol $a$-ario de altura $h$ completo tiene $a^h$ hojas, esta es la longitud máxima que puede tener una cadena generada por éste.
\end{proof}

\subsection{Demostración}

\begin{proof}
    Sea $G = \langle V_N, V_T, P, S \rangle$ una gramática libre de contexto. Definamos $a$ como la cantidad máxima de nodos en el cuerpo de cada producción de $G$:
    $$
        a := \max\ \{2\} \cup \{|\beta| \mid A \to \beta \in P\}
    $$

    Tomemos como constante de pumping al valor $p = a^{|V_N| + 1}$. Luego, sea $\alpha \in \L(G)$ una cadena con longitud $|\alpha| \geq p$ y $\mathcal T(S)$ un árbol de derivación para la misma mínimo en cantidad de nodos.
    
    Si el árbol $\mathcal T(S)$ tiene altura $h$ entonces, por el lema anterior, $a^h \geq |\alpha|$. Además, como $|\alpha| \geq p = a^{|V_N| + 1}$, se tiene que $a^h \geq a^{|V_N| + 1} \iff h \geq |V_N| + 1$.
    
    Entonces, por la definición de altura, hay algún camino de terminales $S, X_1, \dots, X_h$ dentro del árbol. Como $h > |V_N|$, por el principio del palomar, debe haber algún símbolo repetido en ese camino. Sea $A$ el primer símbolo que se repite recorriendo el camino de atrás para adelante.

    Consideremos la primera y la segunda aparición de $A$ al recorrer el árbol de abajo hacia arriba:
    \begin{figure}[H]
        \centering
        % TODO: Hacer una mejor imagen
        \includegraphics*[width=0.4\textwidth]{árbol-de-derivación-pumping.png}
        \caption*{Árbol de derivación con el símbolo repetido $A$.}
    \end{figure}

    Como se puede ver en la ilustración, la cadena $\alpha$ se descompone como $\alpha = r x y z s$, donde:
    \begin{itemize}
        \item $r$ es la cadena generada por los símbolos anteriores a la segunda $A$ en el árbol.
        \item $x$ es la cadena generada por el segundo subárbol enraizado en $A$ con símbolos antes de la segunda $A$.
        \item $y$ es la cadena generada por el primer subárbol enraizado en $A$.
        \item $z$ es la cadena generada por el segundo subárbol enraizado en $A$ con símbolos después de la segunda $A$.
        \item $s$ es la cadena generada por los símbolos posteriores a la segunda $A$ en el árbol.
    \end{itemize}

    Como $A$ es el primer símbolo que se repite, todas las ramas del subárbol $\mathcal T(A)$ tienen a lo sumo $|V_N + 1|$ vértices (si no, se repetirían más etiquetas), así que tiene altura $|V_N| + 1$ y por ende:
    $$
        |xyz| \leq a^{|V_N| + 1} = p
    $$

    Además, $x$ y $z$ no pueden ser ambos nulos, porque si no el árbol de derivación para $\alpha$ no sería mínimo en cantidad de nodos: se podría eliminar el árbol entre ambas $A$ y usar sólo la producción de la primera. Esto implica que:
    $$
        |xz| \geq 1
    $$

    Como se puede ver en árbol de derivación, se tiene:
    $$
    \begin{aligned}
        S & \derivs rAs \\
        A & \derivs xAz \\
        A & \derivs y
    \end{aligned}
    $$

    Estas producciones implican que $S \derivs r x^i y z^i s$ para cualquier $i \geq 0$. Por ende, tenemos que:
    $$
        \forall i \geq 0,\ r x^i y z^i s \in \L(G)
    $$
\end{proof}


\section{Cota en la Cantidad de Derivaciones}

En esta sección se establecerá una cota en la cantidad de derivaciones que puede tomar una gramática en llegar a una cadena dada.

\subsection{Derivación más a la Izquierda/Derecha}

Dada una gramática libre de contexto $G = \langle V_N, V_T, P, S \rangle$ y una cadena $w \in \L(G)$ una \textbf{derivación más a la izquierda} de $w$ es una tal que en cada paso con forma sentencial $X_1 \cdots X_k$ se aplica una derivación sobre el primer $X_i$ no-terminal. Análogamente, una derivación más a la derecha es una en la cual siempre se aplica una derivación sobre el último $X_i$ no-terminal.

Por otro lado, un \textbf{árbol de derivación más a la izquierda/derecha} es un árbol de derivación que corresponde a una derivación más a la izquierda/derecha.

\subsection{Recursiva a Izquierda/Derecha}

Una gramática es \textbf{recursiva a izquierda} cuando hay algún símbolo $A \in V_N$ y cadena $\alpha \in (V_T \cup V_N)^*$ tal que $A \overset{+}{\derivL} A \alpha$. Análogamente, una gramática es \textbf{recursiva a derecha} cuando se puede derivar $A \overset{+}{\derivR} \alpha A$.

\subsection{Demostración de la Cota}

Podemos enunciar y demostrar la cota en la cantidad de derivaciones:

\begin{theorem*}
    Sea $G = \langle V_N, V_T, P, S \rangle$ libre de contexto y no recursiva a izquierda. Luego, existe alguna constante $c$ tal que si $A \overset{i}{\derivL} w B \alpha$ con $w \in V_T^*, |w| = n$ entonces $i \leq c^{n + 2}$.
\end{theorem*}
\begin{proof}
    Denotamos $k = |V_N|$, y $\mathcal T(A)$ al árbol de derivación más a la izquierda correspondiente a $A \overset{i}{\derivL} w B \alpha$.
    \begin{figure}[H]
        \centering
        \includegraphics*[width=0.4\textwidth]{árbol-de-derivación-más-a-la-izquierda.png}
        \caption*{Árbol de derivación para $A \overset{i}{\derivL} w B \alpha$.}
    \end{figure}

    Sea $n_0$ el nodo con la etiqueta $B$ en el árbol. Luego, todas las ramas a la derecha de la de este nodo deben ser más cortas: si no, se habría aplicado alguna derivación a un símbolo posterior a $B$ antes que a éste, lo cual está prohibido por ser ésta una derivación más a la izquierda.

    Supongamos que hay alguna rama de longitud mayor a $k (n + 2)$ arcos. Luego, como todas las ramas a la derecha de $n_0$ tienen longitud menor a ésta, debe haber alguna rama con longitud mayor a $k (n + 2)$ que llega a $n_0$ o algún nodo a su izquierda. Además, por su ubicación, las hojas de esta rama deben formar alguna cadena $x$ contenida en $wB$.

    Consideremos segmentos de longitud $k$ en esta rama $S_1, \dots, S_{n + 2}$. Cada uno de ellos contiene nodos que derivan en hojas, pero cada hoja tiene un único ancestro más bajo en la rama\footnote{Éste es el ancestro que se ubica en el nivel más bajo del árbol, dentro de los que están en la rama. Por ejemplo, todos los nodos tienen como ancestro a la raíz, pero es posible que algunos sean descendientes de nodos que están más abajo (pero dentro de la rama).}. Sea $f: \{S_1, \dots S_{n + 2}\} \to \mathcal P(\textsc{Símbolos}(x))$ la función definida como:
    $$
        f(S_i) := \{a \in \textsc{Símbolos(x)} \mid \exists n \in S_i : n \text{ es el ancestro más bajo de } a\}
    $$

    Como la cadena $x$ está contenida en $wB$, tiene longitud de a lo sumo $|wB| = n + 1$. Al haber $n + 2$ segmentos, debe haber algún $S_i = n_1, \dots, n_k$ tal que $f(S_i) = \emptyset$ (por el principio del palomar).  
    
    Para cada $n_j$ de $S_i$, todos sus descendientes directos anteriores a $n_{j + 1}$ deben ser $\lambda$ o símbolos anulables, porque si no $n_j$ sería el antecesor más bajo de algún símbolo de $x$. Además, como hay $k$ símbolos no terminales, debe haber algún par de nodos con la misma etiqueta $X$. Es decir, la gramática incluye la derivación $X \derivp Y_1 \cdots Y_m X \gamma$ con $Y_1, \dots, Y_m$ anulables ($Y_j \derivs \lambda$), así que:
    $$
        X \derivp Y_1 \cdots Y_m X \gamma \derivp X \gamma
    $$

    Esto es absurdo, porque $G$ no es recursiva a izquierda.

    Sabiendo que todas las ramas del árbol longitud de a lo sumo $k (n + 2)$, esta es la altura máxima del árbol $\mathcal T(A)$. Llamemos $\ell$ a la longitud máxima del cuerpo de una producción:
    $$
        \ell = \max \{ |\beta| \mid A \to \beta \in P \}
    $$

    Luego, como un árbol $\ell$-ario de altura $k (n + 2)$ tiene a lo sumo $\ell^{k (n + 2)}$. Por ende, si $A \overset{i}{\derivL} w B \alpha$, entonces $i \leq \ell^{k (n + 2)}$. Basta con tomar $c = \ell^k$.
\end{proof}


\section{Propiedades de Clausura de Lenguajes Libres de Contexto}

\subsection{Unión}

\begin{theorem*}
    Sean $\L_1, \L_2 \subseteq V_T^*$ lenguajes libres de contexto. Su unión $\L_1 \cup \L_2$ es libre de contexto.
\end{theorem*}
\begin{proof}
    Sean $G_1 = \langle V_T, V_{N_1}, P_1, S_1 \rangle$ y $G_2 = \langle V_T, V_{N_2}, P_2, S_2 \rangle$ gramáticas libres de contexto para los lenguajes $\L_1$ y $\L_2$. Luego, se puede tomar la gramática $G = \langle V_T, \{S\} \cup V_{N_1} \cup V_{N_2}, P_1 \cup P_2 \cup \{S \to S_1, S \to S_2\}, S \rangle$. Veamos que $w \in \L(G) \iff w \in \L_1 \lor w \in \L_2$:
    \begin{itemize}
        \item $\implies$) Si $w \in \L(G)$, entonces debe existir alguna derivación $S \underset{G}{\derivp} w$. Como las únicas 2 derivaciones con cabeza $S$ son $S \to S_1$ y $S \to S_2$, se debe dar $S_1 \underset{G}{\derivp} w \lor S_2 \underset{G}{\derivp} w$. Además, como $S_1$ solo participa en derivaciones de $P_1$, y lo mismo sucede para $S_2$ esto quiere decir que $S_1 \underset{G_1}{\derivp} w$ o $S_2 \underset{G_2}{\derivp} w$. Por ende, $w \in \L_1 \lor w \in \L_2$.
        \item $\impliedby$) Si $w \in \L_1 \lor w \in L_2$, se tiene que $S_1 \underset{G_1}{\derivp} w$ o $S_2 \underset{G_2}{\derivp} w$. Como las producciones de ambas gramáticas están contenidas en las producciones de $G$, también se tiene que $S_1 \underset{G}{\derivp} w \lor S_2 \underset{G}{\derivp} w$. Gracias a las producciones $S \to S_1$ y $S \mid S_2$, en ambos casos se tiene $S \underset{G}{\derivp} w$.
    \end{itemize}
\end{proof}

\subsection{Concatenación}

\begin{theorem*}
    Sean $\L_1, \L_2 \subseteq V_T^*$ lenguajes libres de contexto. Su concatenación $\L_1 \cdot \L_2$ es libre de contexto.
\end{theorem*}
\begin{proof}
    La demostración es análoga a la de la unión: tomamos las gramáticas $G_1 = \langle V_T, V_{N_1}, P_1, S_1 \rangle$ y $G_2 = \langle V_T, V_{N_2}, P_2, S_2 \rangle$ para los lenguajes $\L_1$ y $\L_2$, y construimos una nueva gramática $G = \langle V_T, \{S\} \cup V_{N_1} \cup V_{N_2}, P_1 \cup P_2 \cup \{S \to S_1 S_2\}, S \rangle$, que es libre de contexto y cumple que $w \in \L(G) \iff w \in \L_1 \cdot \L_2$.
\end{proof}

\subsection{Clausura de Kleene}

\begin{theorem*}
    Sea $\L \subseteq V_T^*$ un lenguaje libre de contexto. Su clausura de Kleene $\L^*$ es libre de contexto.
\end{theorem*}
\begin{proof}
    Al igual que en los teoremas anteriores, se demuestra tomando una gramática libre de contexto $G = \langle V_T, V_N, P, S \rangle$ para $\L$ y construyendo una de la forma $G' = \langle V_T, \{S'\} \cup V_N, P \cup \{S' \to S S', S' \to \lambda\}, S' \rangle$. Esta nueva gramática cumple que $w \in \L(G') \iff w \in \L^*$.
\end{proof}

\subsection{Reverso}

\begin{theorem*}
    Sea $\L \subseteq V_T^*$ un lenguaje libre de contexto. Su reverso $\L^*$ es libre de contexto.
\end{theorem*}
\begin{proof}
    Dada una gramática libre de contexto $G = \langle V_T, V_N, P, S \rangle$ para $\L$, se puede tomar la gramática $G' = \langle V_T, V_N, P^R, S \rangle$, donde $P^R = \{A \to \beta^R \mid A \to \beta \in P\}$.
\end{proof}

\subsection{Intersección con Lenguaje Regular}

\begin{theorem*}
    Sea $\L \subseteq \Sigma^*$ libre de contexto y $R \subseteq \Sigma^*$ regular. Luego, la intersección $\L \cap R$ es libre de contexto
\end{theorem*}
\begin{proof}
    Sea $M_L = \langle Q_L, \Sigma, \Gamma, \delta_L, q_{L_0}, Z_0, F_L \rangle$ un autómata de pila para $\L$, y sea $M_R = \langle Q_R, \Sigma, \delta_R, q_{R_0}, F_R \rangle$ un AFND para $R$. Luego, se puede tomar el autómata de pila definido por $M' = \langle Q', \Sigma, \Gamma, \delta', q_0', Z_0, F' \rangle$, donde:
    \begin{itemize}
        \item $Q' = Q_L \times Q_R$
        \item $\delta'((q_L, q_R), a, t) = \{((p_L, p_R), s) \mid (p_L, s) \in \delta_L(q_L, a, t) \land p_R \in \delta_R(q_R, a)\}$
        \item $F' = \{(q_L, q_R) \mid q_L \in F_L \land q_R \in F_R\}$
    \end{itemize}

    Luego, $w \in \L(M') \iff w \in \L \cap R$, así que $\L \cap R$ es libre de contexto.
\end{proof}

\subsection{No-clausura}

Se puede verificar que los lenguajes libres de contexto \textbf{no} están clausurados por ciertas operaciones.

\subsubsection{Intersección}

Para refutar la clausura por intersección, se pueden considerar los lenguajes $\L_1 = \{a^i b^j c^j \mid a,b,c \in \Sigma^*, i,j \geq 0\}$ y $\L_2 = \{a^i b^i c^j \mid a,b,c \in \Sigma^*, i,j \geq 0\}$. Ambos son libres de contexto: son concatenaciones entre lenguajes de la forma $\{x^i y^i\}$ (que son libres de contexto) y lenguajes $z^*$ (que son regulares, así que también son libres de contexto).

Por otro lado, su intersección $\L_1 \cap \L_2 = \{a^i b^i c^i \mid a,b,c \in \Sigma^*, i \geq 0\}$ no puede ser libre de contexto, debido al lema de pumping: tomando la cadena $w = a^p b^p c^p$, cualquier descomposición $w = rxyzs$ tendrá a $xyz$ como una cadena que contiene sólo $a$s y $b$s o sólo $b$s y $c$s (porque tiene longitud máxima $p$). Luego, al bombearla, se agregarían sólo 2 de los 3 tipos de símbolos, así que la nueva cadena no estaría en $\L_1 \cap \L_2$ (así que el lenguaje no es libre de contexto).

\subsubsection{Complemento}

Es fácil ver que los lenguajes libres de contexto no están clausurados por complemento: si lo estuvieran, entonces para dos lenguajes LCs $\L_1, \L_2$ se tendría que $\left(\L_1^c \cup \L_2^c\right)^c = \L_1 \cap \L_2$ es libre de contexto (recién vimos que esto es falso).

\subsubsection{Diferencia}

De nuevo, se puede refutar en base a propiedades anteriores: si los LLCs estuvieran cerrados por diferencia, para cualquier LLC $\L$, el lenguaje $\Sigma^* \setminus \L = \L^c$ también sería LC (absurdo, por la propiedad anterior).


\input{chapters/lenguajes-libres-de-contexto/algoritmos-de-decisión.tex}

\section{Autómatas de Pila Determinísticos}

Un autómata de pila $M = \langle Q, \Sigma, \Gamma, \delta, q_0, Z_0, F \rangle$ se considera \textbf{determinístico} cuando, para cada $a \in \Sigma, q \in Q, Z \in \Gamma$, se cumple alguna de:
\begin{itemize}
    \item $|\delta(q, a, Z)| \leq 1$, y $|\delta(q, \lambda, Z)| = 0$.
    \item $|\delta(q, a, Z)| = 0$ para todo $a$, y $|\delta(q, \lambda, Z)| \leq 1$.
\end{itemize}

Es decir, cada par (estado, símbolo de pila) puede o bien tener transiciones en algunos símbolos de entrada, o no tener ninguna de esas transiciones, teniendo en cambio alguna transición-$\lambda$.

\hyperref[equivalencia-afd-afnd]{A diferencia de los autómatas finitos}, los APDs no tienen el mismo poder expresivo que los autómatas de pila generales. Por ejemplo, el lenguaje $\mathcal L = \{ww^R \mid w \in \Sigma^*\}$ es libre de contexto\footnote{Se puede construir un autómata de pila no-determinístico que lo reconoce: este autómata tendría un alfabeto de pila $\Gamma$ que representa símbolos de $\Sigma$. Luego, debería apilar los símbolos correspondientes a los de $w$ a medida que lee la cadena, y transicionar no-determinísticamente a desapilarlos, validando que es equivalente al que se está leyendo de la entrada.}, pero no puede ser aceptado por un autómata de pila determinístico. Esto se puede ver considerando la cadena $0^n 11 0^n$: para que sea aceptada por un APD, éste debe usar la pila para validar que hay tantos $0$s antes del $11$ como después. Sin embargo, si luego recibiera la entrada $0^m 11 0^m$, ya no podría verificar si $n = m$, porque la pila se vació en el paso anterior.

\subsection{Clausura por Complemento}

A diferencia de los lenguajes libres de contexto generales, los LLC determinísticos están clausurados por complemento:

\begin{theorem*}
    Sea $\L$ un lenguaje libre de contexto determinístico. Luego, $\L^c$ también es libre de contexto determinístico.
\end{theorem*}

Sea $M = \langle Q, \Sigma, \Gamma, \delta, q_0, Z_0, F\rangle$ un APD tal que $\L(M) = \L$. En este caso, no basta con complementar el conjunto de estados finales del autómata, ya que podría pasar que:
\begin{enumerate}
    \item $M$ sea incompleto, y por ende no acepta una cadena por toma alguna transición vacía. En ese caso, el autómata con estados finales complementados también rechazaría esa cadena.
    \item $M$ entre en un ciclo infinito de transiciones-$\lambda$ para alguna cadena. En este caso, la cadena tampoco se acepta en el autómata con estados finales complementados, porque nunca se llega a consumir por completo.
    \item $M$ consume toda la cadena de entrada, pero llega a un punto en el que se pueden alcanzar estados finales y no-finales por medio de transiciones-$\lambda$. En este caso, el autómata con estados finales complementados también aceptaría la cadena.
\end{enumerate}

Debemos mitigar todos estos problemas para encontrar un autómata que acepte exactamente el lenguaje complementado $\L^c$. 

\subsubsection{Configuraciones Ciclantes}

Una configuración $(q, w, \alpha)$ \textbf{cicla} cuando existes secuencias infinitas de estados $p_1, p_2, \dots$ y cadenas $\beta_1, \beta_2, ...$ con $|\beta_i| \geq \alpha$\footnote{Si se tiene algún $\beta_j$ con longitud menor, se podría tomar como el nuevo principio de la cadena de configuraciones ciclantes. Como la longitud está acotada por debajo (por $0$), hay alguna $\beta_m$ en la cadena con longitud mínima.} tales que:
$$
    (q, w, \alpha) \vdash (p_1, w, \beta_1) \vdash (p_2, w, \beta_2) \vdash \cdots \vdash (p_i, w, \beta_i) \vdash \cdots
$$

Es decir, es una secuencia de configuraciones en la cual nunca se leen símbolos de entrada, y la pila mantiene su longitud inicial.

Se puede ver que existe una cota en la cantidad máxima de transiciones que puede hacer $M$ sin leer de la entrada y sin repetir el estado ni el tope de la pila. Para encontrarla, denotemos como $\ell$ al número máximo de símbolos que se escriben en la pila para alguna transición:
$$
    \ell := \max \{|\tau| \mid \exists q, p \in Q, a \in \Sigma, t \in \Gamma : (p, \tau) \in \delta(q, a, t)\}
$$

Luego, el autómata puede escribir hasta $|Q| |\Gamma| \ell$ símbolos en la pila antes de repetir el estado o el tope de la pila. Por ende, la cantidad máxima de configuraciones distintas por las que se pueden pasar sin leer ningún símbolo de la cadena es:
$$
\begin{aligned}
    K & := |Q| \times |\{\tau \in \Gamma^* \mid |\tau| \leq |Q| |\Gamma| \ell\}| \\
    & = |Q| \sum_{i=0}^{|Q| |\Gamma| \ell} |\Gamma|^i \\
    & = |Q| \frac{|\Gamma|^{|Q| |\Gamma| \ell - 1} - 1}{|\Gamma| - 1}
\end{aligned}
$$

Esto implica que es posible detectar cuándo un autómata entra en una configuración ciclante: luego de ejecutarlo por $K + 1$ pasos, o bien repetió alguna configuración (y en tal caso entró en un ciclo infinito), o debe consumir algún símbolo de la entrada.

\subsubsection{APD Continuo}

Un autómata de pila determinístico $M = \langle Q, \Sigma, \Gamma, \delta, q_0, Z_0, F \rangle$ es \textbf{continuo} cuando ``siempre lee toda la cadena de entrada'', es decir, para toda cadena $w \in \Sigma^*$ existen $p \in Q$ y $\alpha \in \Gamma^*$ tales que $(q_0, w, Z_0) \vdash^* (p, \lambda, \alpha)$.

\begin{lemma*}
    Para todo APD $M = \langle Q, \Sigma, \Gamma, \delta, q_0, Z_0, F \rangle$ existe otro APD equivalente y continuo.
\end{lemma*}
\begin{proof}
    Primero, notemos que se puede asumir que $M$ es completo: si tuviera estados que se indefinen para algún símbolo de entrada/pila, se podría agregar un estado trampa al que transicionan.
    
    Se puede tomar el autómata de pila $M' = \langle Q \cup \{f, t\}, \Sigma, \Gamma, \delta', q_0, Z_0, F \cup \{f\} \rangle$, donde la función de transición $\delta'$ ``arregla'' los casos de transiciones ciclantes o incompletas. Para definirla, consideremos los conjuntos:
    $$
    \begin{aligned}
        C_1 & := \{(q, Z) \in Q \times \Gamma \mid (q, \lambda, Z) \text{ es ciclante } \land \nexists g \in F : (\exists \alpha \in \Gamma^* : (q, \lambda, Z) \vdash^* (g, \lambda, \alpha)) \} \\
        C_2 & := \{(q, Z) \in Q \times \Gamma \mid (q, \lambda, Z) \text{ es ciclante } \land \exists g \in F, \alpha \in \Gamma^* : (q, \lambda, Z) \vdash^* (g, \lambda, \alpha)) \}
    \end{aligned}
    $$

    Es decir, $C_1$ es el conjunto de los estados que pertenecen a una secuencia ciclante sin ningún estado final, y $C_2$ es el conjunto de los estados que pertenecen a una secuencia ciclante en la que sí hay algún estado final.

    Luego, podemos definir $\delta'$ de la siguiente manera:
    \begin{itemize}
        \item $\forall (q, Z) \notin (C_1 \cup C_2), \delta'(q, a, Z) = \delta(q, a, Z)\ \forall a \in \Sigma \cup \{\lambda\}$
        \item $\forall (q, Z) \in C_1, \delta'(q, \lambda, Z) = (t, Z)$
        \item $\forall (q, Z) \in C_2, \delta'(q, \lambda, Z) = (f, Z)$
        \item $\forall a \in \Sigma, Z \in \Gamma, \delta'(f, a, Z) = (t, Z) \land \delta'(t, a, Z) = (t, Z)$
    \end{itemize}
\end{proof}

\subsubsection{Demostración de Clausura}

Para demostrar que cualquier LLCD $\L$ está clausurado por complemento, podemos construir un autómata que acepta $\L^c$:

\begin{proof}
    Sea $M = \langle Q, \Sigma, \Gamma, \delta, q_0, Z_0, F \rangle$ un APD continuo tal que $\L(M) = \L$. Luego, podemos tomar el autómata $M' = \langle Q', \Sigma, \Gamma, \delta', q_0', Z_0, F' \rangle$. El conjunto de estados $Q'$ está definido como:
    $$
        Q' = \{[q, k] \mid q \in Q \land k \in \{0, 1, 2\}\}
    $$

    El propósito del número $k$ es detectar si el autómata pasó o no por un estado final desde la última vez que consumió un símbolo. Específicamente:
    \begin{itemize}
        \item Los estados $[q, 0]$ indican que $M$ no pasó por ningún estado final desde la última vez que consumió un símbolo.
        \item Los estados $[q, 1]$ indican que $M$ pasó por algún estado final desde la última vez que consumió un símbolo.
        \item Los estados $[q, 2]$ son los finales, y el aútomata se mueve a éstos luego de pasar por una serie de transiciones-$\lambda$ sin estados finales (es decir, por estados de la forma $[p, 0]$).
    \end{itemize}

    Cuando $M$ está en un $[q, 0]$ y realiza una transición que lee de la entrada, pasa a un $[q, 2]$. Si la ejecución termina ahí, la cadena es aceptada, porque el autómata acaba de pasar por una serie de transiciones-$\lambda$ sin ningún estado final. Por otro lado, si se lee algún símbolo de la entrada, se pasa al estado $[p, 0]$ o $[p, 1]$ según la pertenencia de $p$ a $F$, y se realizan las transiciones-$\lambda$ correspondientes: los estados $[q, 1]$ siempre pasan a otros $[p, 1]$, pero los $[q, 0]$ pasan a $[p, 0]$ cuando $p \notin F$ y $[p, 1] \in F$.

    Formalmente, la función de transición está definida por:
    \begin{itemize}
        \item Si $\delta(q, a, Z) = \{(p, \gamma)\}$, entonces:
        $$
            \delta'([q, 1], a, Z) = \delta'([q, 2], a, Z) =
            \begin{cases}
                \{([p, 0], \gamma)\} & \si p \notin F \\
                \{([p, 1], \gamma)\} & \si p \in F
            \end{cases}
        $$

        Por otro lado, como $\delta(q, a, Z) = \{(p, \gamma)\}$ y $M$ es determinístico, sabemos que $\delta(q, \lambda, Z) = \emptyset$, así que se puede definir la siguiente transición sin violar el determinismo:
        $$
            \delta'([q, 0], \lambda, Z) = \{([p, 2], Z)\}
        $$

        \item Si $\delta(q, \lambda, Z) = \{(p, \gamma)\}$, entonces:
        $$
        \begin{aligned}
            \delta'([q, 2], \lambda, Z) & = \emptyset \\
            \delta'([q, 1], \lambda, Z) & = \{([p, 1], \gamma)\} \\
            \delta'([q, 0], \lambda, Z) & =
            \begin{cases}
                \{([p, 0], \gamma)\} & \si p \notin F \\
                \{([p, 1], \gamma)\} & \si p \in F
            \end{cases}
        \end{aligned}
        $$

        Además, los estados finales y el inicial se definen como:
        $$
        \begin{aligned}
            F & = \{[q, 2] \mid q \notin F\} \\
            q_0' & =
            \begin{cases}
                [q_0, 0] & \si q_0 \notin F \\
                [q_0, 1] & \si q_0 \in F
            \end{cases}
        \end{aligned}
        $$
    \end{itemize}
\end{proof}

\subsection{Otras Propiedades de Clausura}

Los lenguajes libres de contexto determinístico están clausurados por:

\begin{itemize}
    \item \textbf{Complemento}.
    \item $Pre(\L)$ (las cadenas que tienen un prefijo en $\L$).
    \item $Min(\L)$ (las cadenas de $\L$ que \textbf{no tienen} prefijo propio en $\L$).
    \item $Max(\L)$ (las cadenas de $\L$ que no son prefijos de otra cadena en $\L$).
    \item Intersección con lenguaje regular (misma demostración que para LLC general).
    \item Concatenación por derecha con lenguaje regular (se construye el autómata).
\end{itemize}

\subsection{No-Clausura}

Los lenguajes libres de contexto determinístico \textbf{no están} clausurados por:

\begin{itemize}
    \item Intersección (misma demostración que para LLC general).
    \item Unión (si estuvieran, como están clausurados por complemento, $\left(\L_1^c \cup L_2^c\right)^c = \L_1 \cap \L_2$ sería LLCD).
    \item Reverso.
    \item Concatenación.
    \item Clausura de Kleene.
\end{itemize}

\subsection{Algoritmos de Decisión}

Existen algoritmos para:

\begin{itemize}
    \item $\L \overset{?}{=} \emptyset$ (probar las cadenas de longitud menor a la constante de pumping, o computar el conjunto de símbolos de la gramática que derivan en cadenas)
    \item $\L$ es finito? (probar las cadenas con longitud entre $p$ y $2p$)
    \item $\L$ es infinito? (inverso al anterior)
    \item $\L$ es co-finito? (es lo mismo que ``$\L^c$ es finito?'')
    \item $\L = \Sigma^*$ (es lo mismo que $\L^c \overset{?}{=} \emptyset$)
    \item $\L_1 \overset{?}{=} \L_2$
    \item $w \in \L$ (en tiempo lineal)
    \item $\L \overset{?}{=} R$
    \item $\L \overset{?}{\subseteq} R$
\end{itemize}

No hay algoritmos para:
\begin{itemize}
    \item $\L_1 \overset{?}{\subseteq} \L_2$
    \item $\L_1 \cap \L_2 \overset{?}{=} \emptyset$
\end{itemize}


\input{chapters/lenguajes-libres-de-contexto/reescritura-de-gramáticas.tex}


\end{document}
