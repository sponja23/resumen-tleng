\section{Definiciones Básicas}

Primero, algunas definiciones:

\begin{itemize}
    \item Un \textbf{alfabeto} $\Sigma$ es un conjunto finito, no vacío, de \textbf{símbolos}.
    \item Una \textbf{cadena} $\alpha$ es una secuencia finita de símbolos de algún alfabeto: $\alpha = a_1 a_2 ... a_k$.
    \begin{itemize}
        \item La cadena vacía, denotada por $\lambda$, no tiene ningún símbolo.
    \end{itemize}
    \item Una cadena $\alpha = a_1 a_2 ... a_k$ se puede \textbf{concatenar} con un símbolo $b$, formando una nueva cadena $b \cdot \alpha = b a_1 a_2 ... a_k$. A veces se omite el símbolo $\cdot$, dejando sólo la expresión $b \alpha$.
    \item Se puede \textbf{extender} esta definición para permitir concatenar cadenas con otras cadenas:
    $$
    \begin{aligned}
        \alpha \cdot \lambda & := \alpha \\
        \alpha \cdot (b \cdot \beta) & := (\alpha \cdot b) \cdot \beta
    \end{aligned}
    $$
\end{itemize}

\section{Reverso}

Dada una cadena $\alpha = a_1 a_2 ... a_k$, su \textbf{reverso} es la cadena formada por los símbolos de $\alpha$ en el orden inverso: $\alpha^r = a_k a_{k-1} ... a_1$. Se puede definir de forma inductiva:
$$
\begin{aligned}
    \lambda^r & := \lambda \\
    (a \cdot \alpha)^r & := \alpha^r \cdot a
\end{aligned}
$$

Tiene las siguientes propiedades:
\begin{itemize}
    \item Es \textbf{involutiva}: $(\alpha^r)^r = \alpha$.
    \item Es \textbf{antidistributiva} con respecto a la concatenación: $(\alpha \cdot \beta)^r = \beta^r \cdot \alpha^r$.
\end{itemize}

\section{Clausura de Kleene}

Dado un alfabeto $\Sigma$, su \textbf{clausura de Kleene}, denotada $\Sigma^*$, es el conjunto de todas las cadenas formadas por símbolos de $\Sigma$. Se puede definir inductivamente de la siguiente manera:
$$
\begin{aligned}
    & \lambda \in \Sigma^* \\
    \forall a \in \Sigma, \alpha \in \Sigma^*,\ & (\alpha \in \Sigma^* \implies a \alpha \in \Sigma^*)
\end{aligned}
$$
Por otro lado, la \textbf{clausura positiva} de $\Sigma$, denotada por $\Sigma^+$, es el conjunto de todas las cadenas \textbf{no vacías} con símbolos de $\Sigma$. Es decir, $\Sigma^+ = \Sigma^* \setminus \{\lambda\}$.

\section{Lenguajes}

Dado un alfabeto $\Sigma$, un \textbf{lenguaje} $\L$ sobre ese alfabeto es simplemente un conjunto de cadenas de $\Sigma^*$ ($\L \subseteq \Sigma^*$).

\subsection{Operaciones entre Lenguajes}

Si $\L$ y $\L'$ son dos lenguajes definidos sobre el mismo alfabeto $\Sigma^*$, se pueden realizar una serie de operaciones sobre los mismos.

\subsubsection{Operaciones de Conjuntos}

Las operaciones de \textbf{unión}, \textbf{intersección} y \textbf{complemento} son las mismas que se definen para cualquier conjunto:
$$
\begin{aligned}
    \L^c & := \{\alpha \in \Sigma^* \mid \alpha \notin \L\} \\
    \L \cup \L' & := \{\alpha \in \Sigma^* \mid \alpha \in \L \lor \alpha \in \L'\} \\
    \L \cap \L' & := \{\alpha \in \Sigma^* \mid \alpha \in \L \land \alpha \in \L'\} \\
\end{aligned}
$$

\subsubsection{Concatenación}

La \textbf{concatenación} entre los lenguajes $\L$ y $\L'$, denotada $\L \cdot \L'$ o simplemente $\L\L'$, es el lenguaje formado por las cadenas resultantes de concatenar un elemento de $\L$ con uno de $\L'$:
$$\L \cdot \L' := \{\alpha \cdot \beta \mid \alpha \in \L, \beta \in \L'\}$$

Tiene las siguientes propiedades:
\begin{itemize}
    \item Es \textbf{asociativa}: $\L \cdot (\L' \cdot \L'') = (\L \cdot \L') \cdot \L''$
    \item Es \textbf{distributiva} con respecto a la unión: $\L \cdot (\L' \cup \L'') = \L \cdot \L' \cup \L \cdot \L''$
\end{itemize}
